\section{Preliminaries}\label{prelimirares}

In this section we introduce some basic definitions and notations. We
consider that meta-variables defined can appear primed or subscripted.

Meta-variable usage is defined in the paper as follows: $x,y$ denote
term variables, $\alpha, \beta$ ($a, b,...$
in examples) type variables, $e$ a term,
$\tau,\rho$ simple types, $\sigma$ a type, $\Gamma$ a typing context, 
that is, a set of pairs written as $x:\sigma$, and $S$ a
substitution. A constraint is formed by a pair of a class name $C$ and
a sequence of types $\overline{\tau}$. We slighly abuse notation and 
use $\kappa$ to denote both a single constraint and a constraint set.

The notation $\overline{a}^{\,n}$, or simply $\overline{a}$, denotes the
sequence $a_1,\ldots,a_n$, where $n \geq 0$. When used in a context of
a set, it denotes the corresponding set of elements in the sequence
$\{a_1,\ldots,a_n\}$.

A substitution is a function from type variables to simple type
expressions (cf.~Section \ref{Optional-type-classes}). The identity substitution
denoted by \id. $S\sigma$ represents the capture-free operation of
substituting $S(\alpha)$ for each free occurrence of $\alpha$ in
$\sigma$.

We overload the substitution application on constraints, constraint sets
and sets of types. Definition of application on these elements is
straightforward. The symbol $\circ$ denotes function composition and
$\dom{S}=\{\alpha \mid\ S(\alpha) \neq \alpha\}$.

The notation $S[\overline{\alpha}\mapsto\overline{\tau}]$ denotes the
updating of $S$ such that $\overline{\alpha}$ maps to
$\overline{\tau}$, that is, the substitution $S'$ such that $S'(\beta)
= \tau_i$ if $\beta = \alpha_i$, for $i = 1,...,n$, otherwise
$S(\beta)$. Also, $[\overline{\alpha}\mapsto\overline{\tau}] =
\id[\overline{\alpha}\mapsto\overline{\tau}]$.


\subsection{Anti-unification of instance types}
\label{sec:anti-unif}

A type $\tau$ is a generalization --- also called (first-order) 
{\em anti-unification\/} \cite{ModelTheory2012} --- of simple types
$\overline{\tau}^{\,n}$ if there exist substitutions
$\overline{S}^{\,n}$ such that $S_i(\tau)=\tau_i$, for
$i=1,\ldots,n$. We say that $\tau'$ is less general than $\tau$,
written $\tau \leq \tau'$, if there exist $S$ such that $S(\tau) =
\tau'$. The {\it least common generalization} of a set of types
$\tau_i$ is a type $\tau$ such that for all any generalization $\tau'$
of $\tau_i$, we have $\tau \leq \tau'$.

An algorithm for computing the \lcg\ of a finite set of types in
presented in Figure \ref{fig:lcg}. The concept of least common
generalization was studied by Gordon Plotkin
\cite{plotkin1970note,plotkin1971further}, that defined a function for
constructing a generalization of two symbolic
expressions.
In Figure~\ref{fig:lcg}, we present  a function that gives the least
common generalization of a finite set of simple type ($\lcg$). 

\begin{figure*}[ht]
	\[\progfig{
		$\lcg(\mathbb{T})=\tau$ $\:\:\:$ where 
		$(\tau, S)=\lcg'(\mathbb{T},\id)$, for some  $S$ \\ \\
		$\lcg'(\{\tau\},S) = (\tau, S)$  \\ \\		
		$\lcg'(\{\tau_1, \tau_2\} \cup \mathbb{T}, S) = \lcg''(\tau, \tau',S') \:\:\:$ where
		$\begin{array}[t]{ll}
		(\tau, S_0) & = lcg''(\tau_1, \tau_2, S)\\
		(\tau',S')  & = lcg'(\mathbb{T}, S_0)
		\end{array}$  \\ \\		
		xxx\=xxx\=xxx\=xxx\=xxxxx\=xxxxxx\=xxxxxxxx\= \kill
		$\lcg'' (C \: \overline{\tau}^{\,n},\:  D\: \overline{\rho}^{\,m},S)=$\+\\
		\textbf{if}\ $S(\alpha)=( C\:\overline{\tau}^{\,n},\: D\:\overline{\rho}^{\,m})$
		for some $\alpha$ \textbf{then}\ $(\alpha,S)$ \\
		\textbf{else} \+\\
		\textbf{if}\ $n\not=m$ \textbf{then}\
		$(\beta, S [\beta \mapsto ( C \:\overline{\tau}^{\,n},\: D\:\overline{\rho}^{\,m})])$ \+ \\
		where $\beta$ is a fresh type variable \-\\[.1cm]
		\textbf{else}\ $(\psi\: \overline{\tau'}^{\,n}, S_n)$\+\\
		where $\begin{array}[t]{l}
		(\psi,S_0) = \left\{\begin{array}{ll}
		(C ,S) & \textbf{if } C = D \\
		(\alpha, S\,[\alpha\mapsto (C, D)])
		& \text{otherwise, $\alpha$ is fresh }\\
		\end{array}\right. \\[.3cm]
		(\tau'_i,S_i) = lcg''(\tau_i, \rho_i, S_{i-1}), \text{ for } i=1, \ldots, n
		\end{array}$ \-\-\-	
	}
	\]
	\caption{Least Common Generalization} \label{fig:lcg}
\end{figure*}

As examples of least generatizations, consider the following types for
the well known \haskell{map} function for lists and binary trees:
\[
  \begin{array}{c}
       \haskell{(a -> b) -> [a] -> [b]}\\
       \haskell{(a -> b) -> Tree a -> Tree b}
  \end{array}
\]