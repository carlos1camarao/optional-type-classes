\section{Anti-unification of instance types}
\label{sec:anti-unif}

A simple type $\tau$ is a generalization of a set of simple types
$\overline{\tau}^{\,n}$ if there exist substitutions
$\overline{\phi}^{\,n}$ such that $\phi_i(\tau)=\tau_i$, for
$i=1,\ldots,n$. For example, $a_0 \rightarrow a_0$, $a_1 \rightarrow
a_2$, and $a_3$ are generalizations of $\{ \Int \rightarrow \Int,
\Float \rightarrow \Float\}$. A generalization is also called a
(first-order) {\em anti-unification\/} \cite{ModelTheory2012}.

We say that $\tau$ is less general than $\tau'$, written $\tau \leq
\tau'$, if there exists a substitution $\phi$ such that $\phi(\tau') =
\tau$.  For example, $a_0 \rightarrow a_0 \leq a_1 \rightarrow a_2
\leq a_3$.

The {\it least common generalization} (lcg) of a set of types
$S$ and a type $\tau$ holds, written as $\lcgR(S,\tau)$, if, for all generalizations $\tau'$ of
$S$ we have $\tau \leq \tau'$.

The concept of least common generalization was studied by Gordon
Plotkin \cite{plotkin1970note,plotkin1971further}, who defined a
function for constructing a generalization of two symbolic
expressions.  In Figure~\ref{fig:lcg}, we define function \lcg, which
returns a lcg of a finite set of simple types $S$, by recursion on the
structure of $S$, where $\emptyset$ is an empty finite map. Function
$\lcg_2$ computes the generalization of two simple types. For two
types $\tau_1$ and $\tau_2$ the idea is to recursively traverse the
structure of both types using a finite map to store previously
generalized types.

In function $\lcg_2$, $X$ is used to represent either a type
constructor, constant or variable.

Whenever two type expressions of the same arity $\rho_1 =
X_1\:\bar{\tau}_1^n$ and $\rho_2 = X_2\:\bar{\tau}_2^n$ are parameters
of $\lcg_2$, it is checked whether there is an entry that maps $(X_1,
X_2$) to some type variable, i.e.~whether there has been a previous
generalization of $X_1, X_2$. If this is the case, the previous
generalization is returned; otherwise, a fresh type variable is saved
as their generalization. Calls to $\lcg_2$ repeat this process for
each pair of types $*{\tau}_1)_i$, $({\tau}_2)_i$, for
$i=1,\ldots,n$. If two type expressions of distinct arities $X \:
\overline{\tau}^{\,n}$ and $X'\: \overline{\rho}^{\,m}$, where
$n\not=m$, are parameters of $\lcg_2$, then $\lcg_2$ is recursively
called to yield i) the generalization $\tau'$ of $X \:
\overline{\tau}^{\,n-1}$ and $X'\: \overline{\rho}^{\,m-1}$ and ii)
the generalization $\tau''$ of $\tau^{\,n}$ and $\rho^{\,m}$, to yield
$\tau'\:\tau''$, as illustrated in Examples \ref{ex2-lcg} and
\ref{ex-polykindedness}.

In the definition of $\lcg$, the notation used for finite maps $\zeta$
is similar to that used for substitutions. 

\begin{Example}
\normalfont As a first example, consider the following types (of
functions \map\ on lists and trees, respectively):

\progb{
   $(a \rightarrow b)$ $\rightarrow$ [$a$] $\rightarrow$ [$b$]\\
   $(a \rightarrow b)$ $\rightarrow$ \Tree\ $a$ $\rightarrow$ \Tree\ $b$
}

A call of \lcg\ for a set with these types yields type $(a \rightarrow
b) \rightarrow c\:\: a \rightarrow c\:\: b$, where $c$ is a
generalization of type constructors {\tt []} and \Tree\ (for $c$ to be
used in $c\: b$, mapping $({\text{\tt []}},\Tree)\mapsto c$ is saved
in the finite map, so that $c$ can then be reused.

\end{Example}

\begin{Example}
\label{ex2-lcg}
\normalfont
As an example where $\lcg_2$ computes the lcg of two types with type
constructors that have distinct numbers of type arguments, consider.

  \[ \begin{array}[t]{llll}
    \tau_1  & = a  & \rightarrow  & \tau_1' \\
    \tau_1' & = &                 & \IO\: (\IORef\ a) \\
    \tau_2   & = a' & \rightarrow & \tau_2'\\
    \tau_2'  & =    &             & \ST\: s\: (\STRef\ s\: a')
  \end{array}
  \]
%We have that $\tau_1$ can be written as $((\rightarrow)\: a\: (\IO\: (\IORef\: a)))$, and
%  $\tau_2$ as $((\rightarrow)\: a'\: (\ST\: s\: (\STRef\: s\: a')))$).

  We have, letting $S = \{ \tau_1, \tau_2\}$, that $\lcg(S) = \lcg'(S,
  \id) = \lcg_2(\tau_1, \tau_2, \id)$.

The arities of the outermost type constructor, $(\rightarrow)$, of
both $\tau_1$ and $\tau_2$, are equal (to 2), and then we have:
$\lcg_2(\tau_1, \tau_2, \emptyset) = ((\rightarrow)\: a_1\: (b\:
(b_1\: a_1)))$, where $a_1,b,b_1$ are fresh type variables, since:
$(a_1, \zeta_1) = lcg_2(a, a', \emptyset)$, where $\zeta_1 = \emptyset
[(a,a')\mapsto a_1]$, and $(b\: (b_1\: a_1), \zeta_2) = lcg_2(\tau_1',
\tau_2',\zeta_1)$.

The computation of $(b\: (b_1\: a_1), \zeta_2) = lcg_2(\tau_1',
\tau_2', \zeta_1)$ illustrates a case with distinct arities of the
outermost constructors of the two type arguments of $\lcg_2$.  We
have: $lcg_2(\tau_1', \tau_2', \zeta_1) = (b\: (b_1\: a_1), \zeta_2)$
(where $b, b_1$ are fresh type variables), $(b, \zeta_1') = lcg_2(\IO,
\ST\: s, \zeta_1)$, where $\zeta_1' = \zeta_1[(\IO, \ST\: s) \mapsto
  b]$ and $(b_1 \: a_1, \zeta_2) = \lcg_2(\IORef\: a, \STRef\: s\: a',
\zeta_1')$.

The computation of $lcg_2(\IORef\: a, \STRef\: s\: a', \zeta_1')$
involves also a case with distinct arities of type constructors (arity
1 for \IORef\ and 2 for \STRef); we have: $\lcg_2(\IORef\: a, \STRef\:
s\: a', \zeta_1') = (b\: a_1, \zeta_2)$, where $(b, \zeta_2) =
\lcg_2(\IORef, \STRef\: s, \zeta_1')$, $\zeta_2 = \zeta_1'[(\IORef,
  \STRef\: s)\mapsto b]$ and $(a_1, \zeta_2) = \lcg_2(a, a',\zeta_2)$.
     
\end{Example}

The following is another example that illustrates a domain of the
finite used in $\lcg_2$ where the type expressions with distinct
arities in the domain of the finite map (($T$ and $T$ \Int) have the
same outermost type constructor ($T$).

\begin{Example}
  \label{ex-polykindedness}
  
  Consider parameterised algebraic data types, such as e.g.:
  \prog{\data\ $F$ $a$ = $T$ $a$\\
    \data\ $G$ $f$ = $T'$ ($f$ \Int)
    }
  and let $\tau_1 = F\: (T\: \Int)$, $\tau_2 = G\: T$. 

  We have that $\lcg(\{\tau_1,\tau_2\})$ yields the generalization
  $\tau = c\: b$, where $c$ and $b$ are fresh type variables.  We
  have: $\lcg(\{\tau_1, \tau_2\}) = \lcg'(\{\tau_1,\tau_2\},\emptyset)
  = \lcg_2(\tau_1,\tau_2,\emptyset) = (c\:b, \zeta')$, where
  $(c,\zeta) = \lcg_2(F,G,\emptyset)$ and $(b,\zeta') =
  \lcg_2(T,T\:\Int,\zeta)$.

\end{Example}

\begin{figure}[ht]
	\progfig{
            $\lcg(S)=\tau$ $\:\:\:$ where 
               $(\tau, \zeta)=\lcg'(S,\emptyset)$, for some  $\zeta$ \\ \\
            $\lcg'(\{\tau\},\zeta) = (\tau, \zeta)$  \\ \\		
            $\lcg'(\{\tau_1\} \cup S, \zeta) = \lcg_2(\tau_1, \tau',\zeta') \:\:\:$ where
		$\begin{array}[t]{ll}
                   (\tau',\zeta')  & = lcg'(S, \zeta)
		\end{array}$  \\ \\		
            $\lcg_2$\=\ $(X \: \overline{\tau}^{\,n},\:  X'\: \overline{\rho}^{\,m},\zeta)=$\+\\
              {\bf if}\ $\zeta(X\:\overline{\tau}^{\,n},\: X'\:\overline{\rho}^{\,m}) = a$
                      for some $a$ {\bf then}\ $(a,\zeta)$ \\
              {\bf else} \={\bf if}\ $n=m$ {\bf then}\ $(\zeta\: \overline{\tau'}^{\,n}, \zeta_n)$\+\\
                 \hspace*{1cm} where $\begin{array}[t]{ll}
		          (\psi,\zeta_0) & = \left\{\begin{array}{ll}
                                            (T ,\zeta) & {\rm if }\ X = X' \\
                                            (a, \zeta\,[(X, X') \mapsto a])
                                                       & \mbox{\rm{otherwise ($a$ fresh) }}\\
                                          \end{array}\right. \\[.3cm]
                          (\tau'_i,\zeta_i) & = lcg_2 (\tau_i, \rho_i, \zeta_{i-1}), {\rm for }\ i=1, \ldots, n
                        \end{array}$ \\	
                 {\bf else}\ \={\bf if}\ $n=0$ or $m=0$ {\bf then}\=\
                   $(a, \zeta[(X\:\overline{\tau}^{\,n},\: X'\: \overline{\rho}^{\,m}) \mapsto a])$\+\+\\
                 	 where $a$ is a fresh type variable \-\\[.1cm]
                 {\bf else}\ \=$(\tau'\: \tau'', \zeta'')$\+\\
                     where \=$(\tau', \zeta') = \lcg_2 (X\:\overline{\tau}^{\,n-1},\: X'\: \overline{\rho}^{\,m-1}, \zeta)$\+\\
                             $(\tau'',\zeta'') = \lcg_2 (\tau_m, \rho_m, \zeta')$ 
        }
        \vspace{-.2cm}
\caption{Least Common Generalization} \label{fig:lcg}
\end{figure}

The following theorems specify the behavior of \lcg.

\begin{Theorem}[Soundness of \lcg]
For all non-empty sets of simple types $S$, we have that $\lcg(S)$
yields a generalization of $S$.
\label{theorem:lcg-is-sound}
\end{Theorem}

\begin{Theorem}[Completeness of \lcg]
For all non-empty sets of simple types $S$, we have that
$\lcgR(S,\lcg(S))$ holds (i.e.~$\lcg(S)$ is a generalization of $S$)
and, for any $\tau$ that is a generalization of $S$, we have that
$\lcg(S) \leq \tau$.
\label{theorem:lcg-is-complete}
\end{Theorem}

\begin{Theorem}[Compositionality of \lcg]
For all non-empty sets of simple types $S, S'$, we have that
$\lcg(\lcg(S),\lcg(S')) = \lcg(S \cup S')$.
\label{theorem:lcg-is-compositional}
\end{Theorem}

\begin{Theorem}[Uniqueness of \lcg]
For all non-empty sets of simple types $S$, we have that $\lcg(S)$ is
unique, up to variable renaming.
\label{theorem:lcg-is-unique-modulo-variable-renaming}
\end{Theorem}

The proofs use straighforward induction on the number and structural
complexity of elements of $S$.


