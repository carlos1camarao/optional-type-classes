\section{Mini-Haskell with Optional Type Classes}
\label{Optional-type-classes}

In this section we present a type system for mini-Haskell, where type
class declaration is optional. Programmers can overload symbols
without declaring their types in type classes. The type of an
overloaded symbol is, if not explicitly defined in a type class, based
on the anti-unification of instance types defined for the symbol in
the relevant module.

Figure \ref{fig:mini-Haskell-context-free-syntax} shows the
context-free syntax of mini-Haskell: expressions, modules and
programs.  An instance can be specified without specifying a type
class, cf.~second option (after {\tt |}) in Instance Declaration in
Figure \ref{fig:mini-Haskell-context-free-syntax}.


%For simplicity and following common practice, kinds are not considered
%in type expressions and type expressions which are not simple types
%are not explicitly distinguished from simple types. 
%Type expression variables are called simply type variables. 

For simplicity, imported and exported variables and instances must be
explicitly indicated, e.g.~we do not include notations for exporting
and importing all variables of a module.

Multi-parameter type classes are supported. In this paper we do not
consider recursivity, neither in let-bindings nor in instance
declarations. \vspace{-.3cm} 

\begin{figure} 
\begin{mdframed}
\[ \begin{array}[c]{llll}
\textrm{Module Name}        &              & M,N            & \\
\textrm{Program Theory}     &         & P,Q            &\\
\textrm{Variable}           &         & x, y           &\\
\textrm{Expression}         &         & e              & ::= x \,\mid\, \lambda x.\,e  \,\mid\, e\:e' \,\mid\, \mlet x = e\,\iin\,e'\\ 
\textrm{Program}            &         & p              & ::= \overline{m}\\
\textrm{Module}             &         & m              & ::= \module\, M\, (\exportC) \ \where\ \overline{\importC};\: \overline{\!D}\\
\textrm{Export clause}      &         & \exportC\      & ::= \overline{\iitem}\\
\textrm{Import clause}      &         & \importC\      & ::= \import\ M\: (X)\\
\textrm{Item}               &         & \iitem         & ::= x \,\mid\, \instance\ A\: \overline{\tau} \\ 
\textrm{Declaration}        &         & D              & ::= \classDecl \,\mid\, \instDecl \,\mid\, \overline{B}\\
\textrm{Class Declaration}  &         & \classDecl\    & ::= \class\ C \Rightarrow A\: \overline{a}\:\: \where\ \overline{x:\delta}\\
\textrm{Instance Declaration} &       & \instDecl\     & ::= \instance\ C \Rightarrow A\: \overline{\tau}\:\: \where\ \overline{B} 
                                                             \,\mid\, \instance\ B\\
\textrm{Binding}              &       & B              & ::= x = e 
\end{array} \] 
\end{mdframed} \vspace{-.2cm}
\caption{Context-free syntax of mini-Haskell}
\label{fig:mini-Haskell-context-free-syntax}
\end{figure}

A program theory $P$ is a set of axioms of first-order logic,
generated from class and instance declarations occurring in the
program, of the form $C \Rightarrow \pi$, where $C$ is a set of simple
constraints and $\pi$ is a simple constraint (cf.~Figure
\ref{fig:mini-Haskell-context-free-syntax}).

Entailment of a set of constraints $C$ by a program theory $P$ is
written as $P \entail C$ (see
e.g.~\cite{JBCS-Ambiguity-and-constrained-polymorphism}).

Typing contexts are indexed by module names. $\Gamma(M)$ gives a
function on variable names to types: $\Gamma(M)(x)$ gives the type of
$x$ in module $M$ and typing context $\Gamma$.  The notation
$(\Gamma(M),x \mapsto \sigma)$ is used to denote the typing context
$\Gamma'$ that differs from $\Gamma$ only by mapping $x$ to $\sigma$
in module $M$, i.e.~: $\Gamma'(M')(x') = \sigma$ if $M' = M$ and
$x'=x$, otherwise $\Gamma'(M')(x') = \Gamma(M')(x')$.

A special, empty module name, denoted by $\texttt{[]}$, is used for
names exported by modules, to control the scope of names that use
import and export clauses. Also, a reserved name $(\self)$ is used to
refer to the current module, being defined and used in the type system
and relations to control import and export clauses.

It is not necessary to store multiple instance types for the same
variable in a typing context, neither it is necessary to use instance
types in typing contexts (they are needed only in the program theory);
only the lcg of instance types is used, because of lcg
compositionality (theorem \ref{theorem:lcg-is-compositional}). When a
new instance is declared, if it is an instance of a declared class the
type system guarantees that each member is an instance of the type
declared in the type class; otherwise (i.e.~it is the single member of
an undeclared class), its (new) type is given by the lcg of the
existing type (an existing lcg of previous instance types) and the
instance type.

We consider that a constraint set $C'$ can be removed from a
constrained type $C,C' \Rightarrow \tau$ if and only if overloading
for $C'$ has been resolved and there exists a single satisfying
substitution for
$C'$\cite{JBCS-Ambiguity-and-constrained-polymorphism}.

A declarative type system for core-Haskell is presented in Figure
\ref{fig:core-haskell-type-system}, using rules of the form $P;\Gamma
\vdash_0 e:\delta$, which means that $e$ has type $\delta$ in typing
context $\Gamma$ and program theory $P$. 

\begin{figure}
\begin{mdframed}
\[ \begin{array}{cc}
      \displaystyle\frac
        {\begin{array}[t]{lll}
           \Gamma(\self)(x) = (\forall\,\overline{a}.\,C\Rightarrow \tau)\:\:\: & \:\:\:P \entail \phi\,C 
           & \:\:\:\dom{\phi} \subseteq \overline{a}
         \end{array}}
        {P;\Gamma \vdash_0 x: \phi (C \Rightarrow \tau)} \:(\VAR) \\ \\

	\displaystyle\frac
          {(\Gamma(\self),x \mapsto \tau) \vdash_0 e: C \Rightarrow \tau'}
	  {P;\Gamma \vdash_0 \lambda x.\:e: C\Rightarrow \tau \rightarrow \tau'} \:(\ABS)  \\ \\

	\displaystyle\frac
	  {\begin{array}[t]{cc}
             P;\Gamma \vdash_0 e: C \Rightarrow \tau' \rightarrow \tau\: &\:
             P;\Gamma \vdash_0 e': C' \Rightarrow \tau' \\
             V = \tv(\tau) \cup \tv(C) & (C \oplus_V C') \simplifies{P} C''
        \end{array}}
	{P;\Gamma \vdash_0 e\:e': C'' \Rightarrow \tau} \:(\APP) \\ \\

	\displaystyle\frac
	 {\begin{array}{ll}
            P;\Gamma \vdash_0 e\!:C \Rightarrow \tau & C \simplifies{P} C'' \\
            \gen(C'' \Rightarrow \tau,\sigma,\tv(\Gamma))\: & \:P;(\Gamma(\self),x \mapsto\sigma) \vdash_0 e'\!:\,C' \Rightarrow \tau'
          \end{array}}
	 {P;\Gamma \vdash_0 \mlet\ x=e\ \iin\ e': C' \Rightarrow\tau' } \:(\LET)
\end{array} \]
\end{mdframed} \vspace{-.2cm}
\caption{Core-Haskell Type System} 
\label{fig:core-haskell-type-system}
\end{figure}

Rule (\LET) performs constraint set simplification before type
generalization. Constraint set simplification $\simplifies{P}$ is a
relation on cons\-traints, defined as a composition of improvement and
context reduction \cite{JBCS-Ambiguity-and-constrained-polymorphism}.
$\gen(\delta,\sigma,V)$ holds if
  $\sigma=\forall\,\overline{a}.\,\delta$, where
  $\overline{a}=\tv(\delta) - V$; similarly, for constraints, 
$\gen(C\Rightarrow\pi,\theta,V)$ holds if
$\theta=\forall\,\overline{a}.\,C\Rightarrow\pi$, where
$\overline{a}=\tv(C\Rightarrow\pi) - V$. 

$C \oplus_V C'$ denotes the constraint set obtained by adding to $C$
constraints from $D$ that have type variables reachable from $V$:
  \[ C \oplus_V C' = C \cup \{ \pi \in C'\,\mid\, \tv(\pi) \cap \reachableVars(C',V) \not= \emptyset \} \]

In rule (\APP), the constraints on the type of the result are those
that occur in the function type plus not all constraints that occur in
the type of the argument but only those that have variables reachable
from the set of variables that occur in the simple type of the result
or in the constraint set on the function type (cf.~Definition
\ref{def:reachable}).  This allows, for example, to eliminate
constraints on the type of the following expressions, where $o$ is any
expression, with a possibly non-empty set of constraints on its type:
{\tt \flip\ \const\ $o$} (where \const\ has type $\forall a, b.\,a
\rightarrow b \rightarrow a$ and \flip\ has type $\forall a, b, c.\,(a
\rightarrow b \rightarrow c) \rightarrow b \rightarrow a\rightarrow
c$), which should denote an identity function, and \fst\ ($e$, $o$),
which should have the same denotation as $e$.

The extension of core-Haskell to mini-Haskell, which allows (optional)
type classes, modules and modularized instance declarations, is
presented in Figures \ref{fig:mini-haskell-module-rule} through
\ref{fig:mini-haskell-rules-for-declarations}. Rule (\MODULE), in
Figure \ref{fig:mini-haskell-module-rule}, uses relations
($\vdash_{\Downarrow}$) and ($\vdash_{\Uparrow}^X$), which are defined
separately, for clarity, in Figures \ref{fig:import-relation} and
\ref{fig:mini-haskell-rules-for-declarations}.

The import relation $\Gamma \vdash_{\Downarrow} \overline{\importC} :
\Gamma'$ yields a typing context ($\Gamma'$) from a typing context
($\Gamma$) and a sequence of import clauses ($\overline{\importC}$).

Relation $P;\Gamma \vdash_{\Uparrow}^X \overline{\!D}:(E,P',\Gamma')$
is used for specifying the types of a sequence of bindings, from a
typing context ($\Gamma$), a program theory ($P$) and a set of
exported items ($X$); it yields the set ($E$) of exported variables
with their types, together with both i) a new typing context
($\Gamma'$), modified to contain elements of $E$, so that
$\Gamma'(\texttt{[]})$ contains the types of each $x\in E$, and ii) a
new program theory ($P'$), updated from class and instance
declarations. Relation $(\vdash_0)$ is used to check that expressions
of core-Haskell that occur in declarations are well-typed.

There must exist a sequence of derivations for typing a sequence of
modules that composes a program that starts from an empty typing
context, or from a typing context that corresponds to predefined
library modules. Recursive modules are not treated in this paper.

\begin{figure}
\begin{mdframed}
\[ \begin{array}{cc}
	\displaystyle\frac
	 {\begin{array}{ll}
           \Gamma_0 \vdash_{\Downarrow} \overline{\!I} : \Gamma\:\: & \:\:P;\Gamma \vdash_{\Uparrow}^X \overline{\!D} : (E,P',\Gamma') 
          \end{array}}
	 {P;\Gamma_0 \vdash \module\ M\, (\exportC)\ \where\ \overline{\!I};\, \overline{\!D} : (E,P',\Gamma')} \:(\MODULE)
\end{array} \]
\end{mdframed} \vspace{-.2cm}
\caption{Mini-Haskell module rule} 
\label{fig:mini-haskell-module-rule}
\end{figure}

\begin{figure}
\begin{mdframed}
\[ \begin{array}{cc}
	\displaystyle\frac
	 {\begin{array}{ll}
            \Gamma'(M)(x) = \left\{ \begin{array}{ll}
              \Gamma(\texttt{[]})(x) & \text{ if } M = \self \text{ and, for some $1 \leq k \leq n$, }\\
                                     & %\hspace*{.3cm} 
                                         x\hspace*{-.05cm} =\hspace*{-.05cm} \iota_k \text{ or 
                                         ($\iota_k\hspace*{-.05cm} =\hspace*{-.05cm} \instance\ A\: \overline{\tau}$, $x$ is a member of class $A$)}\\
               \Gamma(M)(x)          & \text{ otherwise}
             \end{array}\right.
          \end{array}}
	 {\Gamma \vdash_{\Downarrow} \import\ M\, (\,\overline{\iitem}^{\,n}\,) : \Gamma'}  \\\ \\

	\displaystyle\frac
	 {\begin{array}{ll}
	   \Gamma_0 \vdash_{\Downarrow} \import\ M\, (\,\overline{\iitem}\,) : \Gamma \:\:\: & \:\:\: 
           \Gamma \vdash_{\Downarrow} \overline{\importC} : \Gamma'
          \end{array}}
	 {\Gamma_0 \vdash_{\Downarrow} \import\ M\, (\,\overline{\iitem}\,); \overline{\importC} : \Gamma'} 
\end{array} \]
\end{mdframed} \vspace{-.2cm}
\caption{Import relation}
\label{fig:import-relation}
\end{figure}

\begin{figure}
\begin{mdframed}
\[ \begin{array}{cc}
	\displaystyle\frac
	 {\begin{array}{ll}
            Q;\Gamma \vdash_{\Uparrow}^X \overline{\!D} : (E, Q', \Gamma') \:\: & \:\:
            Q = P \cup \left\{ \begin{array}{ll}
                                  \{ C \Rightarrow A\:\overline{a} \} & \text{ if } C \not= \emptyset \\
                                  \emptyset                              & \text{ otherwise}
                                \end{array}\right. \\
            \multicolumn{2}{c}{
                \Gamma(M)(x) = \left\{ \begin{array}{ll}
                                 \delta_k       & \text{ if } x = x_k, 1 \leq k \leq n, \text{ and } 
                                                              M \in \{ \self, \texttt{[]} \}\\
                                 \Gamma_0(M)(x) & \text{ otherwise} 
                               \end{array}\right. }
          \end{array}}
	 {P;\Gamma_0 \vdash_{\Uparrow}^X \class\ C \Rightarrow A\: \overline{a}\ \where\ \overline{x:\delta}^{\,n};\: \overline{\!D} : 
            (E, Q', \Gamma') } \\\ \\

	\displaystyle\frac
	 {\begin{array}{l}
               P \entail \phi(C \Rightarrow \pi) \:\:\:
               \gen(\phi(C\Rightarrow \pi), \theta, \tv(\Gamma)) \:\:\:
               Q = P \cup \{ \theta \} \:\:\: \\[.2cm]
               Q;\Gamma \vdash_0 e_i : \delta_i \:\:\:\:\: 
               \delta_i = \phi\bigl(\Gamma(\texttt{[]})(x_i)\bigr), \text{ for } i=1,\ldots, n\\[.2cm]
               Q;\Gamma \vdash_{\Uparrow}^{X'} \overline{\!D} : (E,Q',\Gamma') \\[.2cm]
               (X',E') = \left\{ \begin{array}{ll}
                               (X - \{ \iota \}, E \cup \{ \overline{x : \delta}^{\,n} \}) 
                                   & \text{if } \iota \in X, \iota = \instance\ A\: \tau \\
                               (X, E) & \text{otherwise} 
                             \end{array}\right.
          \end{array}}
	 {P;\Gamma \vdash_{\Uparrow}^X \instance\ \phi(C \Rightarrow \pi)\ \where\ \overline{x = e}^{\,n};\: \overline{\!D}
             : (E',Q',\Gamma')}\\\ \\

	\displaystyle\frac
	 {\begin{array}{l}
	    A \text{ is the class name generated for } x \\[.1cm]

            P; \Gamma_0 \vdash_0 e : C \Rightarrow \tau \:\:\:
               \gen(C \Rightarrow A\, \tau, \theta, \tv(\Gamma_0)) \:\:\:  
                Q = P \cup \{ \theta \} \\[.1cm]

           Q;\Gamma \vdash_{\Uparrow}^{X'} \overline{\!D} : (E,Q',\Gamma') \:\:\:\:
            \lcgR(\{\{ \tau\} \cup \{ \Gamma_0(\self)(x)\},\tau')  \\[.1cm]

           P;\Gamma \vdash_{\Uparrow}^{X'} \overline{\!D} : (E,Q',\Gamma') \\[.2cm]
           \Gamma(M)(y) = \left\{ \begin{array}{ll}
                                      \tau' & \text{ if } y = x, (M = \self \text{ or }
                                                                 (M = \texttt{[]}, x \in X)) \\
                                      \Gamma_0(M)(y) & \text{ otherwise } 
                                   \end{array}\right. \\[.3cm] 
           (X',E') = \left\{ \begin{array}{ll}
                               (X - \{ x : C\Rightarrow \tau\}, E \cup \{ x : C\Rightarrow \tau \}) 
                                   & \text{if } (x : C\Rightarrow \tau) \in X\\
                               (X, E) & \text{otherwise} 
                             \end{array}\right. 
          \end{array}}
	 {P;\Gamma_0 \vdash_{\Uparrow}^X \instance\ x = e;\: \overline{\!D} : (E',Q',\Gamma')} \\\ \\

	\displaystyle\frac
	 {\begin{array}{l}
           P;\Gamma_0 \vdash_0 e : C\Rightarrow\tau \:\:\:\:\: \gen(C\Rightarrow\tau,\sigma,\tv(\Gamma_0)) \\[.2cm]
           P;\Gamma \vdash_{\Uparrow}^X \overline{\!D} : (E,P',\Gamma')\\[.1cm]

           \Gamma(M)(y) = \left\{ \begin{array}{ll}
                                    \sigma & \text{ if } y = x, (M = \self \text{ or }
                                                                 (M = \texttt{[]}, x \in X)) \\
                                      \Gamma_0(M)(y) & \text{ otherwise } 
                                   \end{array}\right.\\[.3cm]                   
           (X',E') = \left\{ \begin{array}{ll}
                               (X - \{ x \}, E \cup \{x : C\Rightarrow\tau\}) 
                                   & \text{if } x \in X\\
                               (X, E) & \text{otherwise} 
                             \end{array}\right. 
          \end{array}}
	 {P;\Gamma_0 \vdash_{\Uparrow}^X x = e;\: \overline{\!D} : (E',P',\Gamma')} 
\end{array} \]
\end{mdframed} \vspace{-.2cm}
\caption{Mini-Haskell rules for declarations}
\label{fig:mini-haskell-rules-for-declarations}
\end{figure}

The first and second rules in
Figure~\ref{fig:mini-haskell-rules-for-declarations} specify the
bindigs generated by standard Haskell type classes and instance
declarations, respectively.

The third one is the rule which accounts for instance declarations of
an overloaded symbol $x$ whose type is not explicitly specified in a
type class. As stated previously, the type $\tau'$ of $x$ is the least
common generalization of the set of types $\{\tau\} \cup
\{\Gamma_0(\self)(x)\}$, where $\tau$ is the type of the expression in
the current instance declaration for $x$ and $\Gamma_0(\self)(x)$ is
the type of $x$ in the current type environment (previously computed
from other instance declarations for $x$ that are visible in
$\Gamma_0$). This rule is based on Theorem
\ref{theorem:lcg-is-compositional}.
