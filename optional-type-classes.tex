\section{Mini-Haskell with Optional Type Classes}
\label{Optional-type-classes}

In this section we present a type system for mini-Haskell, where type
class declaration is optional. Programmers can overload symbols
without declaring their types in type classes. The type of an
overloaded symbol is, if not explicitly defined in a type class, based
on the anti-unification of instance types defined for the symbol in
the relevant module.

Figure \ref{fig:mini-Haskell-context-free-syntax} shows meta-variable
usage and the context-free syntax of mini-Haskell: expressions and
their types, modules and programs. Meta-variables can be possibly
primed or subscripted. An instance can be specified without specifying
a type class. 

% (the use of the keywork \instance\ is not strictly
%necessary, it is included here for the sake of simplicity).

For simplicity and following common practice, kinds are not considered
in type expressions and type expressions which are not simple types
are not explicitly distinguished from simple types. Type expression
variables are called simply type variables. 

As usual, we assume the existence of type constructor $\to$, that is
written as an infix operator ($\tau \to \tau'$). A type
$\forall\,\overline{a}.\,C\Rightarrow \tau$ is equivalent to
$C\Rightarrow \tau$ if $\overline{a}$ is empty and, similarly,
$C\Rightarrow \tau$ is equivalent to $\tau$ if $C$ is empty.

For simplicitly, imported and exported variables and instances must be
explicitly indicated, e.g.~we do not include notations for exporting
and importing all variables of a module.

Multi-parameter type classes are supported. In this papert we do not
consider recursivity, neither in let-bindings nor in instance
declarations. 

\begin{figure} 

\[ \begin{array}[c]{llll}
\textrm{Class Name}         &\hspace{.1cm} & A,B            & \\
\textrm{Module Name}        &              & M,N            & \\
\textrm{Type variable}      &         & a,b,\alpha,\beta & \\
\textrm{Type constructor}   &         & T              & \\
\textrm{Simple Constraint}  &         & \pi            & ::= A\,\overline{\tau}\\
\textrm{Unquantified Constraint} &    & \psi           & ::= C\Rightarrow \pi\\
\textrm{Constraint}         &         & \theta         & ::= \forall\,\overline{\alpha}.\,\psi\\
\textrm{Set of Simple Constraints} &  & C,D            & \\
\textrm{Constrained Type}   &         & \delta,\epsilon & ::= C\Rightarrow \tau\\
\textrm{Simple Type}        &         & \tau, \rho     & ::= \alpha \mid T \mid \tau\:\tau' \\
\textrm{Type}               &         & \sigma         & ::= \forall\,\overline{\alpha}.\,\delta\\
\textrm{Program Theory}     &         & P,Q            &\\
\textrm{Variable}           &         & x, y, z        &\\
\textrm{Expression}         &         & e              & ::= x \,\mid\, \lambda x.\,e  \,\mid\, e\:e' \,\mid\, \mlet x = e\,\iin\,e'\\ 
\textrm{Program}            &         & p              & ::= \overline{m}\\
\textrm{Module}             &         & m              & ::= \module\, M\, (\exportC) \ \where\ \overline{\importC};\: \overline{\!d}\\
\textrm{Export clause}      &         & \exportC\      & ::= \overline{\iitem}\\
\textrm{Import clause}      &         & \importC\      & ::= \import\ M\: (\,\overline{\iitem}\,)\\
\textrm{Item}               &         & \iitem         & ::= x \\ % \,\mid\, \instance\ A\: \overline{\tau}\\ 
\textrm{Declaration}        &         & d              & ::= \classDecl \,\mid\, \instDecl \,\mid\, \overline{b}\\
\textrm{Class Declaration}  &         & \classDecl\    & ::= \class\ C \Rightarrow A\: \overline{a}\:\: \where\ \overline{x:\delta}\\
\textrm{Instance Declaration} &       & \instDecl\     & ::= \instance\ C \Rightarrow A\: \overline{\tau}\:\: \where\ \overline{b} 
                                                             \,\mid\, \instance\ b\\
\textrm{Binding}              &       & b              & ::= x = e 
\end{array} \]
\caption{Context-free syntax of mini-Haskell and types}
\label{fig:mini-Haskell-context-free-syntax}
\end{figure}

A program theory $P$ is a set of axioms of first-order logic,
generated from class and instance declarations occurring in the
program, of the form $C \Rightarrow \pi$, where $C$ is a set of simple
constraints and $\pi$ is a simple constraint (cf.~Figure
\ref{fig:mini-Haskell-context-free-syntax}).

Entailment of a set of constraints $C$ by a program theory $P$ is
written as $P \entail C$ (see
e.g.~\cite{JBCS-Ambiguity-and-constrained-polymorphism}).

Typing contexts are indexed by module names. $\Gamma(M)$ gives a
function on variable names to types: $\Gamma(M)(x)$ gives the type of
$x$ in module $M$ and typing context $\Gamma$.
   
A special, empty module name, denoted by $\texttt{[]}$, is used for
names exported by modules, to control the scope of names that use
import and export clauses. Also, a reserved name $\gamma$ is used to
refer to the current module, being defined and used in the type system
and relations to control import and export clauses.

It is not necessary to save multiple instance types for the same
variable in a typing context, neither it is necessary to use instance
types in typing contexts (they are needed only in the program theory);
only the lcg of instance types is used, because of lcg
compositionality (theorem \ref{theorem:lcg-is-compositional}). When a
new instance is declared, if it is an instance of a declared class the
type systems guarantees that each field is an instance of the type
declared in the type class; otherwise (i.e.~it is the single field of
an undeclared class), its (new) type is given by the lcg of the
existing type (an existing lcg of previous instance types) and the
instance type.

A partial order on possibly constrained and possibly quantified types,
based on constraint set entailment, is defined in Figure
\ref{fig:type-partial-order}. We use $P \entail C$ to abbreviate
$P\entail \pi$ for all $\pi\in C$.  Note that type ordering disregards
constraint set satisfiability.  Satisfiability is only important when
considering whether a constraint set $C$ can be removed from a
constrained type $C,D \Rightarrow \tau$ ($C$ can be removed if and
only if overloading for $C$ has been resolved and there exists a
single satisfying substitution for
$C$)\cite{JBCS-Ambiguity-and-constrained-polymorphism}.

\begin{figure}
   \[ \begin{array}{ccc}
   	\displaystyle\frac
          {P \entail \phi\,C \:\:\:\:\: D \subseteq \phi\,C \:\:\:\:\: \overline{b} \subseteq \tv(D) \cup \tv(\phi\,\tau) }
          {\forall\,\overline{a}.\,C \Rightarrow \tau \leq_P \forall\,\overline{b}.\,D \Rightarrow \phi\,\tau}
  \end{array} \]
\caption{Partial order on Types}
\label{fig:type-partial-order}
\end{figure}

A type system for core-Haskell is presented in Figure
\ref{fig:core-haskell-type-system}, using rules of the form $P;\Gamma
\vdash_0 e:\delta$, which means that $e$ has type $\delta$ in typing
context $\Gamma$ and program theory $P$.

Rule (\LET) performs constraint set simplification before type
generalization. Constrait set simplification $\simplifies{P}$ is a
relation on cons\-traints, defined as a composition of improvement and
context reduction \cite{JBCS-Ambiguity-and-constrained-polymorphism}.

In rule (\APP), the constraints on the type of the result are those
that occur in the function type plus not all constraints that occur in
the type of the argument but only those that have variables reachable
from the set of variables that occur in the simple type of the result
or in the constraint set on the function type.  This allows, for
example, not including constraints on the type of the following
expressions, where $o$ is any expression, with a possibly non-empty
set of constraints on its type: {\tt \flip\ \const\ $o$} (where
\const\ has type $\forall a, b.\,a \rightarrow b \rightarrow a$ and
\flip\ has type $\forall a, b, c.\,(a \rightarrow b \rightarrow c)
\rightarrow b \rightarrow a\rightarrow c$), which should denote an
identity function, and \fst\ ($e$, $o$), which should have the same
denotation as $e$.

A variable $a\in \tv(C)$ is called reachable from, or with respect to,
a set of type variables $V$ if $a\in V$ or if $a\in \pi$ for some
$\pi\in C$ such that there exists $b\in \tv(\pi)$ such that $b$ is
reachable. $a\in \tv(C)$ is called unreachable if it is not
reachable. The set of reachable and unreachable type variables of
constraint set $C$ {\em from V\/} is denoted by $\reachableVars(C,V)$.

%We also say that type variables $W$ are reachable in constrained type
%$C\Rightarrow \tau$ if $W\subseteq \reachableVars(C,\tv(\tau))$ (and
%similarly for unreachable type variables and if $W$ is a type variable
%instead of a set of type variables).

$C \oplus_V D$ denotes the constraint set obtained by adding to $C$
constraints from $D$ that have type variables reachable from $V$:
  \[ P \oplus_V Q = P \cup \{ \psi \in Q \mid \tv(\psi) \cap \reachableVars(Q,V) \not= \emptyset \} \]

$\gen(\delta,\sigma,V)$ holds if
  $\sigma=\forall\,\overline{\alpha}.\,\delta$, where
  $\overline{\alpha}=\tv(\delta) - V$.

A type system for mini-Haskell, which extends core-Haskell with
(optional) type classes, modules and modularized instance
declarations, is presented in Figures
\ref{fig:mini-haskell-module-rule} and
\ref{fig:mini-haskell-rules-for-declarations}. Rule (\MODULE) uses
relations ($\vdash_{\Downarrow}$) and ($\vdash_{\Uparrow}^X$), which
are defined separately, for clarity, in Figures
\ref{fig:import-relation} and
\ref{fig:mini-haskell-rules-for-declarations}.

The import relation $\Gamma \vdash_{\Downarrow} \overline{\importC}
\leadsto \Gamma'$ yields a typing context ($\Gamma'$) from a typing
context ($\Gamma$) and a sequence of import clauses
($\overline{\importC}$).

Relation $P;\Gamma \vdash_{\Uparrow}^X \overline{\!d}:(R,P',\Gamma')$
is used for specifying the types a sequence of bindings, from a typing
context ($\Gamma$), a program theory ($P$) and a set of exported names
($X$); it yields a record of elements ($R)$ of the form $(x:\sigma)$,
together with both i) a new typing context ($\Gamma'$), modified to
contain elements of $X$, so that $\Gamma'(\texttt{[]})$ contains the
types of each $x\in X$, and ii) a new program theory ($P'$), updated
from class and instance declarations.  Relation $(\vdash_0)$ is used
to check that expressions of core-Haskell that occur in declarations
are well-typed.

% and updates the typing context if these declarations are exported,
%and updates the program theory from class and instance declarations.

There must exist a sequence of derivations for typing a sequence of
modules that composes a program that starts from an empty typing
context, or from a typing context that corresponds to predefined
library modules. Recursive modules are not treated in this paper.

$C \oplus_V D$ denotes the constraint set obtained by adding to $C$
constraints from $D$ that have type variables reachable from $V$:
  \[ P \oplus_V Q = P \cup \{ \psi \in Q \mid \tv(\psi) \cap \reachableVars(Q,V) \not= \emptyset \} \]

Relation $\oplus$ is also used in Figure
\ref{fig:mini-haskell-rules-for-declarations} for concatenation of
record elements: $\{ x_1, \ldots, x_n \} \oplus \{ y_1, \ldots, y_m \}
= \{ x_1, \ldots, x_n, y_1, \ldots, y_m \}$.

%If $R = \overline{x:\delta}^{\,n}$ and $R'
% =\overline{y:\epsilon}^{\,m}$, then $R \oplus R'$ denotes the record
% $\{ x_1,:\delta_1, \ldots, x_n:\delta_n, y_1:\epsilon_1, \ldots,
% y_m:\delta_m \}$.

\begin{figure}
\[ \begin{array}{cc}
      \displaystyle\frac
        {\begin{array}[t]{lll}
           \Gamma(\gamma)(x) = \sigma\ & \:\sigma \leq_P \delta
         \end{array}}
        {P;\Gamma \vdash_0 x: \delta} \:(\VAR) \\ \\

	\displaystyle\frac
          {(\Gamma(\gamma),x \mapsto \tau) \vdash_0 e: C \Rightarrow \tau'}
	  {P;\Gamma \vdash_0 \lambda x.\:e: C\Rightarrow \tau \rightarrow \tau'} \:(\ABS)  \\ \\

	\displaystyle\frac
	  {\begin{array}[t]{cc}
             P;\Gamma \vdash_0 e: C \Rightarrow \tau' \rightarrow \tau\: &\:
             P;\Gamma \vdash_0 e': C' \Rightarrow \tau' \\
             V = \tv(\tau) \cup \tv(C) & (C \oplus_V C') \simplifies{P} D
           \end{array}}
	{P;\Gamma \vdash_0 e\:e': D\Rightarrow \tau} \:(\APP) \\ \\

	\displaystyle\frac
	 {\begin{array}{ll}
            P;\Gamma \vdash_0 e\!:C \Rightarrow \tau & C \simplifies{P} D\\
             \gen(D\Rightarrow \tau,\sigma,\tv(\Gamma))\: & \:P;(\Gamma(\gamma),x \mapsto\sigma) \vdash_0 e'\!:\delta
          \end{array}}
	 {P;\Gamma \vdash_0 \mlet\ x=e\ \iin\ e':\delta} \:(\LET)
\end{array} \]
\caption{Core-Haskell Type System}
\label{fig:core-haskell-type-system}
\end{figure}

\begin{figure}
\[ \begin{array}{cc}
	\displaystyle\frac
	 {\begin{array}{ll}
           \Gamma_0 \vdash_{\Downarrow} \overline{\!I} : \Gamma\:\: & \:\:P;\Gamma \vdash_{\Uparrow}^X \overline{\!d} : (R,P',\Gamma') \}
          \end{array}}
	 {P;\Gamma_0 \vdash \module\ M\, (\exportC)\ \where\ \overline{\!I};\, \overline{\!d} : R} \:(\MODULE)
\end{array} \]
\caption{Mini-Haskell module rule} 
\label{fig:mini-haskell-module-rule}
\end{figure}

\begin{figure}
\[ \begin{array}{cc}
	\displaystyle\frac
	 {\begin{array}{ll}
            \Gamma'(M)(x) = \left\{ \begin{array}{ll}
              \Gamma(\texttt{[]})(x) & \text{ if } M = \gamma \text{ and } x = \iota_k, %\text{ for some } 
                                                                           1 \leq k \leq n\\
%                                     &                                    & x \text{ is the single member of such } \iota_k)\\
               \Gamma(M)(x)          & \text{ otherwise}
             \end{array}\right.
          \end{array}}
	 {\Gamma \vdash_{\Downarrow} \import\ M\, (\,\overline{\iitem}^{\,n}\,) : \Gamma'} \\ \\

	\displaystyle\frac
	 {\begin{array}{ll}
	   \Gamma_0 \vdash_{\Downarrow} \import\ M\, (\,\overline{\iitem}\,) : \Gamma \:\:\: & \:\:\: 
           \Gamma \vdash_{\Downarrow} \overline{\importC} : \Gamma'
          \end{array}}
	 {\Gamma_0 \vdash_{\Downarrow} \import\ M\, (\,\overline{\iitem}\,); \overline{\importC} : \Gamma'} 
\end{array} \]
\caption{Import relation}
\label{fig:import-relation}
\end{figure}

\begin{figure}
\[ \begin{array}{cc}
	\displaystyle\frac
	 {\begin{array}{ll}
            Q;\Gamma \vdash_{\Uparrow}^X \overline{\!d} : (R, Q', \Gamma') \:\: & \:\:
            Q = P \cup \left\{ \begin{array}{ll}
                                  \{ C \Rightarrow A\:\overline{a} \} & \text{ if } C \not= \emptyset \\
                                  \emptyset                              & \text{ otherwise}
                                \end{array}\right. \\
            \multicolumn{2}{c}{
                \Gamma(M)(x) = \left\{ \begin{array}{ll}
                                 \delta_k       & \text{ if } x = x_k, 1 \leq k \leq n, \text{ and } 
                                                              M \in \{ \gamma, \texttt{[]} \}\\
                                 \Gamma_0(M)(x) & \text{ otherwise} 
                               \end{array}\right. }
          \end{array}}
	 {P;\Gamma_0 \vdash_{\Uparrow}^X \class\ C \Rightarrow A\: \overline{a}\ \where\ \overline{x:\delta}^{\,n};\: \overline{\!d} : 
            (R, Q', \Gamma') } \\ \\

	\displaystyle\frac
	 {\begin{array}{ll}
               Q = P \cup \left \{ \begin{array}{ll}
                         \{ \forall\,\overline{a}.\,C \Rightarrow \pi \} & \text{ if } C \not= \emptyset \\
                         \{ \forall\,\overline{a}.\,\pi \} & \text{ otherwise}
                                \end{array}\right. \: & \: 
               P \entail C \Rightarrow \pi \\
%           \multicolumn{2}{c}{\gen(\delta_k,\sigma_k,\tv(\Gamma_0)), \text{ for } 1\leq k \leq n} \\
%           \multicolumn{2}{c}{
%             \Gamma(M)(y) = \left\{ \begin{array}{ll}
%                                    \sigma_k & \text{ if } y = x_k \text{ and } (M = \gamma \text{ or } 
%                                                                                 M = \texttt{[]} \text{ and } x_k \in X)\\
%                                    \Gamma_0(M)(x) & \text{ otherwise} 
%                            \end{array}\right.}\\[.3cm]
           Q;\Gamma \vdash_0 \overline{e}^{\,n} : \overline{\delta}^{\,n} \: & \: 
	   Q;\Gamma \vdash_{\Uparrow} \overline{\!d} : (R,Q',\Gamma') 
          \end{array}}
	 {P;\Gamma \vdash_{\Uparrow}^X \instance\ C \Rightarrow \pi\ \where\ \overline{x = e}^{\,n};\: \overline{\!d} 
             : (\overline{x : \delta}^{\,n} \oplus R,Q',\Gamma')}\\ \\

	\displaystyle\frac
	 {\begin{array}{ll}
            P; \Gamma \vdash_0 e : C \Rightarrow \tau \:\: & \:\:
            \lcgR(\{ \tau\} \cup \Gamma_0(\gamma)(x), \tau') \\[.1cm]
            \multicolumn{2}{c}{\Gamma(M)(y) = \left\{ \begin{array}{ll}
                                      \tau' & \text{ if } y = x, (M = \gamma \text{ or }
                                                                 (M = \texttt{[]}, x \in X)) \\
                                      \Gamma_0(M)(y) & \text{ otherwise } 
                                   \end{array}\right.}\\[.2cm]                   
           \multicolumn{2}{c}{P;\Gamma \vdash_{\Uparrow}^X \overline{\!d} : (R,Q',\Gamma')} 
          \end{array}}
	 {P;\Gamma_0 \vdash_{\Uparrow}^X \instance\ x = e;\: \overline{\!d} : (\{x:C\Rightarrow \tau\} \oplus R,Q',\Gamma')} \\ \\

	\displaystyle\frac
	 {\begin{array}{ll}
           P;\Gamma_0 \vdash_0 e : \delta \: & \: \gen(\delta,\sigma,\tv(\Gamma_0))\\
           \multicolumn{2}{c}{\Gamma(M)(y) = \left\{ \begin{array}{ll}
                                    \sigma & \text{ if } y = x, (M = \gamma \text{ or }
                                                                 (M = \texttt{[]}, x \in X)) \\
                                      \Gamma_0(M)(y) & \text{ otherwise } 
                                   \end{array}\right. }\\[.2cm]                   
           \multicolumn{2}{c}{P;\Gamma \vdash_{\Uparrow}^X \overline{\!d} : (R,P',\Gamma')}
          \end{array}}
	 {P;\Gamma_0 \vdash_{\Uparrow}^X x = e;\: \overline{\!d} : (\{x:\delta\} \oplus R,P',\Gamma')} 
\end{array} \]
\caption{Mini-Haskell rules for declarations}
\label{fig:mini-haskell-rules-for-declarations}
\end{figure}

