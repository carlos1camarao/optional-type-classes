\subsubsection{Improvement}
\label{sec:improvement}

Improvement removes constraints with unreachable type variables from a
constraint $C$ that occurs on a constrained type $C\Rightarrow \tau$,
based on constraint set entailment: improvement consists of removing
each constraint in $C$ that has unreachable type variables and for
which there exists a single entailing substitution. Improvement is
defined in Figure \ref{fig:constraint-set-improvement}.
If the set $\mathbb{S}$ of entailed instances of
$\unreachableVars(C,\tv(\tau))$ has more than one element, or if it is
empty, there is no improved constraint (improvement is a partial
relation).

%Improvement as defined in this paper is not a satisfiability
%preserving relation. The satisfiable instances of $C_{\tv(\tau)}^u$
%are not part of the constraint set obtained after improvement of $C$,
%if this improved constraint set exists.

%In \ref{sec:satisfiability}, page \pageref{Phi0}, we show how to
%define $\Phi_0$ and the initial program theory $P$ from the class and
%instance declarations that are visible in a program module.

\begin{figure}
   \[ \displaystyle
       \frac{
        \begin{array}{l}
           C' = \{ \pi \mid tv(\pi) \subseteq \unreachableVars(C, \tv(\tau))\} \\
           \entailingSubs(C', P) = \{ \phi \}
        \end{array} }
      {C \Rightarrow \tau \improves {P} (C - C') \Rightarrow \tau}  \]
\caption{Constraint Set Improvement}
\label{fig:constraint-set-improvement}
\end{figure}

