\documentclass{llncs}
\usepackage{amsmath,amssymb}
\usepackage{makeidx}  % allows for indexgeneration
%

\usepackage[american]{babel}
\usepackage[utf8]{inputenc}

\usepackage{float}
\usepackage{amssymb}
\usepackage{color}
\usepackage{fancyhdr}
%\usepackage{minted}

\begin{document}

%% commands used

\newtheorem{Lemma}{Lemma}
\newtheorem{Theorem}{Theorem}
\newtheorem{Definition}{Definition}
\newtheorem{Corollary}{Corollary}
\newtheorem{Example}{Example}

%% notations

\newcommand{\id}{{\it id\/}}
\newcommand{\dom}[1]{\ensuremath{\mathit{dom}(#1)}}
\newcommand{\CO}{\ensuremath{\mathit{C\/}}}
\newcommand{\entailingSubs}{\ensuremath{\mathit{entailingSubs\/}}}
\newcommand{\unify}{{\it unify\/}}
\newcommand{\tv}{{\it tv\/}}
\newcommand{\gtv}{{\it gtv\/}}
\newcommand{\gtc}{{\it gtc\/}}
\newcommand{\rtv}{{\it rtv \/}}
\newcommand{\specialize}{{\it specialize\/}}

%\newcommand{\haskell}[1]{\mintinline{haskell}{#1}}
\newcommand{\haskell}[1]{\tt{#1}}

\newcommand{\SSS}{\mathcal{S}}

\newcommand{\err}{{\bf W}}
\newcommand{\VV}{{\cal V}}
\newcommand{\GT}{{\cal T}_G}
\newcommand{\KK}{{\cal K}}
\newcommand{\TT}{{\cal T}}
% conflict with pstree
\newcommand{\TCC}{\mbox{\it TC}}
\newcommand{\UL}{\mbox{UL}}
\newcommand{\sUL}{\mbox{\scriptsize UL}}
\newcommand{\true}{\mbox{\sf true}}
\newcommand{\SF}{\mbox{$\cal S$}}
\newcommand{\tauvec}{\bar{\tau}}
\newcommand{\tvec}{\bar{t}}
\newcommand{\kvec}{\bar{k}}
\newcommand{\xvec}{\bar{x}}
\newcommand{\muvec}{\bar{\mu}}
\newcommand{\alphavec}{\bar{\alpha}}
\newcommand{\betavec}{\bar{\beta}}
\newcommand{\gammavec}{\bar{\gamma}}
\newcommand{\thetavec}{\bar{\theta}}
\newcommand{\deltavec}{\bar{\delta}}
\newcommand{\bvec}{\bar{b}}
\newcommand{\typo}{\Gamma}
\newcommand{\tenv}{\Gamma}
\newcommand{\typoinit}{\typo_0}
\newcommand{\static}{S}
\newcommand{\dynamic}{D}
\newcommand{\sstatic}{{\scriptstyle S}}
\newcommand{\sdynamic}{{\scriptstyle D}}
\newcommand{\BTA}{\mbox{BTA}}
\newcommand{\baseshape}{\bot}
\newcommand{\TCT}{P}
\newcommand{\Haskell}{H98}
\newcommand{\TCTinit}{P_o}

% figures, rules

\newcommand{\tlabel}[1]{\mbox{(#1)}}
\newcommand{\fig}[3]
        {\begin{figure*}[t]#3\
        \caption{\label{#1}#2}\  \end{figure*}}
\newcommand{\figurebox}[1]
        {\fbox{\begin{minipage}{\textwidth} #1 \end{minipage}}}
\newcommand{\boxfig}[3]
        {\begin{figure*}\figurebox{#3\caption{\label{#1}#2}}\end{figure*}}
\newcommand{\myirule}[2]{{\renewcommand{\arraystretch}{1.2}\ba{c} #1
                      \\ \hline #2 \ea}}

\newcommand{\sirule}[2]{{\renewcommand{\arraystretch}{1.2}\ba{c}
                      \\ \hline #1 \ea}}

% relations, operators

\newcommand{\ti}{\vdash_i}
\newcommand{\tiUm}{\vdash_{i1}}
\newcommand{\sem}{\vdash_a}
\newcommand{\unambig}{\mbox{\it unambig}}
\newcommand{\complete}{\mbox{\it complete}}

\newcommand{\projexclusion}[1]{\mbox{$\bar{\exists}#1$}}
\newcommand{\funinst}{\mbox{\it inst}}
%%\renewcommand{\inst}{\, \vdash^{\scriptstyle i} \,}
%%\newcommand{\inst}{\, \vdash \,}
\newcommand{\ileq}{\preceq}
\newcommand{\ieq}{\simeq}
\newcommand{\topannot}[1]{|#1|}
\newcommand{\gendelta}{\mbox{\it gen}_{\delta}}
\newcommand{\genbeta}{\mbox{\it gen}_{\beta}}
\newcommand{\normalize}{\mbox{\it normalize}}
\newcommand{\fail}{\mbox{\it fail}}
\newcommand{\fixpt}{\mbox{${\cal F}$}}
\newcommand{\testpa}{\mbox{${\cal T}$}}
\newcommand{\addpa}{\mbox{${\cal A}$}}
\newcommand{\matchpa}{\mbox{${\cal M}$}}
\newcommand{\refmatchpa}{\mbox{${\cal R}$}}
\newcommand{\wft}[1]{\mbox{\it wft}(#1)}
\newcommand{\stv}{\mbox{\it semfv}}
\newcommand{\range}{\mbox{\it range}}
\newcommand{\lub}{\wedge}
\newcommand{\tyrel}[2]{(#1 \preceq #2)}
\newcommand{\eqty}[2]{#1=#2}


\newcommand{\mynote}[1]{$\spadesuit${\bf #1}$\clubsuit$}

\newcommand{\textttnf}[1]{\normalfont \texttt{#1}}

\newcommand{\dunderline}[1]{\underline{\underline{#1}}}

% spacing

\newcommand{\sgap}{\quad}
\newcommand{\bgap}{\quad\quad}

% reserved words

\newcommand{\mathem}{\sf}
\newcommand{\IN}{\mbox{\mathem in}}
\newcommand{\MLET}{\mbox{\mathem mlet}}
\newcommand{\LETREC}{\mbox{\mathem letrec}}
\newcommand{\FIX}{\mbox{\mathem fix}}
\newcommand{\REC}{\mbox{\mathem rec}}
\newcommand{\TYCASE}{\mbox{\mathem typeCase}}
\newcommand{\ELSE}{\mbox{\mathem else}}
\newcommand{\IF}{\mbox{\mathem if}}
\newcommand{\OF}{\mbox{\mathem of}}
\newcommand{\TYOF}{\mbox{\underline{\mathem of}}}
\newcommand{\THEN}{\mbox{\mathem then}}
\newcommand{\INST}{\mbox{\mathem instance}}
\newcommand{\OVER}{\mbox{\mathem overload}}
\newcommand{\CLASS}{\mbox{\mathem class}}
\newcommand{\WHERE}{\mbox{\mathem where}}
\newcommand{\OTHERWISE}{\mbox{\mathem otherwise}}
\newcommand{\NOT}{\mbox{\mathem not}}

\newcommand{\search}{\text{\it search\/}}
\newcommand{\unreachableVars}{\text{\it unReachVs\/}}

%%% Local Variables:
%%% mode: latex
%%% TeX-master: "bta"
%%% End:

\newcommand{\figuretwo}[1]
        {\begin{minipage}{8.5cm} #1 \end{minipage}}

\newcommand{\figtwocol}[3]
        {\begin{figure}\hrulefill\ \figuretwo{#3 \caption{\label{#1}#2}}\end{figure}}

\newcommand{\figtwocolht}[3]
        {\begin{figure}[ht!]\hrulefill\ \figuretwo{#3 \caption{\label{#1}#2}}\end{figure}}

\newcommand{\pturns}{\vdash_{POSIX}}
\newcommand{\gturns}{\vdash_{greedy}}
\newcommand{\gturnst}{\vdash_{greedy2}}
\newcommand{\glmlturns}{\vdash_{glr}}
\newcommand{\glmlturnst}{\vdash_{glr2}}
\newcommand{\lmlturns}{\vdash_{lm}}
\newcommand{\lmlturnsTop}{\vdash_{lm_{top}}}

\newcommand{\match}[3]{#1 \turns #2 \leadsto #3}
\newcommand{\posixmatch}[3]{#1 \pturns #2 \leadsto #3}
\newcommand{\gmatch}[3]{#1 \gturns #2 \leadsto #3}
\newcommand{\gmatcht}[3]{#1 \gturnst #2 \leadsto #3}
\newcommand{\glmlmatch}[3]{#1 \glmlturns #2 \leadsto #3}
\newcommand{\glmlmatcht}[3]{#1 \glmlturnst #2 \leadsto #3}
\newcommand{\lmlmatch}[3]{#1 \lmlturns #2 \leadsto #3}
\newcommand{\lmlmatchTop}[3]{#1 \lmlturnsTop #2 \leadsto #3}

\newcommand{\comb}{\diamond} %%{\mbox{\it comb}}

\newcommand{\baseFv}{\mathit{baseFv}}
\newcommand{\fv}{\mathit{fv}}

\newcommand{\emptySeq}{\epsilon}

\newcommand{\Nturns}{\, \vdash_{\mbox{\scriptsize lnf}} \,}
\newcommand{\Lturns}{\, \vdash^{L}_{\mbox{\scriptsize lnf}} \,}
\newcommand{\Rturns}{\, \vdash^{R}_{\mbox{\scriptsize lnf}} \,}
\newcommand{\norm}{{\leadsto}}
\newcommand{\lnorm}{{\lhd}}
\newcommand{\rnorm}{{\rhd}}
\newcommand{\pt}[1]{{\cal P}(#1)}
\newcommand{\ty}[1]{{\cal T}(#1)}
\newcommand{\te}[1]{{\cal G}(#1)}
\newcommand{\ve}[1]{{\cal D}(#1)}
\newcommand{\exlang}[1]{{\cal O}(#1)}
\newcommand{\exsub}[1]{{\cal S}(#1)}
\newcommand{\turnsF}{\, \vdash^{\mbox{\scriptsize F}} \,}
%\newcommand{\implies}{\supset}
\newcommand{\clabel}[1]{\mbox{(#1)}}
\newcommand{\rat}[1]{\rightarrowtail_{#1}}
\newcommand{\arr}{\rightarrow}
\newcommand{\arrow}{\rightarrow}
\newcommand{\Arr}{\Rightarrow}
\newcommand{\atsign}{@}
\newcommand{\simparrow}[0]{\Longleftrightarrow}
\newcommand{\proparrow}[0]{\Longrightarrow}
\newcommand{\comment}[1]{}
\newcommand{\ignore}[1]{}
\newcommand{\martin}[1]{{\bf Martin:#1}}
\newcommand{\rodrigo}[1]{{\bf Rodgrio:#1}}
\newcommand{\carlos}[1]{{\bf Carlos:#1}}
\newcommand{\apair}[2]{\langle{#1;#2}\rangle}


\newcommand{\pow}{\^{}}
\newcommand{\venv}{\Delta}
\newcommand{\mleq}{\mbox{\tt leq}}
\newcommand{\mmleq}{\leq}
\newcommand{\mas}{\mbox{\tt as}}
\newcommand{\mfix}{\mbox{\tt regfix}}

\newcommand{\pp}{\texttt{++}}

\newcommand{\mysection}[1]{\vspace*{-2mm}\section{#1}\vspace*{-1mm}}
\newcommand{\mysubsection}[1]{\vspace*{-1mm}\subsection{#1}\vspace*{-1mm}}
\newcommand{\mymid}{~\|~}
\newcommand{\lang}[1]{{\cal L}{(#1)}}
\newcommand{\bindsto}{\Rightarrow}
\newcommand{\bindstounder}[1]{\bindsto^{#1}}
\newcommand{\iproj}[1]{\pi^{-1}({#1})}
\newcommand{\rev}[1]{{#1}^{\mbox{\scriptsize r}}}
\newcommand{\lquo}{\backslash}
\newcommand{\rquo}{/}
\newcommand{\diff}{-}
\newcommand{\isect}{\cap}
\newcommand{\stripv}[1]{({#1})\downarrow_{v}}
\newcommand{\stript}[1]{({#1})\downarrow_{t}}
%\renewcommand{\implies}{\Longrightarrow}
\newcommand{\polyxduce}{XDuce$^{\forall.C}$}
\newcommand{\deriv}[2]{#1\backslash #2}
\newcommand{\pderiv}[2]{#1{\backslash}_p #2} %% {#1\backslash_p #2}
\newcommand{\mkEmp}[1]{{#1}_{\epsilon}}
\newcommand{\strip}[1]{#1\downarrow}
%% match environment
\newcommand{\menv}[1]{{\it env}(#1)}
\newcommand{\allmatch}[2]{{\it match}(#1,#2)}
\newcommand{\transition}{\xrightarrow{\hspace*{1.5cm}}}
\newcounter{cnt}

\newcommand{\iterate}{\mathit{iterate}}

\newcommand{\pset}[1]{{\cal P}(#1)}

%%%%%%%%%%%%%%%%%%%%%%%%%%%%%%%%%%%%%%%%%%%%%%%%%%%%%%%%%%%%%%%%

\newcommand{\mempty}{{\it{empty\/}}}
\newcommand{\minsert}{{\it{insert\/}}}
\newcommand{\Coll}{{\it{Coll\/}}}

\newcommand{\ento}{\ensuremath{\texttt{ent}_0}}
\newcommand{\entinst}{\ensuremath{\texttt{inst}_0}}
\newcommand{\entmp}{\ensuremath{\texttt{mp}_0}}
\newcommand{\entn}{\ensuremath{\texttt{conj}_0}}

\newcommand{\redo}{\ensuremath{\texttt{red}}}
\newcommand{\instum}{\ensuremath{\texttt{inst}}}
\newcommand{\conj}{\ensuremath{\texttt{conj}}}
\newcommand{\stopo}{\ensuremath{\texttt{stop}}}
\newcommand{\stopFail}{\ensuremath{\texttt{stop}_0}}
\newcommand{\stopNoMatch}{\ensuremath{\texttt{stop}_1}}

\newcommand{\I}{\text{\it I\/}}
\newcommand{\F}{\text{\it F\/}}
\newcommand{\B}{\text{\it B\/}}
\newcommand{\C}{\text{\it C\/}}

\newcommand{\Ifd}{\text{\it I\/}_{\text{Fd}}}
\newcommand{\Fd}{\text{\it Fd\/}}
\newcommand{\instf}{\text{\it inst\/}}
\newcommand{\vS}{\text{\it vSeq\/}}
\newcommand{\varS}{\text{\it vseq\/}}
\newcommand{\compose}{\text{\it comp\/}}
\newcommand{\dc}{\text{\it dc\/}}
\newcommand{\matches}{\text{\it matches\/}}
\newcommand{\mgu}{\rm{\it mgu\/}}
\newcommand{\mgm}{\rm {\it mgm\/}}
\newcommand{\mguI}{\ensuremath{\mbox{\it mgu}}_I}
\newcommand{\matchI}{\text{\it match}_I}
\newcommand{\satI}{\text{\it sat}_I}
\newcommand{\sattt}{\text{\it sat}_1}
\newcommand{\lcgI}{\text{\it lcg}_I}
\newcommand{\lcgp}{\text{\it lcgp\/}}
\newcommand{\lcgg}{\text{\it lcg'\/}}
\newcommand{\entails}{\vdash}
\newcommand{\simp}{\ensuremath{>\!\!\!>}}
\newcommand{\entailI}{\text{\it entails\/}}
\newcommand{\simplifiesI}{{\it simpl\/}}
\newcommand{\wfI}{\text{\it wf}_I}
\newcommand{\ev}{\text{\it ev\/}}
\newcommand{\card}{\text{\it card\/}}

\newcommand{\SatFailUm} {{\text{\tt fail}}_1}
\newcommand{\SatEmptyUm}{{\text{\tt empty}}_1}
\newcommand{\SatConjUm} {{\text{\tt conj}}_1}
\newcommand{\SatInstUm} {{\text{\tt inst}}_1}

\newcommand{\Tau}{\ensuremath {\cal T}}

\newcommand{\git}{\text{\it git\/}}
\newcommand{\ma}{\text{\it m1\/}}
\newcommand{\mb}{\text{\it m2\/}}
\newcommand{\mc}{\text{\it m3\/}}
\newcommand{\NumLit}{\text{\it NumLit\/}}
\newcommand{\Sum}{\text{\it Sum\/}}
\newcommand{\Div}{\text{\it Div\/}}
\newcommand{\Mult}{\text{\it Mult\/}}
\newcommand{\Vector}{\text{\it Vector\/}}
\newcommand{\Matrix}{\text{\it Matrix\/}}
\newcommand{\Ord}{\text{\it Ord\/}}
\newcommand{\TEqInt}{\text{\it TEqInt\/}}
\newcommand{\EqChar}{\text{\it EqChar\/}}
\newcommand{\TEqL}{\text{\it TEqL\/}}
\newcommand{\TEq}{\text{\it TEq\/}}
\newcommand{\teq}{\text{\it teq\/}}
\newcommand{\eq}{\text{\it eq\/}}
\newcommand{\EqL}{\text{\it EqL\/}}
\newcommand{\Num}{\text{\it Num\/}}
\newcommand{\Fail}{\text{\it Fail\/}}
\newcommand{\Zip}{\text{\it Zip\/}}
\newcommand{\zip}{\text{\it zip\/}}
\newcommand{\zipd}{\text{\it zip2\/}}
\newcommand{\Main}{\text{\it Main\/}}
\newcommand{\main}{\text{\it main\/}}
\newcommand{\bs}{\text{\it bs\/}}
\newcommand{\nott}{\text{\it not\/}}
\newcommand{\primEqInt}{\text{\it primEqInt\/}}
\newcommand{\otherw}{\text{\it otherwise\/}}

\newcommand{\sshowL}{\text{\it show'\/}}
\newcommand{\rreadL}{\text{\it read'\/}}

\newcommand{\classMember}{\text{\it{classMember\/}}}
\newcommand{\className}{\text{\it{className\/}}}
\newcommand{\type}{\text{\it{type\/}}}

\newcommand{\toList}{\text{\it toL\/}}
\newcommand{\foldWithKey}{\text{\it foldWithKey\/}}
\newcommand{\fold}{\text{\it fold\/}}
\newcommand{\inter}{\text{\it inter\/}}

\newcommand{\res}{\arrowvert}
\newcommand{\lb}{\ensuremath {\text{\it lb\/}}}

\newcommand{\Unsat}{\ensuremath{\text{\it Unsat\/}}}
\newcommand{\Single}{\ensuremath{\text{\it Single\/}}}
\newcommand{\Ambiguous}{\ensuremath{\text{\it Ambiguous\/}}}

\newcommand{\mif}{{\bf if}\ }
\newcommand{\mthen}{\ {\bf then}\ }
\newcommand{\melse}{\ {\bf else}\ }

\newcommand{\mlet}{{\tt let}\ }
\newcommand{\iin}{\ {\tt in}\ }
\newcommand{\mwhere}{{\tt where}\ }
\newcommand{\mcase}{{\bf case}\ }
\newcommand{\mof}{{\bf of}\ }

\newcommand{\x }{[\![}
\newcommand{\y }{]\!]}
\newcommand{\yy }{]\!]}

\newcommand{\Ddelta}{{\scriptsize \Delta}}



% \definecolor{mygrey}{gray}{0.85}

\newcommand{\prog}[1]{
  \normalfont\ttfamily
  \begin{center}\begin{tabular}{l}
       #1
  \end{tabular} \end{center}
  \normalfont}

\newcommand{\proga}[1]{
 \normalfont\ttfamily
  \begin{tabbing}
        #1
   \end{tabbing}
 \normalfont}

\newcommand{\progfig}[1]{\parbox{11cm}{
  \normalfont
  \begin{center}
  \parbox{\textwidth}{\begin{tabbing}
        #1
  \end{tabbing}}
  \end{center}
  \normalfont}}

\newcommand{\progb}[1]{
  \normalfont\ttfamily
  \begin{tabbing}
        #1
  \end{tabbing}
  \normalfont}

\newcommand{\progn}[1]{
  \begin{center}
  \parbox{\textwidth}{\begin{tabbing}
        #1
  \end{tabbing}}
  \end{center} }

\newcommand{\progbb}[1]{
  \normalfont\ttfamily
  \begin{center}
  \shadowbox{\parbox{\textwidth}{\begin{tabbing}
        #1
  \end{tabbing}}}
  \end{center}
  \normalfont}

\newcommand{\progc}[1]{
  \normalfont\ttfamily
  \begin{center}
  \fbox{\parbox{\textwidth}{\begin{tabbing}
        #1
  \end{tabbing}}}
  \end{center}
  \normalfont}

\newcommand{\progcc}[1]{
  \normalfont\ttfamily
  \begin{center}
  \fbox{\begin{tabular}{p{4cm}}
        #1
  \end{tabular}}
  \end{center}
  \normalfont}

\newcommand{\progaa}[1]{
  \normalfont\ttfamily
  \begin{center}\begin{tabular}{ll}
       #1
  \end{tabular} \end{center}
  \normalfont}

\newcommand{\progaaa}[1]{
  \normalfont\ttfamily
  \begin{center}\begin{tabular}{lll}
       #1
  \end{tabular} \end{center}
  \normalfont}

\newcommand{\CT}{{\it CT\/}}
\newcommand{\CSSAT}{{\it CS-SAT\/}}
\newcommand{\ssat}{{\tt sat}}

\newcommand{\lcg}{{\it lcg\/}}
\newcommand{\lcgR}{{\tt lcg}_r}

\newcommand{\Integer}{{\it Integer\/}}
\newcommand{\Bool}{{\it Bool\/}}
\newcommand{\Int}{{\it Int\/}}
\newcommand{\Char}{{\it Char\/}}
\newcommand{\Float}{{\it Float\/}}
\newcommand{\Double}{{\it Double\/}}

\newcommand{\True}{{\it True\/}}
\newcommand{\False}{{\it False\/}}

\newcommand{\Lit}{{\it Lit\/}}
\newcommand{\Inc}{{\it Inc\/}}
\newcommand{\IsZ}{{\it IsZ\/}}
\newcommand{\If}{{\it If\/}}
\newcommand{\Pair}{{\it Pair\/}}
\newcommand{\Term}{{\it Term\/}}

\newcommand{\eval}{{\it eval\/}}

\newcommand{\MMo}{{\tt MMo}}

\newcommand{\genn}[1]{\bar{\bar{#1}}}
\newcommand{\abrgenn}[1]{\|#1\|}

\newcommand{\T}{{\it T\/}}
\newcommand{\Tu}{{\it T1\/}}
\newcommand{\Td}{{\it T2\/}}
\newcommand{\test}{{\it test\/}}
\newcommand{\sat}{{\it sat\/}}

\newcommand{\Eq}{{\it Eq\/}}

\newcommand{\CASE}{{\tt CASE}}
\newcommand{\IMPROVE}{{\tt IMPROVE}}
\newcommand{\ALTSRec}{{\tt ALTSRec}}
\newcommand{\ALTSEnd}{{\tt ALTSEnd}}
\newcommand{\ALT}{{\tt ALT}}
\newcommand{\alt}{{\tt alt}}
\newcommand{\alts}{{\tt alts}}

\newcommand{\colchetes}{\mbox{\tt []}}
\newcommand{\module}{\mbox{\tt module}}
\newcommand{\export}{\mbox{\tt export}}
\newcommand{\import}{\mbox{\tt import}}
\newcommand{\data}{{\tt data}}
\newcommand{\classDecl}{{\it classDecl\/}}
\newcommand{\instDecl}{{\it instDecl\/}}
\newcommand{\importC}{I}
\newcommand{\exportC}{X}
\newcommand{\class}{\mbox{\tt class}}
\newcommand{\instance}{{\tt instance}}
\newcommand{\where}{\mbox{\tt where}}
\newcommand{\member}{{\it member\/}}
\newcommand{\iitem}{\iota}
\newcommand{\iinsert}{{\it insert\/}}
\newcommand{\eempty}{{\it empty\/}}
\newcommand{\Either}{{\it Either\/}}
\newcommand{\Monad}{{\it Monad\/}}
\newcommand{\mtl}{{\it mtl\/}}
\newcommand{\transformers}{{\it transformers\/}}
\newcommand{\Data}{{\it Data\/}}
\newcommand{\List}{{\it List\/}}
\newcommand{\Tree}{{\it Tree\/}}
\newcommand{\sortBy}{{\it sortBy\/}}

\newcommand{\Mum}{{\it M1\/}}
\newcommand{\SShow}{{\it Show\/}}
\newcommand{\ShowInt}{{\it ShowInt\/}}
\newcommand{\RRead}{{\it Read\/}}
\newcommand{\sshow}{{\it show\/}}
\newcommand{\rread}{{\it read\/}}
\newcommand{\myshow}{{\it myshow\/}}
\newcommand{\myread}{{\it myread\/}}
\newcommand{\String}{{\it String\/}}

\newcommand{\Collection}{{\it Collection\/}}
\newcommand{\flip}{{\it flip\/}}
\newcommand{\map}{{\it map\/}}
\newcommand{\const}{{\it const\/}}
\newcommand{\ddom}{\mathit{dom\/}}

\newcommand{\st}{\mathit{st\/}}
\newcommand{\freshst}{\mathit{freshst\/}}

\newcommand{\MODULE}{\mbox{\tt{\scriptsize{MOD}}}}
\newcommand{\VAR}{\mbox{\tt{\scriptsize{VAR}}}}
\newcommand{\APP}{\mbox{\tt{\scriptsize{APP}}}}
\newcommand{\ABS}{\mbox{\tt{\scriptsize{ABS}}}}
\newcommand{\LET}{\mbox{\tt{\scriptsize{LET}}}}
\newcommand{\VVARi}{\ensuremath{\mbox{\tt{\scriptsize{VAR}}}_i}}
\newcommand{\APPi}{\ensuremath{\mbox{\tt{\scriptsize{APP}}}_i}}
\newcommand{\ABSi}{\ensuremath{\mbox{\tt{\scriptsize{ABS}}}_i}}
\newcommand{\LETi}{\ensuremath{\mbox{\tt{\scriptsize{LET}}}_i}}
\newcommand{\VARs}{\mbox{\tt{\scriptsize{VAR}}}_s}
\newcommand{\APPs}{\mbox{\tt{\scriptsize{APP}}}_s}
\newcommand{\ABSs}{\mbox{\tt{\scriptsize{ABS}}}_s}
\newcommand{\LETs}{\mbox{\tt{\scriptsize{LET}}}_s}

\newcommand{\simplifies}[1]{>\!\!\!>_{#1}}

\newcommand{\fst}{{\it fst\/}}
\newcommand{\snd}{{\it snd\/}}
\newcommand{\reachable}{\arrowvert^{\ast}}
\newcommand{\gen}{\mbox{\it gen}}
\newcommand{\improves}[1]{\ensuremath{\vdash_{\texttt{impr}}^{#1}}}
\newcommand{\contextreduces}[1]{\ensuremath{\vdash_{\texttt{red}}^{#1}}}
\newcommand{\reachableVars}{{\it reachableVs\/}}
\newcommand{\entail}{\vdash_e}
\newcommand{\self}{\mathtt{self}}

\newcommand{\satsUm}{\textit{sats\/}_1}
\newcommand{\satUm}[1]{\ensuremath{\vdash_{\texttt{sat1}}^{#1}}}
\newcommand{\sats}[1]{\ensuremath{\vdash_{\texttt{sats}}^{#1}}}
\newcommand{\leqI}{\textit{leq\/}_I}
\newcommand{\vdashI}{\vdash_I}

\newcommand{\as}{{\it as\/}}
\newcommand{\Address}{{\it Address\/}}
\newcommand{\aaddress}{{\it address\/}}
\newcommand{\Person}{{\it Person\/}}
\newcommand{\new}{{\it new\/}}
\newcommand{\get}{{\it get\/}}
\newcommand{\update}{{\it update\/}}
\newcommand{\name}{{\it name\/}}
\newcommand{\fieldname}{{\it fieldname\/}}

\newcommand{\Button}{{\it Button\/}}
\newcommand{\Canvas}{{\it Canvas\/}}
\newcommand{\reset}{{\it reset\/}}
\newcommand{\RReset}{{\it Reset\/}}
\newcommand{\ddo}{{\tt do}}

\newcommand{\IO}{{\it IO\/}}
\newcommand{\IORef}{{\it IORef\/}}
\newcommand{\Ref}{{\it Ref\/}}
\newcommand{\ST}{{\it ST\/}}
\newcommand{\STRef}{{\it STRef\/}}
\newcommand{\newRef}{{\it newRef\/}}
\newcommand{\readRef}{{\it readRef\/}}
\newcommand{\writeRef}{{\it writeRef\/}}
\newcommand{\readAndPrint}{{\it readAndPrint\/}}
\newcommand{\print}{{\it print\/}}



\input{meta.keys}

%
\mainmatter              % start of the contributions

\title{Optional Type Classes for Haskell}

%
\author{Rodrigo Ribeiro \inst{1} \and Carlos Camar\~ao\inst{2} \and Luc\'ilia
Figueiredo\inst{3} \and Cristiano Vasconcellos\inst{4}}
\institute{DECSI, Universidade Federal de Ouro Preto (UFOP), João Monlevade\\
\email{rodrigo@decsi.ufop.br} \and
DCC, Universidade Federal de Minas Gerais (UFMG), Belo Horizonte\\
\email{camarao@dcc.ufmg.br} \and
DECOM, Universidade Federal de Ouro Preto (UFOP), Ouro Preto\\
\email{luciliacf@gmail.com} \and
DCC, Universidade do Estado de Santa Catarina (UDESC), Joinville\\
\email{cristiano.vasconcellos@udesc.br}
}
\maketitle              % typeset the title of the contribution

\begin{abstract}

This paper explores an approach for allowing type classes to be
optionally declared by programmers, i.e. programmers can overload
symbols without declaring their types in type classes.

The type of an overloaded symbol is, if not explicitly defined in a
type class, automatically determined from the anti-unification of
instance types defined for the symbol in the relevant module. A type
class having the overloaded name as its unique member is automatically
created from this type.

This depends on a modularization of instance visibility, as well as on
a redefinition of Haskell's ambiguity rule. The paper presents the
modifications to Haskell's module system that are necessary for
allowing instances to have a modular scope, based on previous work by
the authors. The definition of the type of overloaded symbols as the
anti-unification of available instance types and the redefined
ambiguity rule is also based on previous works by the authors.

The added flexibility to Haskell-style of overloading is illustrated
by defining a type system and a type inference algorithm that allows
overloaded record fields.

% The redefinition of Haskell's ambiguity rule can also be used to
% address some of the issues related to type directed name resolution.

\end{abstract}

\section{Introduction}
\label{sec:intro}

We use the term {\em constrained polymorphism\/} to refer to the
polymorphism originated by the combination of parametric polymorphism
and context-dependent overloading.  Context-dependent overloading is
characterized by the fact that overloading resolution in expressions
(function calls) $e\: e'$ is based not only on the types of the
function ($e$) and the argument ($e'$), but also on the context in
which the expression ($e\: e'$) occurs. This makes overloading much
more expressive, since:

\begin{itemize}

  \item constants can also be overloaded --- for example, literals
    (like {\tt 1}, {\tt 2} etc.) can be used to represent fixed and
    arbitrary precision integers as well as fractional numbers, so
    that they can be used in expressions such as {\tt \mbox{1 + 2.0}}
    ---, and

  \item overloading resolution for function calls need not occur when
    the argument is encountered.  Functions with types that differ
    only on the type of the result can be overloaded, for example {\em
      read\/} functions with types {\tt \String\ $\rightarrow$ \Bool},
    {\tt \String\ $\rightarrow$ \Int}, each taking a string and
    generating the denoted value in the corresponding type ---, and
    also higher-order curried functions, e.g.~mapping functions with
    types {\tt $(a\rightarrow b)\rightarrow f\:a\rightarrow f\:b$},
    folding functions with types {\tt $(a\rightarrow b\rightarrow
      b)\rightarrow b\rightarrow f\:a\rightarrow b$}, for distinct
    instances of $f$, can be overloaded.

\end{itemize}

In this way, context-dependency allows overloading to have a much more
prominent role in the presence of parametric polymorphism, as explored
mainly in the programming language Haskell.

The main motivation of this paper is to allow symbols to be overloaded
in a language that supports constrained polymorphism without requiring
the types of these symbols to be declared separately in type classes.
We consider also the following:

\begin{description}

\item[Expression Ambiguity:] Ambiguity is defined according to an
  intuitive notion of the existence of two or more distinct instances
  of an overloaded name used when overloading is resolved. It is not
  as a syntactic property of a type, as it is defined in Haskell.
  Ambiguity is discussed in Section \ref{sec:ambiguity}. This
  modification is not necessary for optional type classes to work, but
  allows greater flexibility; together with optional type classes it
  enables, for example, type directed name resolution to be obtained
  for free (Subsection \ref{sec:type-directed-name-resolution}) and
  allows also an easy support of overloaded record fields (Subsection
  \ref{sec:overloaded-record-fields}).


\item[Modular Instance Scope:] Overloaded names with modular scope, as
  occurs with non-overloaded names. The paper presents minimalist
  modifications to Haskell's module system that are necessary for
  overloaded names to have a modular scope, whether their types are
  annotated or not in type classes. This modification is also not
  necessary for optional type classes to work. Instances with a type
  class can maintain their global scope; instances without a type
  class could as well have a global scope, though that design decision
  would be rather peculiar. In our view the benefits of instance
  modularization outweigh its possible drawbacks. This is discussed in
  Section \ref{sec:modular-instances}.

\end{description}

The mechanism of optional type classes presented in this paper is
based on the following:

\begin{enumerate}

 \item The type of an overloaded symbol is, as usual, a constrained
   type, of the form $\forall\,\overline{a}.\,C \Rightarrow \tau$,
   where $C$ is a set of constraints and $\tau$ is a simple
   (unquantified and unconstrained) type. A constraint is a class name
   followed by a sequence of simple types.

\item An overloaded symbol $x$ can be defined by instance declarations
  of the form $\instance\ x=e$, without explicitly declaring its type
  in a type class.
  
\item The type of $x$ is automatically determined from the
  anti-unification of the instance types for $x$ that are visible in a
  module, by creating a type class with a single member ($x$). The
  algorithm used for computing the type of $x$ is presented in
  Section~\ref{sec:anti-unif}.

\end{enumerate}

The proposed approach is formalized in
Section~\ref{Optional-type-classes}, where a type system for a
core-Haskell language where type classes can be optionally declared is
presented.

The added flexibility of the overloading mechanism, with respect to
Haskell, is illustrated by the enabled simple support for overloaded
record fields (Section \ref{sec:overloaded-record-fields}) and type
directed name resolution (Section
\ref{sec:type-directed-name-resolution}).

A type inference algorithm is presented in Section
\ref{sec:type-inference}, which is proved to be sound and to infer,
for a given expression in a given typing context, a type that is a
minimal generalization of types derivable in the type system.

% A semantics is presented in Section \ref{sec:semantics}. 

Related work is outlined in Section \ref{sec:related-work} and Section
\ref{sec:conclusion} concludes.

Appendices include the definition of the entailment relation
(\ref{sec:entailment}), used in the type system, the improvement and
context-reduction relations
(\ref{sec:improvement},\ref{sec:context-reduction}), whose composition
yield the constraint-set simplification relation, also used in the
type system, and the constraint-set satisfiability function
(\ref{sec:satisfiability}), used in the improvement relation and in
the type inference algorithm.

A prototype implementation of a type inference algorithm for Haskell
supporting overloading without the need of defining a type class is
available \cite{opt-rep}.


\section{Anti-unification of instance types}
\label{sec:anti-unif}

A simple type $\tau$ is a generalization of a set of simple types
$\overline{\tau}^{\,n}$ if there exist substitutions
$\overline{\phi}^{\,n}$ such that $\phi_i(\tau)=\tau_i$, for
$i=1,\ldots,n$. For example, $a_0 \rightarrow a_0$,
$a_1 \rightarrow a_2$, and $a_3$ are generalizations of
$\{ \Int \rightarrow \Int, \Float \rightarrow \Float\}$.\footnote{A generalization is also called a (first-order) {\em
  anti-unification\/} \cite{ModelTheory2012}.}

We say that $\tau$ is less general than $\tau'$, written $\tau \leq
\tau'$, if there exist a substitution $\phi$ such that $\phi(\tau') = \tau$.  For
example, $a_0 \rightarrow a_0 \leq a_1 \rightarrow
a_2 \leq a_3$.

The {\it least common generalization} (lcg) of a set of types
$S$ and a type $\tau$ holds, written as $\lcgR(S,\tau)$, if, for all generalizations $\tau'$ of
$S$ we have $\tau \leq \tau'$.

The concept of least common generalization was studied by Gordon Plotkin \cite{plotkin1970note,plotkin1971further}, that defined a
function for constructing a generalization of two symbolic
expressions.  In Figure~\ref{fig:lcg}, we define function \lcg, which 
returns a lcg of a finite set of 
simple types $S$, by recursion on the structure of $S$,  
using function $\lcg'$ to compute the
generalization of two simple types. For two types $\tau_1$ and
$\tau_2$ the idea is to recursively traverse the structure of both
types using a finite map to store previously generalized
types. Whenever we find two different type constructors, we search on
the finite map if they have been previously generalized. If this is
the case, the previous generalization is returned. If these two type
constructors are not in the finite map, we insert them using a fresh
type variable as their generalization and return this new variable.

\begin{figure*}[ht]
	\[\progfig{
            $\lcg(S)=\tau$ $\:\:\:$ where 
               $(\tau, \phi)=\lcg'(S,\id)$, for some  $\phi$ \\ \\
            $\lcg'(\{\tau\},\phi) = (\tau, \phi)$  \\ \\		
            $\lcg'(\{\tau_1\} \cup S, \phi) = \lcg''(\tau_1, \tau',\phi') \:\:\:$ where
		$\begin{array}[t]{ll}
                   (\tau',\phi')  & = lcg'(S, \phi)
		\end{array}$  \\ \\		
            xxx\=xxx\=xxx\=xxx\=xxxxx\=xxxxxx\=xxxxxxxx\= \kill
            $\lcg'' (T \: \overline{\tau}^{\,n},\:  T'\: \overline{\rho}^{\,m},\phi)=$\+\\
              \textbf{if}\ $\phi(a)=( T\:\overline{\tau}^{\,n},\: T'\:\overline{\rho}^{\,m})$
                      for some $a$ \textbf{then}\ $(a,\phi)$ \\
              \textbf{else} \+\\
              \textbf{if}\ $n\not=m$ \textbf{then}\
                 $(b, \phi [b \mapsto ( T \:\overline{\tau}^{\,n},\: T'\:\overline{\rho}^{\,m})])$ \+ \\
		 where $b$ is a fresh type variable \-\\[.1cm]
              \textbf{else}\ $(\psi\: \overline{\tau'}^{\,n}, \phi_n)$\+\\
                 where $\begin{array}[t]{l}
		          (\psi,\phi_0) = \left\{\begin{array}{ll}
                                            (T ,\phi) & \textbf{if } T = T' \\
                                            (a, \phi\,[a\mapsto (T, T')])
                                                      & \text{otherwise, $a$ is fresh }\\
                                          \end{array}\right. \\[.3cm]
                          (\tau'_i,\phi_i) = lcg''(\tau_i, \rho_i, \phi_{i-1}), \text{ for } i=1, \ldots, n
                        \end{array}$ \-\-\-	
        }
        \] \vspace{-.2cm}
\caption{Least Common Generalization} \label{fig:lcg}
\end{figure*}
As an example of the use of \lcg, consider the following types (of
functions \map\ on lists and trees, respectively):

\progb{
   $(a \rightarrow b)$ $\rightarrow$ [$a$] $\rightarrow$ [$b$]\\
   $(a \rightarrow b)$ $\rightarrow$ \Tree\ $a$ $\rightarrow$ \Tree\ $b$
}

A call of \lcg\ for a set with these types yields type $(a \rightarrow
b) \rightarrow c\:\: a \rightarrow c\:\: b$, where $c$ is a
generalization of type constructors {\tt []} and \Tree\ (for $c$ to be
used in $c\: b$, mapping $c \mapsto (\texttt{[]},\Tree)$ is saved in
parameter $\phi$ of $\lcg''$, to be reused).

The following theorems guarantee correctness of function \lcg: 

\begin{Theorem}[Soundness of \lcg]
For all (sets of simple types) $S$, we have that
$\lcg(S)$ yields a generalization of $S$.
\label{theorem:lcg-is-sound}
\end{Theorem}

\begin{Theorem}[Completeness of \lcg]
For all (sets of simple types) $S$, we have that $\lcgR(S,\lcg(S))$
holds (i.e.~$\lcg(S)$ is a generalization of $S$) and, for any $\tau$
that is a generalization of $S$, we have that $\lcg(S) \leq \tau$.
\label{theorem:lcg-is-complete}
\end{Theorem}

\begin{Theorem}[Compositionality of \lcg]
For all non-empty (sets of simple types) $S, S'$, we have that
$\lcg(\lcg(S),\lcg(S')) = \lcg(S \cup S')$.
\label{theorem:lcg-is-compositional}
\end{Theorem}

\begin{Theorem}[Uniqueness of \lcg]
For all (sets of simple types) $S$, we have that
$\lcg(S)$ is unique, up to variable renaming.
\label{theorem:lcg-is-unique-modulo-variable-renaming}
\end{Theorem}


The proofs use straighforward induction on the number and structural
complexity of elements of $S$.




\section{Modularization of Instances}
\label{sec:modular-instances}

This paper does not attempt to discuss any major revision to Haskell's
module system. We summarize in subsection
\ref{subsec:instance-visibility-control} the work, presented in
\cite{Controlling-scope-instances}, that allows a modular control of
the visibility of instance definitions. This has the additional
benefit of enabling type classes to be optionally declared by
programmers, by the introduction of a single additional rule (to
account for the possibility of type classes to be declared or not):

\begin{definition}[Type of overloaded variable]

If the type of an overloaded variable --- i.e.~a variable that is
introduced in an instance definition --- is not explicitly annotated
in a type class declaration, then the variable's type is the
anti-unification of instance types defined for the variable in the
current module; otherwise, it is the annotated type.

\label{overloaded-variable-type}
\end{definition}

Instance modularization and the rule of expression ambiguity, that
considers the context where an expression occurs to detect whether an
expression is ambiguous or not, has profound consequences. Consider,
for example:

\proga{xx\=\kill
\module\ $A$ where\+\\
  \class\ \SShow\ $t$ \ldots\\
  \class\ \RRead\ $t$ \ldots\\
  \instance\ \SShow\ \Int\ \ldots\\
  \instance\ \RRead\ \Int\ \ldots\\
  $f$ = \sshow $\:$.$\:$\rread\-\\ \\

\module\ $B$ \where\+\\
  \import\ $A$\\
  \instance\ \RRead\ \Bool\ \ldots\\
  \instance\ \SShow\ \Bool\ \ldots\\
  $g$ = $f$ "123"
}

In our approach (i.e.~considering ambiguity as a property of an
expression, not of a type), the definition of $f$ in module $A$ is
well-typed (it is not well-typed in Haskell), because constraints
(\SShow\ $a$, \RRead\ $a$) can be removed (in Haskell, type {\tt
  (\SShow\ $a$, \RRead\ $a$) $\Rightarrow$ \String} is ambiguous); the
constraints can be removed because there exists a single instance, in
module $A$, for each constraint, that entails it. As a result, $f$ has
type \String $\rightarrow$ \String. Its use in module $B$ is (then)
also well-typed. That means: $f$'s semantics is a function that
receives a value of type \String\ and returns a value of type \String,
according to the definition of $f$ given in module $A$. The semantics
of an expression involves passing a (dictionary) value that is given
in the context of usage if, {\em and only if}, the expression has a
constrained type.

\subsection{Instance visibility control: a summary}
\label{subsec:instance-visibility-control}

Modularization of instance definitions can be allowed as shown in
\cite{Controlling-scope-instances}. Essentially, import and export
clauses can specify, instead of just names, also {\tt instance $A$
  $\overline{\tau}$}, where $\overline{\tau}$ is a (non-empty)
sequence of types and $A$ is a class name; we have that:

  \[ \text{\module\ $M$ (\instance\ $A$ $\overline{\tau}$, \ldots) \where\ \ldots} \]
specifies that the instance of $\overline{\tau}$ for class $D$ is
exported in module $M$.

  \[ \text{\import\ $M$ (\instance\ $A$ $\overline{\tau}$, \ldots)} \]
specifies that the instance of $\overline{\tau}$ for class $A$ is
imported from $M$, in the module where the import clause occurs.

Alternatively, we can simply give a name to an instance, in an
instance declaration, and use that name in import and export clauses
(see \cite{Controlling-scope-instances}). However, in this paper we
don't need to give a name to an instance, since we only consider
instances of undeclared classes, which have a single member, and we
can thus use the name of the member as the instance name. 

%For example, 
%we can have:
%  \progb{
%   \instance\ $x$ = '1';\\
%   \instance\ $x$ = \True;
%  }    

\subsection{Pros and Cons of Instance Modularization}

Among the advantages of this simple change, we cite (following
\cite{Controlling-scope-instances}):

\begin{itemize}

  \item Programmers have better control of which entities are
    necessary and should be in the scope of each module in a program.

  \item It is possible to define and use more than one instance for
    the same type in a program.

  \item Problems with orphan instances do not occur (orphan instances
    are instances defined in a module where neither the definition of
    the data type nor the definition of the type class occur). For
    example, distinct instances of \Either\ for class \Monad, say one
    from package \mtl\ and another from \transformers, can be used in
    a program.

  \item The introduction of newtypes, as well as the use of functions
    that include additional (-by) parameters, such as e.g.~the (first)
    parameter of function \sortBy\ in module \Data.\List\ can be
    avoided.

\end{itemize}

With instance modularization, programmers need to be aware of which
entities are exported and imported --- i.e.~which entities are visible
in the scope of a module --- {\em and their types}, in particular
whether they are or not overloaded. A simple change like a type
annotation for a variable exported from a module, can lead to a change
in the semantics of using this variable in another module.

% The change is very significant if the type annotation instantiates
% the type of $x$ so that overloading of $x$ is resolved in $A$, since
% this leads to a change in the type (and meaning) of $x$ in module
% $B$.







% \section{Ambiguity Rule}
\label{sec:ambig}



\section{Mini-Haskell with Optional Type Classes}
\label{Optional-type-classes}

In this section we present a type system for mini-Haskell, where type
class declaration is optional. Programmers can overload symbols
without declaring their types in type classes. The type of an
overloaded symbol is, if not explicitly defined in a type class, based
on the anti-unification of instance types defined for the symbol in
the relevant module.

Figure \ref{fig:mini-Haskell-context-free-syntax} shows meta-variable
usage and the context-free syntax of mini-Haskell: expressions and
their types, modules and programs. Meta-variables can be possibly
primed or subscripted. An instance can be specified without specifying
a type class, cf.~second option (after {\tt |}) in Figure
\ref{fig:mini-Haskell-context-free-syntax}.

% (the use of the keywork \instance\ is not strictly
%necessary, it is included here for the sake of simplicity).

For simplicity and following common practice, kinds are not considered
in type expressions and type expressions which are not simple types
are not explicitly distinguished from simple types. Type expression
variables are called simply type variables. 

As usual, we assume the existence of type constructor $\to$, that is
written as an infix operator ($\tau \to \tau'$). A type
$\forall\,\overline{a}.\,C\Rightarrow \tau$ is equivalent to
$C\Rightarrow \tau$ if $\overline{a}$ is empty and, similarly,
$C\Rightarrow \tau$ is equivalent to $\tau$ if $C$ is empty.

For simplicitly, imported and exported variables and instances must be
explicitly indicated, e.g.~we do not include notations for exporting
and importing all variables of a module.

Multi-parameter type classes are supported. In this papert we do not
consider recursivity, neither in let-bindings nor in instance
declarations. 

\begin{figure} 

\[ \begin{array}[c]{llll}
\textrm{Class Name}         &\hspace{.1cm} & A,B            & \\
\textrm{Module Name}        &              & M,N            & \\
\textrm{Type variable}      &         & a,b,\alpha,\beta & \\
\textrm{Type constructor}   &         & T              & \\
\textrm{Simple Constraint}  &         & \pi            & ::= A\,\overline{\tau}\\
\textrm{Unquantified Constraint} &    & \psi           & ::= C\Rightarrow \pi\\
\textrm{Constraint}         &         & \theta         & ::= \forall\,\overline{\alpha}.\,\psi\\
\textrm{Set of Simple Constraints} &  & C,D            & \\
\textrm{Constrained Type}   &         & \delta,\epsilon & ::= C\Rightarrow \tau\\
\textrm{Simple Type}        &         & \tau, \rho     & ::= \alpha \mid T \mid \tau\:\tau' \\
\textrm{Type}               &         & \sigma         & ::= \forall\,\overline{\alpha}.\,\delta\\
\textrm{Program Theory}     &         & P,Q            &\\
\textrm{Variable}           &         & x, y, z        &\\
\textrm{Expression}         &         & e              & ::= x \,\mid\, \lambda x.\,e  \,\mid\, e\:e' \,\mid\, \mlet x = e\,\iin\,e'\\ 
\textrm{Program}            &         & p              & ::= \overline{m}\\
\textrm{Module}             &         & m              & ::= \module\, M\, (\exportC) \ \where\ \overline{\importC};\: \overline{\!d}\\
\textrm{Export clause}      &         & \exportC\      & ::= \overline{\iitem}\\
\textrm{Import clause}      &         & \importC\      & ::= \import\ M\: (\,\overline{\iitem}\,)\\
\textrm{Item}               &         & \iitem         & ::= x \\ % \,\mid\, \instance\ A\: \overline{\tau}\\ 
\textrm{Declaration}        &         & d              & ::= \classDecl \,\mid\, \instDecl \,\mid\, \overline{b}\\
\textrm{Class Declaration}  &         & \classDecl\    & ::= \class\ C \Rightarrow A\: \overline{a}\:\: \where\ \overline{x:\delta}\\
\textrm{Instance Declaration} &       & \instDecl\     & ::= \instance\ C \Rightarrow A\: \overline{\tau}\:\: \where\ \overline{b} 
                                                             \,\mid\, \instance\ b\\
\textrm{Binding}              &       & b              & ::= x = e 
\end{array} \]
\caption{Context-free syntax of mini-Haskell and types}
\label{fig:mini-Haskell-context-free-syntax}
\end{figure}

A program theory $P$ is a set of axioms of first-order logic,
generated from class and instance declarations occurring in the
program, of the form $C \Rightarrow \pi$, where $C$ is a set of simple
constraints and $\pi$ is a simple constraint (cf.~Figure
\ref{fig:mini-Haskell-context-free-syntax}).

Entailment of a set of constraints $C$ by a program theory $P$ is
written as $P \entail C$ (see
e.g.~\cite{JBCS-Ambiguity-and-constrained-polymorphism}).

Typing contexts are indexed by module names. $\Gamma(M)$ gives a
function on variable names to types: $\Gamma(M)(x)$ gives the type of
$x$ in module $M$ and typing context $\Gamma$.
   
A special, empty module name, denoted by $\texttt{[]}$, is used for
names exported by modules, to control the scope of names that use
import and export clauses. Also, a reserved name $\gamma$ is used to
refer to the current module, being defined and used in the type system
and relations to control import and export clauses.

It is not necessary to save multiple instance types for the same
variable in a typing context, neither it is necessary to use instance
types in typing contexts (they are needed only in the program theory);
only the lcg of instance types is used, because of lcg
compositionality (theorem \ref{theorem:lcg-is-compositional}). When a
new instance is declared, if it is an instance of a declared class the
type systems guarantees that each member is an instance of the type
declared in the type class; otherwise (i.e.~it is the single member of
an undeclared class), its (new) type is given by the lcg of the
existing type (an existing lcg of previous instance types) and the
instance type.

A partial order on possibly constrained and possibly quantified types,
based on constraint set entailment, is defined in Figure
\ref{fig:type-partial-order}. We use $P \entail C$ to abbreviate
$P\entail \pi$ for all $\pi\in C$.  Note that type ordering disregards
constraint set satisfiability.  Satisfiability is only important when
considering whether a constraint set $C$ can be removed from a
constrained type $C,D \Rightarrow \tau$ ($C$ can be removed if and
only if overloading for $C$ has been resolved and there exists a
single satisfying substitution for
$C$)\cite{JBCS-Ambiguity-and-constrained-polymorphism}.

\begin{figure}
   \[ \begin{array}{ccc}
   	\displaystyle\frac
          {P \entail \phi\,C \:\:\:\:\: D \subseteq \phi\,C \:\:\:\:\: \overline{b} \subseteq \tv(D) \cup \tv(\phi\,\tau) }
          {\forall\,\overline{a}.\,C \Rightarrow \tau \leq_P \forall\,\overline{b}.\,D \Rightarrow \phi\,\tau}
  \end{array} \]
\caption{Partial order on Types}
\label{fig:type-partial-order}
\end{figure}

A type system for core-Haskell is presented in Figure
\ref{fig:core-haskell-type-system}, using rules of the form $P;\Gamma
\vdash_0 e:\delta$, which means that $e$ has type $\delta$ in typing
context $\Gamma$ and program theory $P$.

Rule (\LET) performs constraint set simplification before type
generalization. Constrait set simplification $\simplifies{P}$ is a
relation on cons\-traints, defined as a composition of improvement and
context reduction \cite{JBCS-Ambiguity-and-constrained-polymorphism}.

In rule (\APP), the constraints on the type of the result are those
that occur in the function type plus not all constraints that occur in
the type of the argument but only those that have variables reachable
from the set of variables that occur in the simple type of the result
or in the constraint set on the function type.  This allows, for
example, not including constraints on the type of the following
expressions, where $o$ is any expression, with a possibly non-empty
set of constraints on its type: {\tt \flip\ \const\ $o$} (where
\const\ has type $\forall a, b.\,a \rightarrow b \rightarrow a$ and
\flip\ has type $\forall a, b, c.\,(a \rightarrow b \rightarrow c)
\rightarrow b \rightarrow a\rightarrow c$), which should denote an
identity function, and \fst\ ($e$, $o$), which should have the same
denotation as $e$.

A variable $a\in \tv(C)$ is called reachable from, or with respect to,
a set of type variables $V$ if $a\in V$ or if $a\in \pi$ for some
$\pi\in C$ such that there exists $b\in \tv(\pi)$ such that $b$ is
reachable. $a\in \tv(C)$ is called unreachable if it is not
reachable. The set of reachable and unreachable type variables of
constraint set $C$ {\em from V\/} is denoted by $\reachableVars(C,V)$.

%We also say that type variables $W$ are reachable in constrained type
%$C\Rightarrow \tau$ if $W\subseteq \reachableVars(C,\tv(\tau))$ (and
%similarly for unreachable type variables and if $W$ is a type variable
%instead of a set of type variables).

$C \oplus_V D$ denotes the constraint set obtained by adding to $C$
constraints from $D$ that have type variables reachable from $V$:
  \[ P \oplus_V Q = P \cup \{ \psi \in Q \mid \tv(\psi) \cap \reachableVars(Q,V) \not= \emptyset \} \]

$\gen(\delta,\sigma,V)$ holds if
  $\sigma=\forall\,\overline{\alpha}.\,\delta$, where
  $\overline{\alpha}=\tv(\delta) - V$.

A type system for mini-Haskell, which extends core-Haskell with
(optional) type classes, modules and modularized instance
declarations, is presented in Figures
\ref{fig:mini-haskell-module-rule} and
\ref{fig:mini-haskell-rules-for-declarations}. Rule (\MODULE) uses
relations ($\vdash_{\Downarrow}$) and ($\vdash_{\Uparrow}^X$), which
are defined separately, for clarity, in Figures
\ref{fig:import-relation} and
\ref{fig:mini-haskell-rules-for-declarations}.

The import relation $\Gamma \vdash_{\Downarrow} \overline{\importC}
\leadsto \Gamma'$ yields a typing context ($\Gamma'$) from a typing
context ($\Gamma$) and a sequence of import clauses
($\overline{\importC}$).

Relation $P;\Gamma \vdash_{\Uparrow}^X \overline{\!d}:(R,P',\Gamma')$
is used for specifying the types a sequence of bindings, from a typing
context ($\Gamma$), a program theory ($P$) and a set of exported names
($X$); it yields a record of elements ($R)$ of the form $(x:\sigma)$,
together with both i) a new typing context ($\Gamma'$), modified to
contain elements of $X$, so that $\Gamma'(\texttt{[]})$ contains the
types of each $x\in X$, and ii) a new program theory ($P'$), updated
from class and instance declarations.  Relation $(\vdash_0)$ is used
to check that expressions of core-Haskell that occur in declarations
are well-typed.

% and updates the typing context if these declarations are exported,
%and updates the program theory from class and instance declarations.

There must exist a sequence of derivations for typing a sequence of
modules that composes a program that starts from an empty typing
context, or from a typing context that corresponds to predefined
library modules. Recursive modules are not treated in this paper.

Concatenation of record elements, denoted by {\tt (++)}, is used in
Figure \ref{fig:mini-haskell-rules-for-declarations}: $\{ x_1, \ldots,
x_n \} \texttt{++} \{ y_1, \ldots, y_m \} = \{ x_1, \ldots, x_n, y_1,
\ldots, y_m \}$.

%If $R = \overline{x:\delta}^{\,n}$ and $R'
% =\overline{y:\epsilon}^{\,m}$, then $R \oplus R'$ denotes the record
% $\{ x_1,:\delta_1, \ldots, x_n:\delta_n, y_1:\epsilon_1, \ldots,
% y_m:\delta_m \}$.

\begin{figure}
\[ \begin{array}{cc}
      \displaystyle\frac
        {\begin{array}[t]{lll}
           \Gamma(\gamma)(x) = \sigma\ & \:\sigma \leq_P \delta
         \end{array}}
        {P;\Gamma \vdash_0 x: \delta} \:(\VAR) \\ \\

	\displaystyle\frac
          {(\Gamma(\gamma),x \mapsto \tau) \vdash_0 e: C \Rightarrow \tau'}
	  {P;\Gamma \vdash_0 \lambda x.\:e: C\Rightarrow \tau \rightarrow \tau'} \:(\ABS)  \\ \\

	\displaystyle\frac
	  {\begin{array}[t]{cc}
             P;\Gamma \vdash_0 e: C \Rightarrow \tau' \rightarrow \tau\: &\:
             P;\Gamma \vdash_0 e': C' \Rightarrow \tau' \\
             V = \tv(\tau) \cup \tv(C) & (C \oplus_V C') \simplifies{P} D
           \end{array}}
	{P;\Gamma \vdash_0 e\:e': D\Rightarrow \tau} \:(\APP) \\ \\

	\displaystyle\frac
	 {\begin{array}{ll}
            P;\Gamma \vdash_0 e\!:C \Rightarrow \tau & C \simplifies{P} D\\
             \gen(D\Rightarrow \tau,\sigma,\tv(\Gamma))\: & \:P;(\Gamma(\gamma),x \mapsto\sigma) \vdash_0 e'\!:\delta
          \end{array}}
	 {P;\Gamma \vdash_0 \mlet\ x=e\ \iin\ e':\delta} \:(\LET)
\end{array} \]
\caption{Core-Haskell Type System}
\label{fig:core-haskell-type-system}
\end{figure}

\begin{figure}
\[ \begin{array}{cc}
	\displaystyle\frac
	 {\begin{array}{ll}
           \Gamma_0 \vdash_{\Downarrow} \overline{\!I} : \Gamma\:\: & \:\:P;\Gamma \vdash_{\Uparrow}^X \overline{\!d} : (R,P',\Gamma') \}
          \end{array}}
	 {P;\Gamma_0 \vdash \module\ M\, (\exportC)\ \where\ \overline{\!I};\, \overline{\!d} : R} \:(\MODULE)
\end{array} \]
\caption{Mini-Haskell module rule} 
\label{fig:mini-haskell-module-rule}
\end{figure}

\begin{figure}
\[ \begin{array}{cc}
	\displaystyle\frac
	 {\begin{array}{ll}
            \Gamma'(M)(x) = \left\{ \begin{array}{ll}
              \Gamma(\texttt{[]})(x) & \text{ if } M = \gamma \text{ and } x = \iota_k, %\text{ for some } 
                                                                           1 \leq k \leq n\\
%                                     &                                    & x \text{ is the single member of such } \iota_k)\\
               \Gamma(M)(x)          & \text{ otherwise}
             \end{array}\right.
          \end{array}}
	 {\Gamma \vdash_{\Downarrow} \import\ M\, (\,\overline{\iitem}^{\,n}\,) : \Gamma'} \\ \\

	\displaystyle\frac
	 {\begin{array}{ll}
	   \Gamma_0 \vdash_{\Downarrow} \import\ M\, (\,\overline{\iitem}\,) : \Gamma \:\:\: & \:\:\: 
           \Gamma \vdash_{\Downarrow} \overline{\importC} : \Gamma'
          \end{array}}
	 {\Gamma_0 \vdash_{\Downarrow} \import\ M\, (\,\overline{\iitem}\,); \overline{\importC} : \Gamma'} 
\end{array} \]
\caption{Import relation}
\label{fig:import-relation}
\end{figure}

\begin{figure}
\[ \begin{array}{cc}
	\displaystyle\frac
	 {\begin{array}{ll}
            Q;\Gamma \vdash_{\Uparrow}^X \overline{\!d} : (R, Q', \Gamma') \:\: & \:\:
            Q = P \cup \left\{ \begin{array}{ll}
                                  \{ C \Rightarrow A\:\overline{a} \} & \text{ if } C \not= \emptyset \\
                                  \emptyset                              & \text{ otherwise}
                                \end{array}\right. \\
            \multicolumn{2}{c}{
                \Gamma(M)(x) = \left\{ \begin{array}{ll}
                                 \delta_k       & \text{ if } x = x_k, 1 \leq k \leq n, \text{ and } 
                                                              M \in \{ \gamma, \texttt{[]} \}\\
                                 \Gamma_0(M)(x) & \text{ otherwise} 
                               \end{array}\right. }
          \end{array}}
	 {P;\Gamma_0 \vdash_{\Uparrow}^X \class\ C \Rightarrow A\: \overline{a}\ \where\ \overline{x:\delta}^{\,n};\: \overline{\!d} : 
            (R, Q', \Gamma') } \\ \\

	\displaystyle\frac
	 {\begin{array}{ll}
               Q = P \cup \left \{ \begin{array}{ll}
                         \{ \forall\,\overline{a}.\,C \Rightarrow \pi \} & \text{ if } C \not= \emptyset \\
                         \{ \forall\,\overline{a}.\,\pi \} & \text{ otherwise}
                                \end{array}\right. \: & \: 
               P \entail C \Rightarrow \pi \\
%           \multicolumn{2}{c}{\gen(\delta_k,\sigma_k,\tv(\Gamma_0)), \text{ for } 1\leq k \leq n} \\
%           \multicolumn{2}{c}{
%             \Gamma(M)(y) = \left\{ \begin{array}{ll}
%                                    \sigma_k & \text{ if } y = x_k \text{ and } (M = \gamma \text{ or } 
%                                                                                 M = \texttt{[]} \text{ and } x_k \in X)\\
%                                    \Gamma_0(M)(x) & \text{ otherwise} 
%                            \end{array}\right.}\\[.3cm]
           Q;\Gamma \vdash_0 \overline{e}^{\,n} : \overline{\delta}^{\,n} \: & \: 
	   Q;\Gamma \vdash_{\Uparrow} \overline{\!d} : (R,Q',\Gamma') 
          \end{array}}
	 {P;\Gamma \vdash_{\Uparrow}^X \instance\ C \Rightarrow \pi\ \where\ \overline{x = e}^{\,n};\: \overline{\!d} 
             : (\overline{x : \delta}^{\,n} ++ R,Q',\Gamma')}\\ \\

	\displaystyle\frac
	 {\begin{array}{ll}
            P; \Gamma \vdash_0 e : C \Rightarrow \tau \:\: & \:\:
            \lcgR(\{ \tau\} \cup \Gamma_0(\gamma)(x), \tau') \\[.1cm]
            \multicolumn{2}{c}{\Gamma(M)(y) = \left\{ \begin{array}{ll}
                                      \tau' & \text{ if } y = x, (M = \gamma \text{ or }
                                                                 (M = \texttt{[]}, x \in X)) \\
                                      \Gamma_0(M)(y) & \text{ otherwise } 
                                   \end{array}\right.}\\[.2cm]                   
           \multicolumn{2}{c}{P;\Gamma \vdash_{\Uparrow}^X \overline{\!d} : (R,Q',\Gamma')} 
          \end{array}}
	 {P;\Gamma_0 \vdash_{\Uparrow}^X \instance\ x = e;\: \overline{\!d} : (\{x:C\Rightarrow \tau\} \texttt{++} R,Q',\Gamma')} \\ \\

	\displaystyle\frac
	 {\begin{array}{ll}
           P;\Gamma_0 \vdash_0 e : \delta \: & \: \gen(\delta,\sigma,\tv(\Gamma_0))\\
           \multicolumn{2}{c}{\Gamma(M)(y) = \left\{ \begin{array}{ll}
                                    \sigma & \text{ if } y = x, (M = \gamma \text{ or }
                                                                 (M = \texttt{[]}, x \in X)) \\
                                      \Gamma_0(M)(y) & \text{ otherwise } 
                                   \end{array}\right. }\\[.2cm]                   
           \multicolumn{2}{c}{P;\Gamma \vdash_{\Uparrow}^X \overline{\!d} : (R,P',\Gamma')}
          \end{array}}
	 {P;\Gamma_0 \vdash_{\Uparrow}^X x = e;\: \overline{\!d} : (\{x:\delta\} \texttt{++} R,P',\Gamma')} 
\end{array} \]
\caption{Mini-Haskell rules for declarations}
\label{fig:mini-haskell-rules-for-declarations}
\end{figure}



\section{Records with overloaded fields}
\label{sec:overloaded-record-fields}



% \input{type-directed-named-resolution}

\section{Related Work}
\label{sec:related-work}

Haskell type system has been extended with several advanced typing features
such as functional dependencies~\cite{Jones2008}, type families~\cite{Chakravarty2005} and
GADTs~\cite{Chen2016}, just to name a few. To the best of our knowledge, there's no
previous work on optional declaration of type classes. In this section, we
summarize some recent Haskell type system extensions.

Functional dependencies (FDs) were introduced by Mark Jones as a way to specify
type class parameter dependencies to improve infered typesin the
context of MPTCs. FDs where also used to support some form of type level
programming~\cite{Hallgren2000} and to define heterogeneous lists and
extensible records~\cite{KiselyovLS04}.

Type families~\cite{Chakravarty2005} (TFs) where introduced as a ``more functional'' alternative
to FDs, that is relational in nature. However, there are some issues with
type families injectivity~\cite{Eisenberg2014} that motivated the so-called closed type
families and type families dependencies~\cite{Eisenberg2014a}. Closed type families define all
possible instances of a type family a priori and type families dependencies allows
the specification of parameter dependencies, in a similar fashion of FDs.
All type families related extensions cater to better type improvement.

Datatype promotion~\cite{Yorgey2012,Eisenberg2014} is useful Haskell extension that lifts user
defined algebraic datatypes to kinds and data constructors to types. Such
extension allows for defining some types dependently typed programs.
Singleton types and promoted functions~\cite{Eisenberg2012} have been used to automate
(through Template Haskell) some constructions commonly needed in Haskell-style
dependent types. Lindley and McBride~\cite{Lindley2013} describes some dependently
typed programs in Haskell and how to use GHC's constraint solver as theorem
prover to discharge proof obligations in a implementation of a merge-sort
algorithm.

Type level literals~\cite{type-lits} is an extension that complements datatype
promotion to numeric and string types. The Haskell prime proposal for
overloaded records relies on this extension to overload field
access and update functions. Our approach, based on optional declaration of
type classes, is simpler since it doesn't demand type promotion features and
it doesn't need to create an instance for each record field (overloaded or not)
as GHC's proposal.


\section{Conclusion}
\label{sec:conclusion}

This paper has presented an approach for allowing programmers to
overload symbols without declaring their types in type classes. In
this approach, the type of an overloaded symbol is automatically
determined from the anti-unification of instance types defined for the
symbol in the relevant module.

The paper explores this approach in the presence of instance
modularization and an ambiguity rule that is defined differently than
in Haskell.

The approach allows, for example, overloaded record fields and type
directed name resolution to be supported in a simple way.




\bibliographystyle{plain}
\bibliography{main}

\end{document}
