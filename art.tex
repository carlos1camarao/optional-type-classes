\documentclass{llncs}
\usepackage{amsmath,amssymb}
\usepackage{makeidx}  % allows for indexgeneration
%

\usepackage[american]{babel}
\usepackage[utf8]{inputenc}

\usepackage{float}
\usepackage{amssymb}
\usepackage{color}
\usepackage{fancyhdr}
%\usepackage{minted}

\begin{document}

%% commands used

\newtheorem{Lemma}{Lemma}
\newtheorem{Theorem}{Theorem}
\newtheorem{Definition}{Definition}
\newtheorem{Corollary}{Corollary}
\newtheorem{Example}{Example}

%% notations

\newcommand{\id}{\text{\it id\/}}
\newcommand{\dom}[1]{\ensuremath{\textit{dom}(#1)}}
\newcommand{\unify}{\text{\it unify\/}}
\newcommand{\tv}{\text{\it tv\/}}
\newcommand{\gtv}{\text{\it gtv\/}}
\newcommand{\gtc}{\text{\it gtc\/}}
\newcommand{\rtv}{\text{\it rtv \/}}
\newcommand{\specialize}{\text{\it specialize\/}}

%% keywords

\newcommand{\mif}{{\tt if}\ }
\newcommand{\mthen}{\ {\tt then}\ }
\newcommand{\melse}{\ {\tt else}\ }

\newcommand{\mlet}{{\tt let}\ }
\newcommand{\iin}{\ {\tt in}\ }
\newcommand{\mwhere}{{\tt where}\ }
\newcommand{\mcase}{{\tt case}\ }
\newcommand{\mof}{{\tt of}\ }


%% haskell code

%\newcommand{\haskell}[1]{\mintinline{haskell}{#1}}
\newcommand{\haskell}[1]{\texttt{#1}}

\input{meta.keys}

%
\mainmatter              % start of the contributions

\title{Optional Type Classes for Haskell}

%
\author{Rodrigo Ribeiro \inst{1} \and Carlos Camar\~ao\inst{2} \and Luc\'ilia
Figueiredo\inst{3} \and Cristiano Vasconcellos\inst{4}}
\institute{DECSI, Universidade Federal de Ouro Preto (UFOP), João Monlevade\\
\email{rodrigo@decsi.ufop.br} \and
DCC, Universidade Federal de Minas Gerais (UFMG), Belo Horizonte\\
\email{camarao@dcc.ufmg.br} \and
DECOM, Universidade Federal de Ouro Preto (UFOP), Ouro Preto\\
\email{luciliacf@gmail.com} \and
DCC, Universidade do Estado de Santa Catarina (UDESC), Joinville\\
\email{cristiano.vasconcellos@udesc.br}
}
\maketitle              % typeset the title of the contribution

\begin{abstract}

This paper explores an approach for allowing type classes to be
optionally declared by programmers, i.e. programmers can overload
symbols without declaring their types in type classes.

The type of an overloaded symbol is, if not explicitly defined in a
type class, automatically determined from the anti-unification of
instance types defined for the symbol in the relevant module. A type
class having the overloaded name as its unique member is automatically
created from this type.

This depends on a modularization of instance visibility, as well as on
a redefinition of Haskell's ambiguity rule. The paper presents the
modifications to Haskell's module system that are necessary for
allowing instances to have a modular scope, based on previous work by
the authors. The definition of the type of overloaded symbols as the
anti-unification of available instance types and the redefined
ambiguity rule is also based on previous works by the authors.

The added flexibility to Haskell-style of overloading is illustrated
by defining a type system and a type inference algorithm that allows
overloaded record fields.

% The redefinition of Haskell's ambiguity rule can also be used to
% address some of the issues related to type directed name resolution.

\end{abstract}

\section{Introduction}
\label{sec:intro}

The versions of Haskell supported by GHC \cite{GHC} --- the prevailing
Haskell compiler --- are becoming complex, to the point of affecting
the view of Haskell as the best choice for general-purpose software
development. A basic issue in this regard is the need of extending the
language to allow multiple parameter type classes (MPTCs). This
extension is thought to require additional mechanisms, such as
functional dependencies or type families. In another paper
\cite{JBCS-Ambiguity-and-constrained-polymorphism}, we have shown that
the introduction of MPTCs in the language can be done without the need
of additional mechanisms: a simplifying change is sufficient, to
Haskell's ambiguity rule. Interested readers are referred to
\cite{JBCS-Ambiguity-and-constrained-polymorphism}. The main ideas are
summarized below.

Haskell with MPTCs uses constrained types of the form $\forall
\overline{a}.\,C \Rightarrow \tau$, where $C$ is a set of constraints
and $\tau$ is a simple (unconstrained, unquantified) type; a
constraint is a class name followed by a sequence of type variables.

In (GHC) Haskell, ambiguity is a property of a type: a type $\forall
\overline{a}.\,C \Rightarrow \tau$ is ambiguous if there exists a type
variable that occurs in the constraints ($C$) that is not uniquely
determined from the set of type variables that occur in the simple
type ($\tau$). This unique determination specifies that, for each type
variable $a$ that occurs in $C$ but not in $\tau$ there must exist a
functional dependency $b \mapsto c$ for some $b$ in $\tau$ (or a
similar unique determination specified via type families). Notation $b
\mapsto c$ is used, instead of $b \rightarrow c$, to avoid confusion
with the notation used to denote functional types.

A new definition, which we prefer to call here {\em expression
  ambiguity\/} (in \cite{JBCS-Ambiguity-and-constrained-polymorphism}
it is called {\em delayed closure ambiguity\/}), uses a similar
property, of variable reachability, that is independent of functional
dependencies and type families: a type variable $a$ that occurs in a
set of constraints is reachable from the set of type variables in
$\tau$ if it occurs in $\tau$ or there exists a type variable $b$ in
$C$ that is reachable. For example, in $C \Rightarrow b$, where
$C=(D\: a\: b, E\: a)$, type variable $a$ is reachable from the set of
type variables in $b$.

The presence of unreachable variables in a constraint $\pi\in C$
characterizes overloading resolution, or, in other words, that
overloading is resolved for $\pi$: it characterizes that there is no
context in which an expression with such a type could be placed that
could instantiate such unreachable variables. The presence of
unreachable variables does not necessarily imply ambiguity. Ambiguity
is a property of an expression, and it depends on the context in which
the expression occurs, and on entailment of the constraints on the
expression's type.

Entailment of constraints and its algorithmic (functional) counterpart
are well-known in the Haskell world (see
e.g.~\cite{MarkJones94a,TheoryOfOverloading,JBCS-Ambiguity-and-constrained-polymorphism}).

Informally, a set of constraints $C$ is entailed (or satisfied) in a
program $P$ if there exists a substitution $\phi$ such that $\phi(C)$
is contained in the set of instance declarations of $P$, or is implied
by the transitivity implied by the set of class and instance
declarations occuring in $P$. For a formal definition, see
e.g.~\cite{MarkJones94a,JBCS-Ambiguity-and-constrained-polymorphism}. In
this case we say that $C$ is entailed by $\phi(C)$. 

For example, {\tt \Eq\ [[\Integer]]} is entailed if we have instances
{\tt \Eq\ \Integer} and {\tt \Eq\ $a$ => \Eq\ [$a$]}, visible in the
context where an expression whose type has a constraint {\tt
  \Eq\ [[\Integer]]} occurs.

If overloading is resolved for a constraint $C$ occurring in a type
$\sigma = C,D \Rightarrow \tau$ then exactly one of the following
holds:
\begin{itemize}

\item $C$ is entailed by a single instance; in this case a type
  simplification (also called ``improvement'') occurs: $\sigma$ can be
  simplified to $D \Rightarrow \tau$;

\item $C$ is entailed by more than instance; in this case we have a
  type error: ambiguity; 

\item $C$ is not entailed (by any instance); in this case we have also
  a type error: unsatisfiability.

\end{itemize}

Note that variables in a single constraint are either all reachable or
all unreachable. If they are unreachable, either the constraint can be
removed, in the case of single entailment, or there is a type error
(either ambiguity, in the case of two or more entailments, or
unsatisfiability, in the case of no entaiment).

Instead of being dependent on the specification or not of functional
dependencies or type families, ambiguity depends on the existence of
(two or more) instances in a program context when overloading is
resolved for a constraint on the type of an expression.

The possibility of a modular control of the visibility of instance
definitions conforms to this simplifying change. This is the subject
of Section \ref{sec:modular-instances}.

Also in conformance with this change is the possibility, explored in
this paper, of allowing type classes to be optionally declared by
programmers, i.e. for allowing programmers to overload symbols without
having to declare the types of these symbols in type classes. 

A type system and a type inference algorithm for a core-Haskell
language where type classes can be optionally declared is presented in
Section \ref{Optional-type-classes}.  The idea is based on defining
the type of unanottated overloaded symbol as the anti-unification of
instance types defined for the symbol in a module, by automatically
creating a type class with a single overloaded name. This depends on a
modularization of instance visibility (as well as on a redefinition of
Haskell's ambiguity rule).

The paper presents the modifications to Haskell's module system that
are necessary to allow instances to have a modular scope, based on
previous work published by one of the authors. The definition of the
type of overloaded symbols as the anti-unification of available
instance types and the redefined ambiguity rule is also based on
previous works by the authors.

The added flexibility to Haskell-style of overloading is illustrated
by defining a type system and a type inference algorithm that allows
overloaded record fields (Section \ref{sec:overloaded-record-fields}).

% The redefinition of Haskell's ambiguity rule can also be used to
% address some of the issues related to type directed name resolution
% (Section \ref{sec:type-directed-name-resolution}).



\section{Preliminaries}\label{prelimirares}

In this section we introduce some basic definitions and notations. We
consider that meta-variables defined can appear primed or subscripted.

Meta-variable usage is defined in the paper as follows: $x,y$ denote
term variables, $\alpha, \beta$ ($a, b,...$
in examples) type variables, $e$ a term,
$\tau,\rho$ simple types, $\sigma$ a type, $\Gamma$ a typing context, 
that is, a set of pairs written as $x:\sigma$, and $S$ a
substitution. A constraint is formed by a pair of a class name $C$ and
a sequence of types $\overline{\tau}$. We slighly abuse notation and 
use $\kappa$ to denote both a single constraint and a constraint set.

The notation $\overline{a}^{\,n}$, or simply $\overline{a}$, denotes
the sequence $a_1 \cdots a_n$, or $a_1, \ldots, a_n$, or
$a_1;\ldots;a_n$, depending on the context where it is used, where
$n\geq 0$. When used in a context of a set, it denotes
$\{a_1,\ldots,a_n\}$. It can be used with more than one variable; for
example, in $\overline{x = e}^{\,n}$, it denotes the sequence $x_1 =
e_1, \ldots, x_n = e_n$.

A substitution is a function from type variables to simple type
expressions (cf.~Section \ref{Optional-type-classes}). The identity
substitution denoted by \id. $\phi(\sigma)$ (or simply $\phi\,\sigma$)
represents the capture-free operation of substituting $\phi(\alpha)$
for each free occurrence of $\alpha$ in $\sigma$.

We overload the substitution application on constraints, constraint
sets and sets of types. Definition of application on these elements is
straightforward. The symbol $\circ$ denotes function composition and
$\dom{\phi}=\{\alpha \mid\ \phi(\alpha) \neq \alpha\}$.

The notation $\phi[\overline{\alpha}\mapsto\overline{\tau}]$ denotes
the updating of $\phi$ such that $\overline{\alpha}$ maps to
$\overline{\tau}$, that is, the substitution $\phi'$ such that
$\phi'(\beta) = \tau_i$ if $\beta = \alpha_i$, for $i = 1,...,n$,
otherwise $\phi(\beta)$. Also, $[\overline{\alpha}\mapsto\overline{\tau}]
= \id[\overline{\alpha}\mapsto\overline{\tau}]$.

\subsection{Anti-unification of instance types}
\label{sec:anti-unif}

A type $\tau$ is a generalization of a set of simple types
$\overline{\tau}^{\,n}$ if there exist substitutions
$\overline{\phi}^{\,n}$ such that $\phi_i(\tau)=\tau_i$, for
$i=1,\ldots,n$. A generalization is also called a (first-order) {\em
  anti-unification\/} \cite{ModelTheory2012}.

We say that $\tau'$ is less general than $\tau$, written $\tau \leq
\tau'$, if there exist $\phi$ such that $\phi(\tau) = \tau'$.

The {\it least common generalization} (lcg) of a set of types
$\mathbb{T}$ and a type $\tau$ holds, written as
$\lcgR(\mathbb{T},\tau)$, if, for all generalizations $\tau'$ of
$\mathbb{T}$, we have $\tau \leq \tau'$.

An algorithm for computing the lcg of a finite set of types in
presented in Figure \ref{fig:lcg}. The concept was studied by Gordon
Plotkin \cite{plotkin1970note,plotkin1971further}, that defined a
function for constructing a generalization of two symbolic
expressions.  In Figure~\ref{fig:lcg}, we present function \lcg, that
gives the lcg of a finite set of simple types by recursion on the
structure of set $\mathbb{T}$, using a function to compute the
generalization of two simple types. For two types $\tau_1$ and
$\tau_2$ the idea is to recursively traverse the structure of both
types using a finite map to store previously generalized
types. Whenever we found two different type constructors, we search on
the finite map if they have been previously generalized. If this was
the case, the generalization is returned. But if these two type
constructors aren't in the finite map we insert them using a fresh
type variable as their generalization and return this new variable.

\begin{figure*}[ht]
	\[\progfig{
            $\lcg(\mathbb{T})=\tau$ $\:\:\:$ where 
               $(\tau, \phi)=\lcg'(\mathbb{T},\id)$, for some  $\phi$ \\ \\
            $\lcg'(\{\tau\},\phi) = (\tau, \phi)$  \\ \\		
            $\lcg'(\{\tau_1\} \cup \mathbb{T}, \phi) = \lcg''(\tau_1, \tau',\phi') \:\:\:$ where
		$\begin{array}[t]{ll}
                   (\tau',\phi')  & = lcg'(\mathbb{T}, \phi)
		\end{array}$  \\ \\		
            xxx\=xxx\=xxx\=xxx\=xxxxx\=xxxxxx\=xxxxxxxx\= \kill
            $\lcg'' (C \: \overline{\tau}^{\,n},\:  D\: \overline{\rho}^{\,m},\phi)=$\+\\
              \textbf{if}\ $\phi(\alpha)=( C\:\overline{\tau}^{\,n},\: D\:\overline{\rho}^{\,m})$
                      for some $\alpha$ \textbf{then}\ $(\alpha,\phi)$ \\
              \textbf{else} \+\\
              \textbf{if}\ $n\not=m$ \textbf{then}\
                 $(\beta, \phi [\beta \mapsto ( C \:\overline{\tau}^{\,n},\: D\:\overline{\rho}^{\,m})])$ \+ \\
		 where $\beta$ is a fresh type variable \-\\[.1cm]
              \textbf{else}\ $(\psi\: \overline{\tau'}^{\,n}, \phi_n)$\+\\
                 where $\begin{array}[t]{l}
		          (\psi,\phi_0) = \left\{\begin{array}{ll}
                                            (C ,\phi) & \textbf{if } C = D \\
                                            (\alpha, \phi\,[\alpha\mapsto (C, D)])
                                                      & \text{otherwise, $\alpha$ is fresh }\\
                                          \end{array}\right. \\[.3cm]
                          (\tau'_i,\phi_i) = lcg''(\tau_i, \rho_i, \phi_{i-1}), \text{ for } i=1, \ldots, n
                        \end{array}$ \-\-\-	
        }
        \]
\caption{Least Common Generalization} \label{fig:lcg}
\end{figure*}
As an example of the use of \lcg, consider the following types (of
functions \map\ on lists and trees, respectively):

\progb{
   $(a \rightarrow b)$ $\rightarrow$ [$a$] $\rightarrow$ [$b$]\\
   $(a \rightarrow b)$ $\rightarrow$ \Tree\ $a$ $\rightarrow$ \Tree\ $b$
}

A call of \lcg\ for a set with these types yields type $(a \rightarrow
b) \rightarrow c\:\: a \rightarrow c\:\: b$, where $c$ is a
generalization of type constructors {\tt []} and \Tree.

We have: 

\begin{Theorem}[Soundness of \lcg]
For all (sets of simple types) $\mathbb{T}$, we have that
$\lcg(\mathbb{T})$ yields a generalization of $\mathbb{T}$.
\label{theorem:lcg-is-sound}
\end{Theorem}

\begin{Theorem}[Completeness of \lcg]
For all (sets of simple types) $\mathbb{T}$, we have that
$\lcgR(\mathbb{T},\lcg(\mathbb{T}))$ holds, i.e.~if $\tau$ is a
generalization of $\mathbb{T}$ then $\lcg(\mathbb{T}) \leq \tau$.
\label{theorem:lcg-is-complete}
\end{Theorem}

\begin{Theorem}[Compositionality of \lcg]
For all non-empty (sets of simple types) $\mathbb{T}, \mathbb{T'}$, we
have that $\lcg(\lcg(\mathbb{T}),\lcg(\mathbb{T'})) = \lcg(\mathbb{T}
\cup \mathbb{T'})$.
\label{theorem:lcg-is-compositional}
\end{Theorem}

\begin{Theorem}[Uniqueness of \lcg]
For all (sets of simple types) $\mathbb{T}$, we have that
$\lcg(\mathbb{T})$ is unique, up to variable renaming.
\label{theorem:lcg-is-unique-modulo-variable-renaming}
\end{Theorem}



The proofs use straighforward induction on the number and complexity
of elements of $\mathbb{T}$.




\section{Modularization of Instances}
\label{sec:modular-instances}

This paper does not attempt to discuss any major revision to Haskell's
module system. We summarize in subsection
\ref{subsec:instance-visibility-control} the work, presented in
\cite{Controlling-scope-instances}, that allows a modular control of
the visibility of instance definitions. This has the additional
benefit of enabling type classes to be optionally declared by
programmers, by the introduction of a single additional rule (to
account for the possibility of type classes to be declared or not):

\begin{definition}[Type of overloaded variable]

If the type of an overloaded variable --- i.e.~a variable that is
introduced in an instance definition --- is not explicitly annotated
in a type class declaration, then the variable's type is the
anti-unification of instance types defined for the variable in the
current module; otherwise, it is the annotated type.

\label{overloaded-variable-type}
\end{definition}

Instance modularization and the rule of expression ambiguity, that
considers the context where an expression occurs to detect whether an
expression is ambiguous or not, has profound consequences. Consider,
for example:

\proga{xx\=\kill
\module\ $A$ where\+\\
  \class\ \SShow\ $t$ \ldots\\
  \class\ \RRead\ $t$ \ldots\\
  \instance\ \SShow\ \Int\ \ldots\\
  \instance\ \RRead\ \Int\ \ldots\\
  $f$ = \sshow $\:$.$\:$\rread\-\\ \\

\module\ $B$ \where\+\\
  \import\ $A$\\
  \instance\ \RRead\ \Bool\ \ldots\\
  \instance\ \SShow\ \Bool\ \ldots\\
  $g$ = $f$ "123"
}

The definition of $f$ in module $A$ is well-typed, because constraints
(\SShow\ $a$, \RRead\ $a$) can be removed; this occurs because there
exists a single instance, in module $A$, for each constraint, that
entails it. As a result, $f$ has type \String $\rightarrow$
\String. Its use in module $B$ is (then) also well-typed. That means:
$f$'s semantics is a function that receives a value of type
\String\ and returns a value of type \String, according to the
definition of $f$ given in module $A$. The semantics of an expression
involves passing a (dictionary) value that is given in the context of
usage if, {\em and only if}, the expression has a constrained type.

\subsection{Instance visibility control: a summary}
\label{subsec:instance-visibility-control}

Modularization of instance definitions can be allowed by means of the
importation and exportion of instances as shown in
\cite{Controlling-scope-instances}. Essentially, import and export
clauses can specify, instead of just names, occurrences of {\tt
  instance $A$ $\overline{\tau}$}, where $\overline{\tau}$ is a
(non-empty) sequence of types and $A$ is a class name.  We have:

  \[ \text{\module\ $M$ (\instance\ $A$ $\overline{\tau}$, \ldots) \where\ \ldots} \]
specifies that the instance of $\overline{\tau}$ for class $D$ is
exported in module $M$.

  \[ \text{\import\ $M$ (\instance\ $A$ $\overline{\tau}$, \ldots)} \]
specifies that the instance of $\overline{\tau}$ for class $A$ is
imported from $M$, in the module where the import clause occurs.

Alternatively, we can simply give a name to an instance, in an
instance declaration, and use that name in import and export clauses
(see \cite{Controlling-scope-instances}). However, in this paper we
don't need to give a name to an instance, since we only consider
instances of undeclared classes, which have a single member, and we
can thus use the name of the member as the instance name. 

%For example, 
%we can have:
%  \progb{
%   \instance\ $x$ = '1';\\
%   \instance\ $x$ = \True;
%  }    

\subsection{Pros and Cons of Instance Modularization}

Among the advantages of this simple change, we cite (following
\cite{Controlling-scope-instances}):

\begin{itemize}

  \item programmers have better control of which entities are
    necessary and should be in the scope of each module in a program;

  \item it is possible to define and use more than one instance for
    the same type in a program;

  \item problems with orphan instances (i.e.~instances defined in a
    module where neither the definition of the data type nor the
    definition of the type class) do not occur (for example, distinct
    instances of \Either\ for class \Monad, say one from package
    \mtl\ and another from \transformers, can be used in a program);

  \item the introduction of newtypes, as well as the use of functions
    that include additional (-by) parameters, such as e.g.~the (first)
    parameter of function \sortBy\ in module \Data.\List\ can be
    avoided.

\end{itemize}

With instance modularization, programmers need to be aware of which
entities are exported and imported (i.e.~which entities are visible in
the scope of a module) and their types, in particular if they are
overloaded or not.  A simple change like a type annotation for a
variable exported from a module, can lead to a change in the semantics
of using this variable in another module.

% The change is very significant if the type annotation instantiates
% the type of $x$ so that overloading of $x$ is resolved in $A$, since
% this leads to a change in the type (and meaning) of $x$ in module
% $B$.







% \section{Ambiguity Rule}
\label{sec:ambig}



\section{Mini-Haskell with Optional Type Classes}
\label{Optional-type-classes}

In this section we present a type system for mini-Haskell, where type
class declaration is optional. Programmers can overload symbols
without declaring their types in type classes. The type of an
overloaded symbol is, if not explicitly defined in a type class, based
on the anti-unification of instance types defined for the symbol in
the relevant module. Mini-Haskell extends core-Haskell expressions
(Subsection \ref{sec:core-Haskell}) with class and instance
declarations, allowing type classes to be optionally declared, and
modules, which can export and import names and instances (Subsection
\ref{sec:mini-Haskell}).

Figure \ref{fig:mini-Haskell-context-free-syntax} shows the
context-free syntax of mini-Haskell: expressions, modules and
programs. An instance can be specified without specifying a type
class, cf.~second option (after {\tt |}) in Instance Declaration in
Figure \ref{fig:mini-Haskell-context-free-syntax}.

For simplicity, imported and exported variables and instances must be
explicitly indicated, e.g.~we do not include notations for exporting
and importing all variables of a module.

Multi-parameter type classes are supported. In this paper we do not
consider recursivity, neither in let-bindings nor in instance
declarations. 

\begin{figure} 
\[ \begin{array}[c]{lll}
{\rm Module\ Name}          &    M,N            & \\
{\rm Program\ Theory}       &    P,Q            &\\
{\rm Variable}              &    x, y           &\\
{\rm Expression}            &    e              & ::= x \,\mid\, \lambda x.\,e  \,\mid\, e\:e' \,\mid\, \mlet x = e\,\iin\,e'\\ 
{\rm Program}               &    p              & ::= \overline{m}\\
{\rm Module}                &    m              & ::= \module\, M\, (\exportC) \ \where\ \overline{\importC};\: \overline{\!D}\\
{\rm Export\ clause}        &    \exportC\      & ::= \overline{\iitem}\\
{\rm Import\ clause}        &    \importC\      & ::= \import\ M\: (X)\\
{\rm Item}                  &    \iitem         & ::= x \,\mid\, \instance\ A\: \overline{\tau} \\ 
{\rm Declaration}           &    D              & ::= \classDecl \,\mid\, \instDecl \,\mid\, B\\
{\rm Class\ Declaration}    &    \classDecl\    & ::= \class\ C \Rightarrow A\: \overline{a}\:\: \where\ \overline{x:\delta}\\
{\rm Instance\ Declaration} &    \instDecl\     & ::= \instance\ C \Rightarrow A\: \overline{\tau}\:\: \where\ \overline{B}  \,\mid\, \\
                            &                   & \hspace*{.6cm} \instance\ B\\
{\rm Binding}               &    B              & ::= x = e 
\end{array} \] 
\caption{Context-free syntax of mini-Haskell}
\label{fig:mini-Haskell-context-free-syntax}
\end{figure}

A program theory $P$ is a set of axioms of first-order logic,
generated from class and instance declarations occurring in the
program, of the form $C \Rightarrow \pi$, where $C$ is a set of simple
constraints and $\pi$ is a simple constraint (Figure
\ref{fig:mini-Haskell-context-free-syntax}).

Typing contexts are indexed by module names. $\Gamma(M)$ gives a
function on variable names to types: $\Gamma(M)(x)$ gives the type of
$x$ in module $M$ and typing context $\Gamma$.  The notation
$(\Gamma(M),x \mapsto \sigma)$ is used to denote the typing context
$\Gamma'$ that differs from $\Gamma$ only by mapping $x$ to $\sigma$
in module $M$, i.e.~: $\Gamma'(M')(x') = \sigma$ if $M' = M$ and
$x'=x$, otherwise $\Gamma'(M')(x') = \Gamma(M')(x')$. To avoid
introducing a name for instance definitions, the domain of $\Gamma(M)$
(for any module $M$) can consist of not only normal variable names but
also items of the form {\tt \instance\ $C \Rightarrow A\:\tau$} (where
$C$ is a constraint set $C$, $A$ is a class name, $\tau$ is a simple
type).

An empty module name, denoted by {\tt []}, is used for a module of
exported names, to control the scope of names in import and export
clauses. The reserved name $(\self)$ is used to refer to the current
module, used in the type system and relations to control the scope of
names in import and export clauses.

It is not necessary to store multiple instance types for the same
variable in a typing context, neither it is necessary to use instance
types in typing contexts (they are needed only in the program theory);
only the lcg of instance types is used, because of lcg
compositionality (theorem \ref{theorem:lcg-is-compositional}). When a
new instance is declared, if it is an instance of a declared class the
type system guarantees that each member is an instance of the type
declared in the type class; otherwise (i.e.~it is the single member of
an undeclared class), its (new) type is given by the lcg of the
existing type (an existing lcg of previous instance types) and the
instance type. We define $\st(\Gamma,M,x)$ below to yield the
singleton set containing a fresh instance of the type of $x$ in
$\Gamma(M)$, if $x \in \ddom(\Gamma(M))$, where we consider
$\freshst(\sigma_x)$ yields the simple type in $\sigma_x$ with type
variables renamed to be fresh (for example,
$\freshst(\forall\,\alpha.\,\alpha)$ yields a fresh type variable
$\alpha'$):

\[ \st(\Gamma,M,x) = \left\{ \begin{array}{ll}
      \emptyset & \mathrm{if\ } x \not\in \ddom(\Gamma(M)) \\
      \{ \tau_x \} &  \mathrm{otherwise, where\ } \Gamma(M)(x) = \sigma_x, 
      \tau_x = \freshst(\sigma_x)
\end{array}\right.
\]

\subsection{Core-Haskell}
\label{sec:core-Haskell}

A declarative type system for core-Haskell is presented in Figure
\ref{fig:core-haskell-type-system}, using rules of the form $P;\Gamma
\vdash e:\delta$, which means that $e$ has type $\delta$ in typing
context $\Gamma$ and program theory $P$. 

The type system uses entailment of a set of constraints $C$ by a
program theory $P$, written as $P \entail C$. Entailment is defined in
Subsection \ref{sec:entailment}.
The type system uses also the constraint set simplification relation,
$\simplifies{P}$, which is defined as a composition of the improvement
and context reduction relations, defined respectively in
\ref{sec:improvement} and \ref{sec:context-reduction}.

Improvement is also defined in terms of constraint set entailment.  It
is simply a process of removing the subset of constraints for which
overloading is resolved and there exists a single substitution that
entails the resolved constraint. In Subsection
\ref{sec:satisfiability} we define constraint set satisfiability, the
functional counterpart of the entailment relation.

%For simplicity and following common practice, kinds are not considered
%in type expressions and type expressions which are not simple types
%are not explicitly distinguished from simple types. 
%Type expression variables are called simply type variables. 

\begin{figure}
%\begin{mdframed}
\[ \begin{array}{cc}
      \displaystyle\frac
        {\begin{array}[t]{lll}
           \Gamma(\self)(x) = (\forall\,\overline{a}.\,C\Rightarrow \tau)\:\:\: & \:\:\:P \entail \phi\,C 
           & \:\:\:\ddom(\phi) \subseteq \overline{a}
         \end{array}}
        {P;\Gamma \vdash x: \phi (C \Rightarrow \tau)} \:(\VAR) \\ \\

	\displaystyle\frac
          {P;(\Gamma(\self),x \mapsto \tau) \vdash e: C \Rightarrow \tau'}
	  {P;\Gamma \vdash \lambda x.\:e: C\Rightarrow \tau \rightarrow \tau'} \:(\ABS)  \\ \\

	\displaystyle\frac
	  {\begin{array}[t]{cc}
             P;\Gamma \vdash e: C \Rightarrow \tau' \rightarrow \tau\: &\:
             P;\Gamma \vdash e': C' \Rightarrow \tau' \\
             V = \tv(\tau) \cup \tv(C) & (C \oplus_V C') \simplifies{P} C''
        \end{array}}
	{P;\Gamma \vdash e\:e': C'' \Rightarrow \tau} \:(\APP) \\ \\

	\displaystyle\frac
	 {\begin{array}{ll}
            P;\Gamma \vdash e\!:C \Rightarrow \tau & C \simplifies{P} C'' \\
            \gen(C'' \Rightarrow \tau,\sigma,\tv(\Gamma))\: & \:P;(\Gamma(\self),x \mapsto\sigma) \vdash e'\!:\,C' \Rightarrow \tau'
          \end{array}}
	 {P;\Gamma \vdash \mlet\ x=e\ \iin\ e': C' \Rightarrow\tau' } \:(\LET)
\end{array} \]
%\end{mdframed} \vspace{-.2cm}
\caption{Core-Haskell Type System} 
\label{fig:core-haskell-type-system}
\end{figure}

Rules (\VAR) and (\ABS) are standard. Rule (\VAR) enables constrained
types to be derived for a variable, by instantiation of possibly
polymorphic constrained types, requiring that instantation yields
entailed constraints in the program theory.

Rule (\LET) performs constraint set simplification before type
generalization.

We define that $\gen(\delta,\sigma,V)$ holds if
$\sigma=\forall\,\overline{a}.\,\delta$, where
$\overline{a}=\tv(\delta) - V$; similarly, for constraints,
$\gen(C\Rightarrow\pi,\theta,V)$ is defined to hold if
$\theta=\forall\,\overline{a}.\,C\Rightarrow\pi$, where
$\overline{a}=\tv(C\Rightarrow\pi) - V$.

$C \oplus_V C'$ denotes the constraint set obtained by adding to $C$
constraints from $C'$ that have type variables reachable from $V$:
  \[ C \oplus_V C' = C \cup \{ \pi \in C'\,\mid\, \tv(\pi) \cap \reachableVars(C',V) \not= \emptyset \} \]

A constraint set $C'$ can be removed from a constrained type $C,C'
\Rightarrow \tau$ if and only if overloading for $C'$ has been
resolved and there exists a single satisfying substitution for
$C'$. 

In rule (\APP), the use of $\simplifies{P}$ allows constraints on the
type of the result to be those that occur in the function type plus
those that have variables reachable from the set of variables that
occur in the simple type of the result or in the constraint set on the
function type (cf.~Definition \ref{def:reachable}).  This allows, for
example, to eliminate constraints on the type of the following
expressions, where $o$ is any expression, with a possibly non-empty
set of constraints on its type: {\tt \flip\ \const\ $o$} (where
\const\ has type $\forall a, b.\,a \rightarrow b \rightarrow a$ and
\flip\ has type $\forall a, b, c.\,(a \rightarrow b \rightarrow c)
\rightarrow b \rightarrow a\rightarrow c$), which should denote an
identity function, and \fst\ ($e$, $o$), which should have the same
denotation as $e$.

We have the following:

\begin{Theorem}[Substituition]
  For all $P;\Gamma \vdash e: C\Rightarrow \tau$ and all substitutions
  $\phi$ such that $P \entail \phi C$ holds, we have that
  $P;\phi\Gamma \vdash e: \phi(C\Rightarrow \tau)$ holds.
 
\label{thm:substitution}
\end{Theorem}

{\em Proof\/}: By a straightforward induction on the structure of $e$. $\qed$.

The overloading resolution theorem below considers a type system that
differs from the mini-Haskell type system only by disregarding
improvement, and thus not removing any resolved constraint. We use
$\vdash_1$ instead of $\vdash$ in typing formulas of this
(mini-Haskell without improvement) type system. We also define a
program context $\CO[e]$ as any expression that has $e$ as a
subexpression.

\begin{Lemma}
  \label{lemma:unreachable-means-all-unreachable}
  For all $\pi, V$, we have that $\unreachableVars(\{\pi\},V)\not=\emptyset$
  if and only if $\unreachableVars(\{\pi\},V)=\tv(\pi)$.
\end{Lemma}

{\em Proof\/}: Directly from Definition \ref{def:reachable}. $\qed$

\begin{Theorem}[Overloading Resolution]
  For all $P, \Gamma, e, C, \tau$ such that $P;\Gamma \vdash_1 e:
  C\Rightarrow \tau$ holds, $\pi\in C$ and
  $a\in\unreachableVars(\{\pi\},\tv(\phi(\tau)))$, we have that, for
  all program contexts $\CO[e]$, if $P;\phi\Gamma \vdash_1 \CO[e]:
  (C'\Rightarrow \tau')$ holds then $\pi \in C'$ and $a\in
  \unreachableVars(\{\pi\},\tv(\tau'))$.
\label{thm:overloading-resolution}
\end{Theorem}

{\em Proof\/}: By induction on the structure of $\CO[e]$, using Lemma
\ref{lemma:unreachable-means-all-unreachable}. $\qed$

Informally speaking, theorem \ref{thm:overloading-resolution} shows
that there is no program context where an expression can be used that
will cause an unreachable type variable to be instantiated, neither a
constraint with an unreachable type variable to be changed.

\subsection{Mini-Haskell}
\label{sec:mini-Haskell}

Mini-Haskell extends core-Haskell with declarations of modules (Figure
\ref{fig:mini-haskell-module-rule}), import clauses for instances
(Figure \ref{fig:import-relation}) and declarations of classes,
instances and non-overloaded names (Figure
\ref{fig:mini-haskell-rules-for-declarations}).

Rule (\MODULE), in Figure \ref{fig:mini-haskell-module-rule}, uses
relations ($\vdash_{\Downarrow}$) and ($\vdash_{\Uparrow}^X$), which
are defined separately, for clarity, in Figures
\ref{fig:import-relation} and
\ref{fig:mini-haskell-rules-for-declarations}).

The import relation $\Gamma \vdash_{\Downarrow} \overline{\importC} :
\Gamma'$ yields a typing context ($\Gamma'$) from a typing context
($\Gamma$) and a sequence of import clauses ($\overline{\importC}$).
It inserts in the scope of the importing module pairs of variable
names and their types, that occurr in module {\tt []}, the module of
exported names. 

Relation $P;\Gamma \vdash_{\Uparrow}^X \overline{\!D}:(P',\Gamma')$ is
used for specifying the types of a sequence of bindings, from a typing
context ($\Gamma$) and a program theory ($P$); it yields a new typing
context ($\Gamma'$), so that $\Gamma'(\mbox{\tt{[]}})$ contains the
types of exported names, and a new program theory ($P'$), updated from
class and instance declarations. Relation $(\vdash)$ is used to
check that expressions of core-Haskell that occur in declarations are
well-typed.

There must exist a sequence of derivations for typing a sequence of
modules that composes a program that starts from an empty typing
context, or from a typing context with variables of predefined library
modules with their types. Recursive modules are not treated in this
paper.

\begin{figure}[b]
\[ \begin{array}{cc}
	\displaystyle\frac
	 {\begin{array}{ll}
           \Gamma_0 \vdash_{\Downarrow} \overline{\!I} : \Gamma\:\: & \:\:P;\Gamma \vdash_{\Uparrow}^X \overline{\!D} : (P',\Gamma') 
          \end{array}}
	 {P;\Gamma_0 \vdash \module\ M\, (\exportC)\ \where\ \overline{\!I};\, \overline{\!D} : (P',\Gamma')} \:(\MODULE)
\end{array} \]
\caption{Mini-Haskell module rule} 
\label{fig:mini-haskell-module-rule}
\end{figure}

The first and second rules in
Figure~\ref{fig:mini-haskell-rules-for-declarations} specify the
bindings generated by standard Haskell type classes and instance
declarations, respectively.  For simplicity, we omit special rules for
validity of type class and instance declarations (see \cite{GHC}),
that are not relevant here (for example, that the class hierachy is
acyclic).

\begin{figure}
\[ \begin{array}{cc}
	\displaystyle\frac
	 {\begin{array}{ll}
            \!\!\Gamma'(M)(x)\! = \!\left\{ \begin{array}{ll}
              \Gamma(\mbox{\tt{[]}})(x) & \mathrm{ if } M = \self\ \mathrm{ and, for\ some\ } 1 \leq k \leq n,\\
              & x\! = \! \iota_k
                    \,\mathrm{or}\, (\iota_k\! =\! \instance\ \!A\, \overline{\tau},\, x \mathrm{\ member\ of\ class\ } A)\\
               \!\!\Gamma(M)(x)          & \mathrm{ otherwise}
             \end{array}\right.
          \end{array}}
	 {\Gamma \vdash_{\Downarrow} \import\ M\, (\,\overline{\iitem}^{\,n}\,) : \Gamma'}  \\ \\
	\displaystyle\frac
	 {\begin{array}{ll}
	   \Gamma_0 \vdash_{\Downarrow} \import\ M\, (\,\overline{\iitem}\,) : \Gamma \:\:\: & \:\:\: 
           \Gamma \vdash_{\Downarrow} \overline{\importC} : \Gamma'
          \end{array}}
	 {\Gamma_0 \vdash_{\Downarrow} \import\ M\, (\,\overline{\iitem}\,); \overline{\importC} : \Gamma'} 
\end{array} \]
\caption{Import relation}
\label{fig:import-relation}
\end{figure} 

\begin{figure}
%\begin{mdframed}
\[ \begin{array}{cc}
	\displaystyle\frac
	 {\begin{array}{ll}
            Q;\Gamma \vdash_{\Uparrow}^X \overline{\!D} : (Q', \Gamma') \:\: & \:\:
            Q = P \cup \left\{ \begin{array}{ll}
                                  \{ C \Rightarrow A\:\overline{a} \} & {\rm if\ } C \not= \emptyset \\
                                  \emptyset                              & {\rm otherwise}
                                \end{array}\right. \\
            \multicolumn{2}{c}{
                \Gamma(M)(x) = \left\{ \begin{array}{ll}
                                 \delta_k       & {\rm if\ } x = x_k, 1 \leq k \leq n, {\rm and\ } 
                                                              M \in \{ \self, \mbox{\tt{[]}} \}\\
                                 \Gamma_0(M)(x) & {\rm otherwise} 
                               \end{array}\right. }
          \end{array}}
	 {P;\Gamma_0 \vdash_{\Uparrow}^X \class\ C \Rightarrow A\: \overline{a}\ \where\ \overline{x:\delta}^{\,n};\: \overline{\!D} : 
            (Q', \Gamma') } \\ \\
	\displaystyle\frac
	 {\begin{array}{l}
               P \entail \phi(C \Rightarrow \pi) \:\:\:
               \gen(\phi(C\Rightarrow \pi), \theta, \tv(\Gamma)) \:\:\:
               Q = P \cup \{ \theta \} \\
               Q;\Gamma \vdash e_i : \delta_i \:\:\:\:\: 
               \delta_i = \phi (\Gamma(\colchetes)(x_i))), \mathrm{\ for\ } i=1,\ldots, n\\
               Q;\Gamma \vdash_{\Uparrow}^X \overline{\!D} : (Q',\Gamma')
          \end{array}}
	 {P;\Gamma \vdash_{\Uparrow}^X \instance\ \phi(C \Rightarrow \pi)\ \where\ \overline{x = e}^{\,n};\: \overline{\!D}
             : (Q',\Gamma')}\\ \\
	\displaystyle\frac
	 {\begin{array}{l}
	    A \mathrm{\ is\ the\ class\ name\ generated\ for\ } x \\[.1cm]
            P; \Gamma_0 \vdash e : C \Rightarrow \tau \:\:\:\:\:
            \gen(C \Rightarrow A \:\tau, \theta, \tv(\Gamma_0)) \:\:\:\:\:
            Q = P \cup \{ \theta \} \\[.1cm] 
            Q;\Gamma \vdash_{\Uparrow}^X \overline{\!D} : (E,Q',\Gamma') \:\:\:\:\:
            \lcgR(\{ \tau\} \cup \st(\Gamma_0, \self, x),\tau')  \\[.1cm]
           \Gamma(M)(y) = \left\{ \begin{array}{ll}
                                      A\: \tau' \Rightarrow \tau' & \mathrm{ if\ } y = x, (M = \self {\rm\ or\ }
                                                                 (M = \mbox{\tt{[]}}, x \in X)) \\
                                      \Gamma_0(M)(y) & \mathrm{ otherwise } 
                                   \end{array}\right.
          \end{array}}
	 {P;\Gamma_0 \vdash_{\Uparrow}^X \instance\ x = e;\: \overline{\!D} : (Q',\Gamma')} \\\ \\

	\displaystyle\frac
	 {\begin{array}{l}
             P;\Gamma_0 \vdash e : C\Rightarrow\tau \:\: C \simplifies{P} C' \:\:
             \gen(C'\Rightarrow\tau,\sigma,\tv(\Gamma_0)) \:\:
           P;\Gamma \vdash_{\Uparrow}^X \overline{\!D} : (P',\Gamma')\\[.1cm]
           \Gamma(M)(y) = \left\{ \begin{array}{ll}
                                    \sigma & \mathrm{if\ } y = x, (M = \self\ \mathrm{ or\ }
                                                                 (M = \mbox{\tt{[]}}, x \in X)) \\
                                      \Gamma_0(M)(y) & \mathrm{ otherwise } 
                                   \end{array}\right.
          \end{array}}
	 {P;\Gamma_0 \vdash_{\Uparrow}^X x = e;\: \overline{\!D} : (P',\Gamma')} 
\end{array} \]
%\end{mdframed} \vspace{-.3cm}
\caption{Mini-Haskell rules for declarations}
\label{fig:mini-haskell-rules-for-declarations}
\end{figure}

The third rule accounts for instance declarations of an overloaded
symbol $x$ whose type is not explicitly specified in a type class. As
stated previously, the type $\tau'$ of $x$ is the least common
generalization of the set of types $\{\tau\} \cup
\{\Gamma_0(\self)(x)\}$, where $\tau$ is the type of the expression in
the current instance declaration for $x$ and $\Gamma_0(\self)(x)$ is
the type of $x$ in the current type environment (previously computed
from other instance declarations for $x$ that are visible in
$\Gamma_0$). This rule is based on Theorem
\ref{theorem:lcg-is-compositional}.

The automatically generated class name for an overloaded name can be
any unique class name, i.e.~any (of course preferably mnemonic) name
that does not occur as another declared class name. In this paper, we
use the overloaded name also as the automatically generated class
name.

Also, we define that automatically generated classes have a single
parameter, which is the type of the overloaded name. An alternative,
more similar to what happens for normal, non-automatically generated
classes, would be to define as parameters of an automatically
generated type class the sequence of type variables that occur in the
overloaded name's type. However, this alternative would imply choosing
an order to write these type variables, which would be necessary for
instantiating types of instance definitions and types in class
constraints. Instead, we just specify and instantiate the overloaded
name's type. 

The fourth rule is similar to rule (\LET); it defines how the typing
context is updated upon a declaration of a non-overloaded name.

\subsection{Use of type system rules}
\label{sec:use-of-rules}

This section presents examples that illustrate the use of Mini-Haskell
rules for declarations.

Consider the example of Figure \ref{fig:fst-ex}, where $e_1$ is
assumed to be an expression of type {\tt \Char\ $\rightarrow$ \Int}
and $e_2$ an expression of type {\tt \Int\ $\rightarrow$ \Bool}.

\begin{figure}
  \progfig{
      \module\ $A$ (\=\instance\ $f$ (\Char\ $\rightarrow$ \Int), \+\\
      \instance\ $f$ (\Int\ $\rightarrow$ \Bool), \\
      $g$ -- :: $(f\: (a \rightarrow b)) \Rightarrow a\rightarrow b$\\
      $\!\!$) \where\ \-\\ 
      xx\+\=\kill                      
      $f$\,::\,\Char\ $\rightarrow$ \Int\\
      \instance\ $f$ = $e_1$ \\ 
      $f$\,::\,\Int\ $\rightarrow$ \Bool\\
      \instance\ $f$ = $e_2$\\ 
      $g$ = $f$
  }
  \label{fig:fst-ex}
  \caption{Example for illustrating Mini-Haskell rules for declarations}
\end{figure}

Let $\Gamma_0$ be the typing context with the (possibly empty) set of
used variables of predefined library modules, with their types, and
$P_0$ be the program theory with the set of constraints that
corresponds to the (possibly empty) set of used prelude instance
definitions. Let also $\overline{D}$ be the body of \module\ $A$ and:
  \[ X = \{ \begin{array}[t]{l}
              \instance\ f\: (\Char\: \rightarrow \Int), 
              \instance\ f\: (\Int\: \rightarrow \Bool),\, g\: \}
  \end{array}
  \]  

We have:

\begin{prooftree}
  \AxiomC{$P_0;\Gamma_0 \vdash_{\Uparrow}^X \overline{\!D} : (P,\Gamma)$}
  \RightLabel{(\MODULE)}
  \UnaryInfC{$P_0;\Gamma_0 \vdash \module\ A\, (X)\ \where\ \overline{\!D} : (P,\Gamma)$}
%  \DisplayProof
\end{prooftree}
where, letting $\theta_1 = f \:(\Char \rightarrow \Int)$ and
$\overline{\!D_1}$ be the sequence of declarations {\tt \instance\ $f$
  = $e_2$; $g$ = $f$} (a rewritten version of last three lines in
Figure \ref{fig:fst-ex}):


\begin{prooftree}
  \AxiomC{
    $\begin{array}{l}
            P_0; \Gamma_0 \vdash e_1 : \Char \rightarrow \Int \:\:\:\:\:
            \gen(\theta_1, \theta_1, \tv(\Gamma_0)) \:\:\:\:\:
            P_1 = P_0 \cup \{ \theta_1 \} \\[.1cm] 
            P_1;\Gamma_1 \vdash_{\Uparrow}^X \overline{\!D_1} : (P,\Gamma) \:\:\:\:\:
            \lcgR(\{ \Char \rightarrow \Int \} \}, \Char \rightarrow \Int)  \\[.1cm]
           \Gamma_1(M)(y) = \left\{ \begin{array}{ll}
                                      \Char \rightarrow \Int & \mathrm{ if\ } y = f, (M = A {\rm\ or\ } M = \mbox{\tt{[]}}) \\
                                      \Gamma_0(M)(y) & \mathrm{ otherwise } 
                                   \end{array}\right.
    \end{array}$
  }
  \UnaryInfC{$P_0;\Gamma_0 \vdash_{\Uparrow}^X  \instance\ f \mathtt{ = } e_1;\: \overline{\!D_1} : (P,\Gamma)$}
\end{prooftree}

The derivation of $P_1;\Gamma_1 \vdash_{\Uparrow}^X \overline{\!D_1} : (P,\Gamma)$
is shown below, letting $\theta_2 = f\:(\Int\!\rightarrow\!\Bool)$
and using $\lcgR(\{\Int\!\rightarrow\!\Bool, \Char\!\rightarrow\!\Int \}, a\!\rightarrow\!b)$,
where $a$ and $b$ are fresh type variables: 

\begin{prooftree}
  \AxiomC{
    $\begin{array}{l}
            P_1; \Gamma_1 \vdash e_2 : \Int \rightarrow \Bool \:\:\:\:\:\:
            \gen(\theta_2, \theta_2, \tv(\Gamma_1)) \:\:\:\:\:\:
            P = P_1 \cup \{ \theta_2 \} \\[.1cm] 
            P;\Gamma_2 \vdash_{\Uparrow}^X g\: {\mathtt{ = }}\, f : (P,\Gamma) \:\:\:\:
        \lcgR(\{ \Int\!\rightarrow\!\Bool, \Char\!\rightarrow\!\Int \}, a\!\rightarrow\!b) \mathrm{\ where\ } a,b \mathrm{\ fresh} \\[.1cm]
           \Gamma_2(M)(y) = \left\{ \begin{array}{ll}
              f\:(a \rightarrow b) \Rightarrow a \rightarrow b &
                                       \mathrm{ if\ } y = f, (M = A {\mathrm{\ or\ }} M = \mbox{\tt{[]}}) \\
              \Gamma_1(M)(y) & \mathrm{ otherwise } 
                                   \end{array}\right.
    \end{array}$
  }
  \UnaryInfC{$P_1;\Gamma_1 \vdash \overline{\!D_1} : (P,\Gamma)$}
\end{prooftree}

We have, also, where $\sigma = \forall a,b.\, f\: (a \rightarrow b)
\Rightarrow (a\rightarrow b)$:

\begin{prooftree}
  \AxiomC{
    $\begin{array}{l}
      P;\Gamma_2 \vdash f : f\:(a \rightarrow b) \Rightarrow (a\rightarrow b) \:\:\:\:
         \gen(f\:(a \rightarrow b) \Rightarrow (a\rightarrow b),\sigma,\tv(\Gamma_2)) \\[.1cm]
      \Gamma(M)(y) = \left\{ \begin{array}{ll}
        \sigma & \mathrm{if\ } y = f, (M = A \mathrm{\ or\ } M = \mbox{\tt{[]}}) \\
        \Gamma_2(M)(y) & \mathrm{ otherwise } 
        \end{array}\right.
    \end{array}$
  }
  \UnaryInfC{$P;\Gamma_2 \vdash g\: \mathtt{ = } f : (P,\Gamma)$}
\end{prooftree}


\subsection{Entailment}
\label{sec:entailment}

We define in this appendix constraint set provability, called
entailment in Haskell terminology. Entailment of a set of constraints
is defined with respect to the set of class and instance declarations
that occur in a program, a so-called program theory
(cf.~\cite{Understanding-FDs-via-CHRs}).

\begin{Definition}

A program theory $P$ is a set of axioms of first-order logic generated
from class and instance declarations occurring in the program, as
follows (where $C \Rightarrow \pi$ is considered syntactically
equivalent to $\pi$ if $C$ is empty):

\begin{itemize}

\item For each class declaration {\tt {class $C$ $\Rightarrow$ \TCC\ $\overline{a}$ where \ldots}}
the program theory contains the following formula if $C$ is not empty:
    $\forall\,\overline{a}.\,C \Rightarrow \TCC\ \overline{a}$.

\item For each instance declaration {\tt {instance $C$ $\Rightarrow$ \TCC\ $\overline{t}$ where \ldots}}
the program theory contains the following formula:
  $\forall\,\overline{a}.\,C \Rightarrow \TCC\ \overline{t}$, 
where \linebreak $\overline{a} = \tv(\overline{t}) \cup \tv(C)$;
if $C$ is empty, then the instance declaration is of the form 
  {\tt {instance \TCC\ $\overline{t}$ where \ldots}}
and the program theory contains the formula:
  $\forall\,\overline{a}.\,\TCC\:\: \overline{t}$.
\end{itemize}
\label{program-theory-def}
\end{Definition}

\vspace*{-.5\baselineskip}
The property that a set of constraints $C$ is entailed by a program
theory $P$, written as $P \entail C$, is defined in Figure
\ref{Entailment-fig}.  Following
\cite{Associated-types-with-class,Associated-type-synonyms},
entailment is obtained from quantified constraints contained in a program
theory $P$.

%  (unlike in \cite{MarkJones94a}, entailment and satisfiability rules
% do not move constraints from types to non-closed constraints in the
% program theory, nor vice-versa).

\begin{figure}
   \[ \begin{array}{r}
         \begin{array}{cr}
   		\displaystyle\frac{}{P \entail \emptyset} (\ento)\hspace*{.3cm} &
		\displaystyle\frac
                        {(\forall\,\overline{a}.\,C\Rightarrow \pi) \in P}
			{P \entail \{ (C\Rightarrow\pi)[\overline{a}\,\mapsto\,\overline{\tau}] \} }
			(\entinst)
         \end{array}\\ \\
         \begin{array}{cr}
		\displaystyle\frac
			{P \entail C \:\:\: P \entail \{C\Rightarrow\pi\} }
			{P \entail \{ \pi \}} (\entmp)
			&
		\displaystyle\frac
			{P \entail C \:\:\: P \entail D}
			{P \entail C \cup D} (\entn)
	 \end{array}
       \end{array}
   \]
\caption{Constraint Set Entailment}
\label{Entailment-fig}
\end{figure}

\begin{Definition}[Entailed instances and Entailing Substitutions]

  \normalfont
  
$\lfloor C \rfloor_P$ is the set of {\em entailed instances\/} of constraint set
$C$ with respect to program theory $P$:
\[ \lfloor C \rfloor_P = \{\,\phi(C) \,\mid\, P\, \entail \phi(C)\, \} \]
and the corresponding substitutions as {\em entailing substitutions\/}:
  \[ \entailingSubs(C,P) = \{\,\phi \,\mid\, P\, \entail \phi(C)\, \} \]

%If $\phi(C) \in \lfloor C \rfloor_P$ then $\phi$, denoted by  is called a
%entailing substitution for $C$ in $P$.

\end{Definition}

\begin{Example} {\rm
As an example, consider:
  \[ P = \{ \forall a,b.\,D\, a\,
             b\Rightarrow C\, \text{\tt{[$a$]}}\, b, D\, \text{\it Bool\/}\,
             \text{\tt {[{\it Bool\/}]}}\}\]
We have that
  $\lfloor C\:\:a\:\:a\rfloor_P$ =
  $\lfloor C\:\text{\tt{[\it Bool\/}]}\: \text{\tt{[\it Bool\/}]}\rfloor_P$.
Both constraints
  $D\:\text{\Bool\ [\Bool]} \Rightarrow C\:\text{\tt{[{\it Bool\/}]}}\: \text{\tt [{\it Bool\/}]}$
and
  $C\:\text{\tt{[{\it Bool\/}]}}\: \text{\tt [{\it Bool\/}]}$
are members of
  $\lfloor C\:\:a\:\:a\rfloor_P$ and also members of
  $\lfloor C\:\text{\tt{[\it Bool\/}]}\: \text{\tt{[\it Bool\/}]}\rfloor_P$.

A proof that $P \entail \{ C\:\text{\tt{[\it Bool\/}]}\: \text{\tt{[\it Bool\/}]} \}$
holds can be given from the entailment rules given in Figure \ref{Entailment-fig},
since this is the conclusion of rule (\entmp) with premises
  $P \entail \{ D\:\text{\it Bool\/}\: \text{\tt{[\it Bool\/}]} \}$ and
  $P \entail \{ D\:\text{\Bool\ [\Bool]} \Rightarrow C\:\text{\tt{[{\it Bool\/}]}}
                                                                               \:\text{\tt{[{\it Bool\/}]}}\}$,
and these two premises can be derived by using rule (\entinst).}

\end{Example}

Equality of constraint sets is considered modulo type variable
renaming. That is, constraint sets $C,D$ are also equal if there
exists a renaming substitution $\phi$ that can be applied to $C$ to
make $\phi\,C$ and $D$ equal.

$\phi$ is a renaming substitution if for all $a\in\dom(S)$ we have
that $\phi(a)=b$, for some type variable $b\not\in\dom{\phi}$.



\subsection{Constraint-set Simplification}

Relation $\simplifies{P}$ is a simplification relation on
cons\-traints, defined as a composition of improvement and context
reduction, defined respectively in Subsections \ref{sec:improvement}
and \ref{sec:context-reduction}.

\begin{figure}
   \[ \displaystyle \frac
        {C \improves {P} C' \hspace*{.4cm} C' \contextreduces {P} D}
        {C \simplifies {P} D}
  \]
\caption{Constraint set simplification}
\label{fig:constraint-set-simplification}
\end{figure}

\subsubsection{Improvement}
\label{sec:improvement}

Improvement removes constraints with unreachable type variables from a
constraint $C$ that occurs on a constrained type $C\Rightarrow \tau$,
based on constraint set entailment: improvement consists of removing
each constraint in $C$ that has unreachable type variables and for
which there exists a single entailing substitution. Improvement is
defined in Figure \ref{fig:constraint-set-improvement}.
If the set $\mathbb{S}$ of entailed instances of
$\unreachableVars(C,\tv(\tau))$ has more than one element, or if it is
empty, there is no improved constraint (improvement is a partial
relation).

%Improvement as defined in this paper is not a satisfiability
%preserving relation. The satisfiable instances of $C_{\tv(\tau)}^u$
%are not part of the constraint set obtained after improvement of $C$,
%if this improved constraint set exists.

%In \ref{sec:satisfiability}, page \pageref{Phi0}, we show how to
%define $\Phi_0$ and the initial program theory $P$ from the class and
%instance declarations that are visible in a program module.

\begin{figure}
   \[ \displaystyle
       \frac{
        \begin{array}{l}
           C' = \{ \pi \mid tv(\pi) \subseteq \unreachableVars(C, \tv(\tau))\} \\
           \entailingSubs(C', P) = \{ \phi \}
        \end{array} }
      {C \Rightarrow \tau \improves {P} (C - C') \Rightarrow \tau}  \]
\caption{Constraint Set Improvement}
\label{fig:constraint-set-improvement}
\end{figure}


\subsubsection{Context Reduction}
\label{sec:context-reduction}

Context reduction is a process that reduces a constraint $\pi$ into
constraint set $D$ according to a {\it matching instance\/} for $\pi$
in a program theory $P$: if there exists
$(\forall\,\overline{\alpha}.\,C\Rightarrow \pi')\in P$ such that
$\phi(\pi') = \pi$, for some $\phi$ such that $\phi(C)$ reduces to
$D$; if there is no matching instance for $\pi$ or no reduction of
$\phi(C)$ is possible, then $\pi$ reduces to a constraint set
containing only itself.

%(in Haskell-terminology, $P$ is called the context and $\pi$ the head
%of constrained type $P\Rightarrow \pi$).

As an example of a context reduction, consider an instance declaration
that introduces $\forall a.\,{\it Eq\/}\, a \Rightarrow \text{\tt {\it
    Eq\/}[$a$]}$ in program theory $P$; then {\tt {\it Eq\/}[$a$]} is
reduced to {\it Eq\/} $a$.

Context reduction can also occur due to the presence of superclass
class declarations, but we only consider the case of instance
declarations in this paper, which is the more complex process. The
treatment of reducing constraints due to the existence of superclasses
is standard; see e.g.~\cite{MarkJones94a,Hall96,Faxen2002}.

Context reduction uses $\matches$, defined as follows:
  \[ \begin{array}{l}
        \matches\bigl(\pi,(P,\Phi'), \Delta) \text{ holds if }\\
          \:\:\:\: \Delta = \left\{ (\phi(C_0), \pi_0, \Phi')\,\,
          						\begin{array}{|l}
                             		\,\,(\forall\,\overline{\alpha}.\,C_0\Rightarrow \pi_0) \in P,\\
                             		\,\,\mgm(\pi_0 = \pi,\phi),\, \Phi' = \Phi[\pi_0,\pi]
                             	\end{array}\right\}
     \end{array}
  \]
where $\mgm$ is analogous to $\mgu$ but denotes the most general
matching substitution, instead of the most general unifier.

The third parameter of $\matches$ is either empty or a singleton set,
since overlapping instances \cite{ghc-users-guide} are not considered.

Context reduction, defined in Figure~\ref{Context-reduction-fig}, uses
rules of the form $C \contextreduces {P,\Phi} D;\Phi'$, meaning that
either $C$ reduces to $D$ under program theory $P$ and least
constraint value function $\Phi$, causing $\Phi$ to be updated to
$\Phi'$, or $C \contextreduces {P,\Fail} C;\Fail$. Failure is used to
define a reduction of a constraint set to itself.

The least constraint value function is used as in the definition of
{\it sats\/} to guarantee that context reduction is a decidable
relation.

\begin{figure}

  \[ \begin{array}{c}
       \begin{array}{cc}
         \displaystyle\frac{}
                           {\emptyset \contextreduces {P,\Phi} \emptyset;\Phi} (\redo) &
         \displaystyle\frac{\{ \pi \} \contextreduces {P,\Phi} C;\Phi_1  \:
                            D \contextreduces {P,\Phi_1} D';\Phi'}
      	                   {\{ \pi \} \cup D \contextreduces {P,\Phi} C \cup D';\Phi'} (\conj)
      \end{array}\\[.8cm]

       \displaystyle\frac{\matches\bigl(\pi,(P,\Phi),\{(C,\pi',\Phi')\}\bigr)
                           \:\:\: C \contextreduces {P,\Phi'} D;\Phi''}
       	                 {\{ \pi \} \contextreduces {P,\Phi} D;\Phi''}\: (\instum)\\[.8cm]
       \displaystyle\frac{\matches\bigl(\pi,(P,\Phi),\{(C,\pi',\Phi')\}\bigr) \:\:\:
                            C \contextreduces {P,\Phi'} D;\Fail}
       	                 {\{ \pi \} \contextreduces {P,\Phi} \{ \pi \}; \Fail} \: (\stopFail) \\[.8cm]
       \begin{array}{c}
       \displaystyle\frac{\matches\bigl(\pi,(P,\Phi),\{(C,\pi',\Fail)\}\bigr)}
       	                 {\{ \pi \} \cup C \contextreduces {P,\Phi} \{ \pi \}\cup C;\Fail} \: (\stopo)
       \end{array}
    \end{array}
  \]
\caption{Context Reduction}
\label{Context-reduction-fig}
\end{figure}

An empty constraint set reduces to itself ($\redo$).  Rule ($\conj$)
specifies that constraint set simplification works, unlike constraint
set satisfiability, by performing a union of the result of
simplifying separately each constraint in the constraint set.
To see that a rule similar to ($\conj$) cannot be used in the case of
constraint set satisfiability, consider a simple example, of
satisfiability of $C = \{A\:a, B\: a\}$ in $P = \{A\:\Int,A\:
\Bool,B\: \Int,B\: \Char\}$. Satisfiability of $C$ yields a single
substitution where $a$ maps to $\Int$, not the union of computing
satisfiability for $A\:a$ and $B\:a$ separately.

Rule ($\instum$) specifies that if there exists a constraint axiom
$\forall\,\overline{\alpha}.\,C \Rightarrow A\,\overline{\tau}$, such
that $A\,\overline{\tau}$ matches with an input constraint $\pi$, then
$\pi$ reduces to any constraint set $D$ that $C$ reduces to.

Rules ($\stopFail$) and ($\stopo$) deal with failure due to updating
of the constraint-head-value function.



%We have the following:
%
%\begin{Lemma}[Type of overloaded variable]
%If the type of an overloaded variable is not explicitly annotated in a
%type class declaration, then the variable's type is the
%anti-unification of instance types that are in scope in the current
%module; otherwise, it is the annotated type.
%\label{overloaded-variable-type}
%\end{Lemma}
%
%{\em Proof\/}: Directly from rule \ref(\VAR) in Figure
%\ref{fig:core-haskell-type-system}, page
%\pageref{fig:core-haskell-type-system}, in case the type of the
%overloaded variable is explicitly annotated in a type class, and
%otherwise from the third rule in Figure
%\ref{fig:mini-haskell-rules-for-declarations}, page
%\pageref{fig:mini-haskell-rules-for-declarations}.


\section{Records with overloaded fields}
\label{sec:overloaded-record-fields}



% \input{type-directed-named-resolution}

\section{Related Work}
\label}{sec:related-work}


\section{Conclusion}
\label{sec:conclusion}

This paper has presented an approach for allowing programmers to
overload symbols without declaring their types in type classes. In
this approach, the type of an overloaded symbol is automatically
determined from the anti-unification of instance types defined for the
symbol in the relevant module.

The paper explores this approach in the presence of instance
modularization and an ambiguity rule that is defined differently than
in Haskell.

The approach allows, for example, overloaded record fields and type
directed name resolution to be supported in a simple way.




\bibliographystyle{plain}
\bibliography{main}

\end{document}
