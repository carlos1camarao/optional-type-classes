\documentclass{article}

\author{\ }
\title{Optional Type Classes}
\date{\today}

\begin{document}

\maketitle

\begin{abstract}

This paper explores an approach for allowing type classes to be
optionally declared by programmers, i.e. for allowing programmers to
overload symbols without having to declare the types of these symbols
in type classes.

The idea is based on defining the type of un-anottated overloaded
symbol as the anti-unification of instance types defined for the
symbol in a module, by automatically creating a type class with a
single overloaded name. This depends on a modularization of instance
visibility (as well as on a redefintion of Haskell's ambiguity rule).
The paper presents the modifications to Haskell's module system that
are necessary for allowing instances to have a modular scope (based on
previous work published by one of the authors). The definition of the
type of overloaded symbols as the anti-unification of available
instance types and the redefined ambiguity rule is also based on
previous works by the authors. 

The added flexibility to Haskell-style of overloading is illustrated
by defining a type system and a type inference algorithm that allows
overloaded record fields. 

\end{abstract}

\section{Introduction}
\label{sec:intro}

The versions of Haskell supported by GHC \cite{GHC} --- the prevailing
Haskell compiler --- are becoming complex, to the point of affecting
the view of Haskell as the best choice for general-purpose software
development. A basic issue in this regard is the need of extending the
language to allow multiple parameter type classes (MPTCs). This
extension is thought to require additional mechanisms, such as
functional dependencies or type families. In another paper
\cite{JBCS-Ambiguity-and-constrained-polymorphism}, we have shown that
the introduction of MPTCs in the language can be done without the need
of additional mechanisms: a simplifying change is sufficient, to
Haskell's ambiguity rule. Interested readers are referred to
\cite{JBCS-Ambiguity-and-constrained-polymorphism}. The main ideas are
summarized below.

Haskell with MPTCs uses constrained types of the form $\forall
\overline{a}.\,C \Rightarrow \tau$, where $C$ is a set of constraints
and $\tau$ is a simple (unconstrained, unquantified) type; a
constraint is a class name followed by a sequence of type variables.

In (GHC) Haskell, ambiguity is a property of a type: a type $\forall
\overline{a}.\,C \Rightarrow \tau$ is ambiguous if there exists a type
variable that occurs in the constraints ($C$) that is not uniquely
determined from the set of type variables that occur in the simple
type ($\tau$). This unique determination specifies that, for each type
variable $a$ that occurs in $C$ but not in $\tau$ there must exist a
functional dependency $b \mapsto c$ for some $b$ in $\tau$ (or a
similar unique determination specified via type families). Notation $b
\mapsto c$ is used, instead of $b \rightarrow c$, to avoid confusion
with the notation used to denote functional types.

A new definition, which we prefer to call here {\em expression
  ambiguity\/} (in \cite{JBCS-Ambiguity-and-constrained-polymorphism}
it is called {\em delayed closure ambiguity\/}), uses a similar
property, of variable reachability, that is independent of functional
dependencies and type families: a type variable $a$ that occurs in a
set of constraints is reachable from the set of type variables in
$\tau$ if it occurs in $\tau$ or there exists a type variable $b$ in
$C$ that is reachable. For example, in $C \Rightarrow b$, where
$C=(D\: a\: b, E\: a)$, type variable $a$ is reachable from the set of
type variables in $b$.

The presence of unreachable variables in a constraint $\pi\in C$
characterizes overloading resolution, or, in other words, that
overloading is resolved for $\pi$: it characterizes that there is no
context in which an expression with such a type could be placed that
could instantiate such unreachable variables. The presence of
unreachable variables does not necessarily imply ambiguity. Ambiguity
is a property of an expression, and it depends on the context in which
the expression occurs, and on entailment of the constraints on the
expression's type.

Entailment of constraints and its algorithmic (functional) counterpart
are well-known in the Haskell world (see
e.g.~\cite{MarkJones94a,TheoryOfOverloading,JBCS-Ambiguity-and-constrained-polymorphism}).

Informally, a set of constraints $C$ is entailed (or satisfied) in a
program $P$ if there exists a substitution $\phi$ such that $\phi(C)$
is contained in the set of instance declarations of $P$, or is implied
by the transitivity implied by the set of class and instance
declarations occuring in $P$. For a formal definition, see
e.g.~\cite{MarkJones94a,JBCS-Ambiguity-and-constrained-polymorphism}. In
this case we say that $C$ is entailed by $\phi(C)$. 

For example, {\tt \Eq\ [[\Integer]]} is entailed if we have instances
{\tt \Eq\ \Integer} and {\tt \Eq\ $a$ => \Eq\ [$a$]}, visible in the
context where an expression whose type has a constraint {\tt
  \Eq\ [[\Integer]]} occurs.

If overloading is resolved for a constraint $C$ occurring in a type
$\sigma = C,D \Rightarrow \tau$ then exactly one of the following
holds:
\begin{itemize}

\item $C$ is entailed by a single instance; in this case a type
  simplification (also called ``improvement'') occurs: $\sigma$ can be
  simplified to $D \Rightarrow \tau$;

\item $C$ is entailed by more than instance; in this case we have a
  type error: ambiguity; 

\item $C$ is not entailed (by any instance); in this case we have also
  a type error: unsatisfiability.

\end{itemize}

Note that variables in a single constraint are either all reachable or
all unreachable. If they are unreachable, either the constraint can be
removed, in the case of single entailment, or there is a type error
(either ambiguity, in the case of two or more entailments, or
unsatisfiability, in the case of no entaiment).

Instead of being dependent on the specification or not of functional
dependencies or type families, ambiguity depends on the existence of
(two or more) instances in a program context when overloading is
resolved for a constraint on the type of an expression.

The possibility of a modular control of the visibility of instance
definitions conforms to this simplifying change. This is the subject
of Section \ref{sec:modular-instances}.

Also in conformance with this change is the possibility, explored in
this paper, of allowing type classes to be optionally declared by
programmers, i.e. for allowing programmers to overload symbols without
having to declare the types of these symbols in type classes. 

A type system and a type inference algorithm for a core-Haskell
language where type classes can be optionally declared is presented in
Section \ref{Optional-type-classes}.  The idea is based on defining
the type of unanottated overloaded symbol as the anti-unification of
instance types defined for the symbol in a module, by automatically
creating a type class with a single overloaded name. This depends on a
modularization of instance visibility (as well as on a redefinition of
Haskell's ambiguity rule).

The paper presents the modifications to Haskell's module system that
are necessary to allow instances to have a modular scope, based on
previous work published by one of the authors. The definition of the
type of overloaded symbols as the anti-unification of available
instance types and the redefined ambiguity rule is also based on
previous works by the authors.

The added flexibility to Haskell-style of overloading is illustrated
by defining a type system and a type inference algorithm that allows
overloaded record fields (Section \ref{sec:overloaded-record-fields}).

% The redefinition of Haskell's ambiguity rule can also be used to
% address some of the issues related to type directed name resolution
% (Section \ref{sec:type-directed-name-resolution}).


\section{Modularization of Instances}
\label{sec:modular-instances}

This paper does not attempt to discuss any major revision to Haskell's
module system. We summarize in subsection
\ref{subsec:instance-visibility-control} the work, presented in
\cite{Controlling-scope-instances}, that allows a modular control of
the visibility of instance definitions. This has the additional
benefit of enabling type classes to be optionally declared by
programmers, by the introduction of a single additional rule (to
account for the possibility of type classes to be declared or not):

\begin{definition}[Type of overloaded variable]

If the type of an overloaded variable --- i.e.~a variable that is
introduced in an instance definition --- is not explicitly annotated
in a type class declaration, then the variable's type is the
anti-unification of instance types defined for the variable in the
current module; otherwise, it is the annotated type.

\label{overloaded-variable-type}
\end{definition}

Instance modularization and the rule of expression ambiguity, that
considers the context where an expression occurs to detect whether an
expression is ambiguous or not, has profound consequences. Consider,
for example:

\proga{xx\=\kill
\module\ $A$ where\+\\
  \class\ \SShow\ $t$ \ldots\\
  \class\ \RRead\ $t$ \ldots\\
  \instance\ \SShow\ \Int\ \ldots\\
  \instance\ \RRead\ \Int\ \ldots\\
  $f$ = \sshow $\:$.$\:$\rread\-\\ \\

\module\ $B$ \where\+\\
  \import\ $A$\\
  \instance\ \RRead\ \Bool\ \ldots\\
  \instance\ \SShow\ \Bool\ \ldots\\
  $g$ = $f$ "123"
}

The definition of $f$ in module $A$ is well-typed, because constraints
(\SShow\ $a$, \RRead\ $a$) can be removed; this occurs because there
exists a single instance, in module $A$, for each constraint, that
entails it. As a result, $f$ has type \String $\rightarrow$
\String. Its use in module $B$ is (then) also well-typed. That means:
$f$'s semantics is a function that receives a value of type
\String\ and returns a value of type \String, according to the
definition of $f$ given in module $A$. The semantics of an expression
involves passing a (dictionary) value that is given in the context of
usage if, {\em and only if}, the expression has a constrained type.

\subsection{Instance visibility control: a summary}
\label{subsec:instance-visibility-control}

Modularization of instance definitions can be allowed by means of the
importation and exportion of instances as shown in
\cite{Controlling-scope-instances}. Essentially, import and export
clauses can specify, instead of just names, occurrences of {\tt
  instance $A$ $\overline{\tau}$}, where $\overline{\tau}$ is a
(non-empty) sequence of types and $A$ is a class name.  We have:

  \[ \text{\module\ $M$ (\instance\ $A$ $\overline{\tau}$, \ldots) \where\ \ldots} \]
specifies that the instance of $\overline{\tau}$ for class $D$ is
exported in module $M$.

  \[ \text{\import\ $M$ (\instance\ $A$ $\overline{\tau}$, \ldots)} \]
specifies that the instance of $\overline{\tau}$ for class $A$ is
imported from $M$, in the module where the import clause occurs.

Alternatively, we can simply give a name to an instance, in an
instance declaration, and use that name in import and export clauses
(see \cite{Controlling-scope-instances}). However, in this paper we
don't need to give a name to an instance, since we only consider
instances of undeclared classes, which have a single member, and we
can thus use the name of the member as the instance name. 

%For example, 
%we can have:
%  \progb{
%   \instance\ $x$ = '1';\\
%   \instance\ $x$ = \True;
%  }    

\subsection{Pros and Cons of Instance Modularization}

Among the advantages of this simple change, we cite (following
\cite{Controlling-scope-instances}):

\begin{itemize}

  \item programmers have better control of which entities are
    necessary and should be in the scope of each module in a program;

  \item it is possible to define and use more than one instance for
    the same type in a program;

  \item problems with orphan instances (i.e.~instances defined in a
    module where neither the definition of the data type nor the
    definition of the type class) do not occur (for example, distinct
    instances of \Either\ for class \Monad, say one from package
    \mtl\ and another from \transformers, can be used in a program);

  \item the introduction of newtypes, as well as the use of functions
    that include additional (-by) parameters, such as e.g.~the (first)
    parameter of function \sortBy\ in module \Data.\List\ can be
    avoided.

\end{itemize}

With instance modularization, programmers need to be aware of which
entities are exported and imported (i.e.~which entities are visible in
the scope of a module) and their types, in particular if they are
overloaded or not.  A simple change like a type annotation for a
variable exported from a module, can lead to a change in the semantics
of using this variable in another module.

% The change is very significant if the type annotation instantiates
% the type of $x$ so that overloading of $x$ is resolved in $A$, since
% this leads to a change in the type (and meaning) of $x$ in module
% $B$.







\section{Ambiguity Rule}
\label{sec:ambig}



\section{Preliminaries}\label{prelimirares}

In this section we introduce some basic definitions and notations. We
consider that meta-variables defined can appear primed or subscripted.

Meta-variable usage is defined in the paper as follows: $x,y$ denote
term variables, $\alpha, \beta$ ($a, b,...$
in examples) type variables, $e$ a term,
$\tau,\rho$ simple types, $\sigma$ a type, $\Gamma$ a typing context, 
that is, a set of pairs written as $x:\sigma$, and $S$ a
substitution. A constraint is formed by a pair of a class name $C$ and
a sequence of types $\overline{\tau}$. We slighly abuse notation and 
use $\kappa$ to denote both a single constraint and a constraint set.

The notation $\overline{a}^{\,n}$, or simply $\overline{a}$, denotes
the sequence $a_1 \cdots a_n$, or $a_1, \ldots, a_n$, or
$a_1;\ldots;a_n$, depending on the context where it is used, where
$n\geq 0$. When used in a context of a set, it denotes
$\{a_1,\ldots,a_n\}$. It can be used with more than one variable; for
example, in $\overline{x = e}^{\,n}$, it denotes the sequence $x_1 =
e_1, \ldots, x_n = e_n$.

A substitution is a function from type variables to simple type
expressions (cf.~Section \ref{Optional-type-classes}). The identity
substitution denoted by \id. $\phi(\sigma)$ (or simply $\phi\,\sigma$)
represents the capture-free operation of substituting $\phi(\alpha)$
for each free occurrence of $\alpha$ in $\sigma$.

We overload the substitution application on constraints, constraint
sets and sets of types. Definition of application on these elements is
straightforward. The symbol $\circ$ denotes function composition and
$\dom{\phi}=\{\alpha \mid\ \phi(\alpha) \neq \alpha\}$.

The notation $\phi[\overline{\alpha}\mapsto\overline{\tau}]$ denotes
the updating of $\phi$ such that $\overline{\alpha}$ maps to
$\overline{\tau}$, that is, the substitution $\phi'$ such that
$\phi'(\beta) = \tau_i$ if $\beta = \alpha_i$, for $i = 1,...,n$,
otherwise $\phi(\beta)$. Also, $[\overline{\alpha}\mapsto\overline{\tau}]
= \id[\overline{\alpha}\mapsto\overline{\tau}]$.

\subsection{Anti-unification of instance types}
\label{sec:anti-unif}

A type $\tau$ is a generalization of a set of simple types
$\overline{\tau}^{\,n}$ if there exist substitutions
$\overline{\phi}^{\,n}$ such that $\phi_i(\tau)=\tau_i$, for
$i=1,\ldots,n$. A generalization is also called a (first-order) {\em
  anti-unification\/} \cite{ModelTheory2012}.

We say that $\tau'$ is less general than $\tau$, written $\tau \leq
\tau'$, if there exist $\phi$ such that $\phi(\tau) = \tau'$.

The {\it least common generalization} (lcg) of a set of types
$\mathbb{T}$ and a type $\tau$ holds, written as
$\lcgR(\mathbb{T},\tau)$, if, for all generalizations $\tau'$ of
$\mathbb{T}$, we have $\tau \leq \tau'$.

An algorithm for computing the lcg of a finite set of types in
presented in Figure \ref{fig:lcg}. The concept was studied by Gordon
Plotkin \cite{plotkin1970note,plotkin1971further}, that defined a
function for constructing a generalization of two symbolic
expressions.  In Figure~\ref{fig:lcg}, we present function \lcg, that
gives the lcg of a finite set of simple types by recursion on the
structure of set $\mathbb{T}$, using a function to compute the
generalization of two simple types. For two types $\tau_1$ and
$\tau_2$ the idea is to recursively traverse the structure of both
types using a finite map to store previously generalized
types. Whenever we found two different type constructors, we search on
the finite map if they have been previously generalized. If this was
the case, the generalization is returned. But if these two type
constructors aren't in the finite map we insert them using a fresh
type variable as their generalization and return this new variable.

\begin{figure*}[ht]
	\[\progfig{
            $\lcg(\mathbb{T})=\tau$ $\:\:\:$ where 
               $(\tau, \phi)=\lcg'(\mathbb{T},\id)$, for some  $\phi$ \\ \\
            $\lcg'(\{\tau\},\phi) = (\tau, \phi)$  \\ \\		
            $\lcg'(\{\tau_1\} \cup \mathbb{T}, \phi) = \lcg''(\tau_1, \tau',\phi') \:\:\:$ where
		$\begin{array}[t]{ll}
                   (\tau',\phi')  & = lcg'(\mathbb{T}, \phi)
		\end{array}$  \\ \\		
            xxx\=xxx\=xxx\=xxx\=xxxxx\=xxxxxx\=xxxxxxxx\= \kill
            $\lcg'' (C \: \overline{\tau}^{\,n},\:  D\: \overline{\rho}^{\,m},\phi)=$\+\\
              \textbf{if}\ $\phi(\alpha)=( C\:\overline{\tau}^{\,n},\: D\:\overline{\rho}^{\,m})$
                      for some $\alpha$ \textbf{then}\ $(\alpha,\phi)$ \\
              \textbf{else} \+\\
              \textbf{if}\ $n\not=m$ \textbf{then}\
                 $(\beta, \phi [\beta \mapsto ( C \:\overline{\tau}^{\,n},\: D\:\overline{\rho}^{\,m})])$ \+ \\
		 where $\beta$ is a fresh type variable \-\\[.1cm]
              \textbf{else}\ $(\psi\: \overline{\tau'}^{\,n}, \phi_n)$\+\\
                 where $\begin{array}[t]{l}
		          (\psi,\phi_0) = \left\{\begin{array}{ll}
                                            (C ,\phi) & \textbf{if } C = D \\
                                            (\alpha, \phi\,[\alpha\mapsto (C, D)])
                                                      & \text{otherwise, $\alpha$ is fresh }\\
                                          \end{array}\right. \\[.3cm]
                          (\tau'_i,\phi_i) = lcg''(\tau_i, \rho_i, \phi_{i-1}), \text{ for } i=1, \ldots, n
                        \end{array}$ \-\-\-	
        }
        \]
\caption{Least Common Generalization} \label{fig:lcg}
\end{figure*}
As an example of the use of \lcg, consider the following types (of
functions \map\ on lists and trees, respectively):

\progb{
   $(a \rightarrow b)$ $\rightarrow$ [$a$] $\rightarrow$ [$b$]\\
   $(a \rightarrow b)$ $\rightarrow$ \Tree\ $a$ $\rightarrow$ \Tree\ $b$
}

A call of \lcg\ for a set with these types yields type $(a \rightarrow
b) \rightarrow c\:\: a \rightarrow c\:\: b$, where $c$ is a
generalization of type constructors {\tt []} and \Tree.

We have: 

\begin{Theorem}[Soundness of \lcg]
For all (sets of simple types) $\mathbb{T}$, we have that
$\lcg(\mathbb{T})$ yields a generalization of $\mathbb{T}$.
\label{theorem:lcg-is-sound}
\end{Theorem}

\begin{Theorem}[Completeness of \lcg]
For all (sets of simple types) $\mathbb{T}$, we have that
$\lcgR(\mathbb{T},\lcg(\mathbb{T}))$ holds, i.e.~if $\tau$ is a
generalization of $\mathbb{T}$ then $\lcg(\mathbb{T}) \leq \tau$.
\label{theorem:lcg-is-complete}
\end{Theorem}

\begin{Theorem}[Compositionality of \lcg]
For all non-empty (sets of simple types) $\mathbb{T}, \mathbb{T'}$, we
have that $\lcg(\lcg(\mathbb{T}),\lcg(\mathbb{T'})) = \lcg(\mathbb{T}
\cup \mathbb{T'})$.
\label{theorem:lcg-is-compositional}
\end{Theorem}

\begin{Theorem}[Uniqueness of \lcg]
For all (sets of simple types) $\mathbb{T}$, we have that
$\lcg(\mathbb{T})$ is unique, up to variable renaming.
\label{theorem:lcg-is-unique-modulo-variable-renaming}
\end{Theorem}



The proofs use straighforward induction on the number and complexity
of elements of $\mathbb{T}$.




\input{records-with-overloaded-fieds}

\section{Conclusion}
\label{sec:conclusion}

This paper has presented an approach for allowing programmers to
overload symbols without declaring their types in type classes. In
this approach, the type of an overloaded symbol is automatically
determined from the anti-unification of instance types defined for the
symbol in the relevant module.

The paper explores this approach in the presence of instance
modularization and an ambiguity rule that is defined differently than
in Haskell.

The approach allows, for example, overloaded record fields and type
directed name resolution to be supported in a simple way.




\bibliographystyle{plain}
\bibliography{main}

\end{document}
