\documentclass{article}

\author{\ }
\title{Optional Type Classes}
\date{\today}

\begin{document}

\maketitle

\begin{abstract}

This paper explores an approach for allowing type classes to be
optionally declared by programmers, i.e. for allowing programmers to
overload symbols without having to declare the types of these symbols
in type classes.

The idea is based on defining the type of un-anottated overloaded
symbol as the anti-unification of instance types defined for the
symbol in a module, by automatically creating a type class with a
single overloaded name. This depends on a modularization of instance
visibility (as well as on a redefintion of Haskell's ambiguity rule).
The paper presents the modifications to Haskell's module system that
are necessary for allowing instances to have a modular scope (based on
previous work published by one of the authors). The definition of the
type of overloaded symbols as the anti-unification of available
instance types and the redefined ambiguity rule is also based on
previous works by the authors. 

The added flexibility to Haskell-style of overloading is illustrated
by defining a type system and a type inference algorithm that allows
overloaded record fields. 

\end{abstract}

\section{Introduction}
\label{sec:intro}

We use the term {\em constrained polymorphism\/} to refer to the
polymorphism originated by the combination of parametric polymorphism
and context-dependent overloading.  Context-dependent overloading is
characterized by the fact that overloading resolution in expressions
(function calls) $e\: e'$ is based not only on the types of the
function ($e$) and the argument ($e'$), but also on the context in
which the expression ($e\: e'$) occurs. This makes overloading much
more expressive, since:

\begin{itemize}

  \item constants can also be overloaded --- for example, literals
    (like {\tt 1}, {\tt 2} etc.) can be used to represent fixed and
    arbitrary precision integers as well as fractional numbers, so
    that they can be used in expressions such as {\tt \mbox{1 + 2.0}}
    ---, and

  \item overloading resolution for function calls need not occur when
    the argument is encountered.  Functions with types that differ
    only on the type of the result can be overloaded, for example {\em
      read\/} functions with types {\tt \String\ $\rightarrow$ \Bool},
    {\tt \String\ $\rightarrow$ \Int}, each taking a string and
    generating the denoted value in the corresponding type ---, and
    also higher-order curried functions, e.g.~mapping functions with
    types {\tt $(a\rightarrow b)\rightarrow f\:a\rightarrow f\:b$},
    folding functions with types {\tt $(a\rightarrow b\rightarrow
      b)\rightarrow b\rightarrow f\:a\rightarrow b$}, for distinct
    instances of $f$, can be overloaded.

\end{itemize}

In this way, context-dependency allows overloading to have a much more
prominent role in the presence of parametric polymorphism, as explored
mainly in the programming language Haskell.

The main motivation of this paper is to allow symbols to be overloaded
in a language that supports constrained polymorphism without requiring
the types of these symbols to be declared separately in type classes.
We consider also the following:

\begin{description}

\item[Expression Ambiguity:] Ambiguity is defined according to an
  intuitive notion of the existence of two or more distinct instances
  of an overloaded name used when overloading is resolved. It is not
  as a syntactic property of a type, as it is defined in Haskell.
  Ambiguity is discussed in Section \ref{sec:ambiguity}. This
  modification is not necessary for optional type classes to work, but
  allows greater flexibility; together with optional type classes it
  enables, for example, type directed name resolution to be obtained
  for free (Subsection \ref{sec:type-directed-name-resolution}) and
  allows also an easy support of overloaded record fields (Subsection
  \ref{sec:overloaded-record-fields}).


\item[Modular Instance Scope:] Overloaded names with modular scope, as
  occurs with non-overloaded names. The paper presents minimalist
  modifications to Haskell's module system that are necessary for
  overloaded names to have a modular scope, whether their types are
  annotated or not in type classes. This modification is also not
  necessary for optional type classes to work. Instances with a type
  class can maintain their global scope; instances without a type
  class could as well have a global scope, though that design decision
  would be rather peculiar. In our view the benefits of instance
  modularization outweigh its possible drawbacks. This is discussed in
  Section \ref{sec:modular-instances}.

\end{description}

The mechanism of optional type classes presented in this paper is
based on the following:

\begin{enumerate}

 \item The type of an overloaded symbol is, as usual, a constrained
   type, of the form $\forall\,\overline{a}.\,C \Rightarrow \tau$,
   where $C$ is a set of constraints and $\tau$ is a simple
   (unquantified and unconstrained) type. A constraint is a class name
   followed by a sequence of simple types.

\item An overloaded symbol $x$ can be defined by instance declarations
  of the form $\instance\ x=e$, without explicitly declaring its type
  in a type class.
  
\item The type of $x$ is automatically determined from the
  anti-unification of the instance types for $x$ that are visible in a
  module, by creating a type class with a single member ($x$). The
  algorithm used for computing the type of $x$ is presented in
  Section~\ref{sec:anti-unif}.

\end{enumerate}

The proposed approach is formalized in
Section~\ref{Optional-type-classes}, where a type system for a
core-Haskell language where type classes can be optionally declared is
presented.

The added flexibility of the overloading mechanism, with respect to
Haskell, is illustrated by the enabled simple support for overloaded
record fields (Section \ref{sec:overloaded-record-fields}) and type
directed name resolution (Section
\ref{sec:type-directed-name-resolution}).

A type inference algorithm is presented in Section
\ref{sec:type-inference}, which is proved to be sound and to infer,
for a given expression in a given typing context, a type that is a
minimal generalization of types derivable in the type system.

% A semantics is presented in Section \ref{sec:semantics}. 

Related work is outlined in Section \ref{sec:related-work} and Section
\ref{sec:conclusion} concludes.

Appendices include the definition of the entailment relation
(\ref{sec:entailment}), used in the type system, the improvement and
context-reduction relations
(\ref{sec:improvement},\ref{sec:context-reduction}), whose composition
yield the constraint-set simplification relation, also used in the
type system, and the constraint-set satisfiability function
(\ref{sec:satisfiability}), used in the improvement relation and in
the type inference algorithm.

A prototype implementation of a type inference algorithm for Haskell
supporting overloading without the need of defining a type class is
available \cite{opt-rep}.

\section{Modularization of Instances}
\label{sec:modular-instances}

This paper does not attempt to discuss any major revision to Haskell's
module system. We summarize in subsection
\ref{subsec:instance-visibility-control} the work, presented in
\cite{Controlling-scope-instances}, that allows a modular control of
the visibility of instance definitions. This has the additional
benefit of enabling type classes to be optionally declared by
programmers, by the introduction of a single additional rule (to
account for the possibility of type classes to be declared or not):

\begin{definition}[Type of overloaded variable]

If the type of an overloaded variable --- i.e.~a variable that is
introduced in an instance definition --- is not explicitly annotated
in a type class declaration, then the variable's type is the
anti-unification of instance types defined for the variable in the
current module; otherwise, it is the annotated type.

\label{overloaded-variable-type}
\end{definition}

Instance modularization and the rule of expression ambiguity, that
considers the context where an expression occurs to detect whether an
expression is ambiguous or not, has profound consequences. Consider,
for example:

\proga{xx\=\kill
\module\ $A$ where\+\\
  \class\ \SShow\ $t$ \ldots\\
  \class\ \RRead\ $t$ \ldots\\
  \instance\ \SShow\ \Int\ \ldots\\
  \instance\ \RRead\ \Int\ \ldots\\
  $f$ = \sshow $\:$.$\:$\rread\-\\ \\

\module\ $B$ \where\+\\
  \import\ $A$\\
  \instance\ \RRead\ \Bool\ \ldots\\
  \instance\ \SShow\ \Bool\ \ldots\\
  $g$ = $f$ "123"
}

In our approach (i.e.~considering ambiguity as a property of an
expression, not of a type), the definition of $f$ in module $A$ is
well-typed (it is not well-typed in Haskell), because constraints
(\SShow\ $a$, \RRead\ $a$) can be removed (in Haskell, type {\tt
  (\SShow\ $a$, \RRead\ $a$) $\Rightarrow$ \String} is ambiguous); the
constraints can be removed because there exists a single instance, in
module $A$, for each constraint, that entails it. As a result, $f$ has
type \String $\rightarrow$ \String. Its use in module $B$ is (then)
also well-typed. That means: $f$'s semantics is a function that
receives a value of type \String\ and returns a value of type \String,
according to the definition of $f$ given in module $A$. The semantics
of an expression involves passing a (dictionary) value that is given
in the context of usage if, {\em and only if}, the expression has a
constrained type.

\subsection{Instance visibility control: a summary}
\label{subsec:instance-visibility-control}

Modularization of instance definitions can be allowed as shown in
\cite{Controlling-scope-instances}. Essentially, import and export
clauses can specify, instead of just names, also {\tt instance $A$
  $\overline{\tau}$}, where $\overline{\tau}$ is a (non-empty)
sequence of types and $A$ is a class name; we have that:

  \[ \text{\module\ $M$ (\instance\ $A$ $\overline{\tau}$, \ldots) \where\ \ldots} \]
specifies that the instance of $\overline{\tau}$ for class $D$ is
exported in module $M$.

  \[ \text{\import\ $M$ (\instance\ $A$ $\overline{\tau}$, \ldots)} \]
specifies that the instance of $\overline{\tau}$ for class $A$ is
imported from $M$, in the module where the import clause occurs.

Alternatively, we can simply give a name to an instance, in an
instance declaration, and use that name in import and export clauses
(see \cite{Controlling-scope-instances}). However, in this paper we
don't need to give a name to an instance, since we only consider
instances of undeclared classes, which have a single member, and we
can thus use the name of the member as the instance name. 

%For example, 
%we can have:
%  \progb{
%   \instance\ $x$ = '1';\\
%   \instance\ $x$ = \True;
%  }    

\subsection{Pros and Cons of Instance Modularization}

Among the advantages of this simple change, we cite (following
\cite{Controlling-scope-instances}):

\begin{itemize}

  \item Programmers have better control of which entities are
    necessary and should be in the scope of each module in a program.

  \item It is possible to define and use more than one instance for
    the same type in a program.

  \item Problems with orphan instances do not occur (orphan instances
    are instances defined in a module where neither the definition of
    the data type nor the definition of the type class occur). For
    example, distinct instances of \Either\ for class \Monad, say one
    from package \mtl\ and another from \transformers, can be used in
    a program.

  \item The introduction of newtypes, as well as the use of functions
    that include additional (-by) parameters, such as e.g.~the (first)
    parameter of function \sortBy\ in module \Data.\List\ can be
    avoided.

\end{itemize}

With instance modularization, programmers need to be aware of which
entities are exported and imported --- i.e.~which entities are visible
in the scope of a module --- {\em and their types}, in particular
whether they are or not overloaded. A simple change like a type
annotation for a variable exported from a module, can lead to a change
in the semantics of using this variable in another module.

% The change is very significant if the type annotation instantiates
% the type of $x$ so that overloading of $x$ is resolved in $A$, since
% this leads to a change in the type (and meaning) of $x$ in module
% $B$.







\section{Ambiguity Rule}
\label{sec:ambig}



\section{Anti-unification of instance types}
\label{sec:anti-unif}

A simple type $\tau$ is a generalization of a set of simple types
$\overline{\tau}^{\,n}$ if there exist substitutions
$\overline{\phi}^{\,n}$ such that $\phi_i(\tau)=\tau_i$, for
$i=1,\ldots,n$. For example, $a_0 \rightarrow a_0$,
$a_1 \rightarrow a_2$, and $a_3$ are generalizations of
$\{ \Int \rightarrow \Int, \Float \rightarrow \Float\}$.\footnote{A generalization is also called a (first-order) {\em
  anti-unification\/} \cite{ModelTheory2012}.}

We say that $\tau$ is less general than $\tau'$, written $\tau \leq
\tau'$, if there exist a substitution $\phi$ such that $\phi(\tau') = \tau$.  For
example, $a_0 \rightarrow a_0 \leq a_1 \rightarrow
a_2 \leq a_3$.

The {\it least common generalization} (lcg) of a set of types
$S$ and a type $\tau$ holds, written as $\lcgR(S,\tau)$, if, for all generalizations $\tau'$ of
$S$ we have $\tau \leq \tau'$.

The concept of least common generalization was studied by Gordon Plotkin \cite{plotkin1970note,plotkin1971further}, that defined a
function for constructing a generalization of two symbolic
expressions.  In Figure~\ref{fig:lcg}, we define function \lcg, which 
returns a lcg of a finite set of 
simple types $S$, by recursion on the structure of $S$,  
using function $\lcg'$ to compute the
generalization of two simple types. For two types $\tau_1$ and
$\tau_2$ the idea is to recursively traverse the structure of both
types using a finite map to store previously generalized
types. Whenever we find two different type constructors, we search on
the finite map if they have been previously generalized. If this is
the case, the previous generalization is returned. If these two type
constructors are not in the finite map, we insert them using a fresh
type variable as their generalization and return this new variable.

\begin{figure*}[ht]
	\[\progfig{
            $\lcg(S)=\tau$ $\:\:\:$ where 
               $(\tau, \phi)=\lcg'(S,\id)$, for some  $\phi$ \\ \\
            $\lcg'(\{\tau\},\phi) = (\tau, \phi)$  \\ \\		
            $\lcg'(\{\tau_1\} \cup S, \phi) = \lcg''(\tau_1, \tau',\phi') \:\:\:$ where
		$\begin{array}[t]{ll}
                   (\tau',\phi')  & = lcg'(S, \phi)
		\end{array}$  \\ \\		
            xxx\=xxx\=xxx\=xxx\=xxxxx\=xxxxxx\=xxxxxxxx\= \kill
            $\lcg'' (T \: \overline{\tau}^{\,n},\:  T'\: \overline{\rho}^{\,m},\phi)=$\+\\
              \textbf{if}\ $\phi(a)=( T\:\overline{\tau}^{\,n},\: T'\:\overline{\rho}^{\,m})$
                      for some $a$ \textbf{then}\ $(a,\phi)$ \\
              \textbf{else} \+\\
              \textbf{if}\ $n\not=m$ \textbf{then}\
                 $(b, \phi [b \mapsto ( T \:\overline{\tau}^{\,n},\: T'\:\overline{\rho}^{\,m})])$ \+ \\
		 where $b$ is a fresh type variable \-\\[.1cm]
              \textbf{else}\ $(\psi\: \overline{\tau'}^{\,n}, \phi_n)$\+\\
                 where $\begin{array}[t]{l}
		          (\psi,\phi_0) = \left\{\begin{array}{ll}
                                            (T ,\phi) & \textbf{if } T = T' \\
                                            (a, \phi\,[a\mapsto (T, T')])
                                                      & \text{otherwise, $a$ is fresh }\\
                                          \end{array}\right. \\[.3cm]
                          (\tau'_i,\phi_i) = lcg''(\tau_i, \rho_i, \phi_{i-1}), \text{ for } i=1, \ldots, n
                        \end{array}$ \-\-\-	
        }
        \] \vspace{-.2cm}
\caption{Least Common Generalization} \label{fig:lcg}
\end{figure*}
As an example of the use of \lcg, consider the following types (of
functions \map\ on lists and trees, respectively):

\progb{
   $(a \rightarrow b)$ $\rightarrow$ [$a$] $\rightarrow$ [$b$]\\
   $(a \rightarrow b)$ $\rightarrow$ \Tree\ $a$ $\rightarrow$ \Tree\ $b$
}

A call of \lcg\ for a set with these types yields type $(a \rightarrow
b) \rightarrow c\:\: a \rightarrow c\:\: b$, where $c$ is a
generalization of type constructors {\tt []} and \Tree\ (for $c$ to be
used in $c\: b$, mapping $c \mapsto (\texttt{[]},\Tree)$ is saved in
parameter $\phi$ of $\lcg''$, to be reused).

The following theorems guarantee correctness of function \lcg: 

\begin{Theorem}[Soundness of \lcg]
For all (sets of simple types) $S$, we have that
$\lcg(S)$ yields a generalization of $S$.
\label{theorem:lcg-is-sound}
\end{Theorem}

\begin{Theorem}[Completeness of \lcg]
For all (sets of simple types) $S$, we have that $\lcgR(S,\lcg(S))$
holds (i.e.~$\lcg(S)$ is a generalization of $S$) and, for any $\tau$
that is a generalization of $S$, we have that $\lcg(S) \leq \tau$.
\label{theorem:lcg-is-complete}
\end{Theorem}

\begin{Theorem}[Compositionality of \lcg]
For all non-empty (sets of simple types) $S, S'$, we have that
$\lcg(\lcg(S),\lcg(S')) = \lcg(S \cup S')$.
\label{theorem:lcg-is-compositional}
\end{Theorem}

\begin{Theorem}[Uniqueness of \lcg]
For all (sets of simple types) $S$, we have that
$\lcg(S)$ is unique, up to variable renaming.
\label{theorem:lcg-is-unique-modulo-variable-renaming}
\end{Theorem}


The proofs use straighforward induction on the number and structural
complexity of elements of $S$.




\input{records-with-overloaded-fieds}

\section{Conclusion}
\label{sec:conclusion}

This paper has presented an approach for allowing programmers to
overload symbols without declaring their types in type classes. In
this approach, the type of an overloaded symbol is automatically
determined from the anti-unification of instance types defined for the
symbol in the relevant module.

The paper explores this approach in the presence of instance
modularization and an ambiguity rule that is defined differently than
in Haskell.

The approach allows, for example, overloaded record fields and type
directed name resolution to be supported in a simple way.




\bibliographystyle{plain}
\bibliography{main}

\end{document}
