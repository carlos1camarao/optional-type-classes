\documentclass{article}

\author{\ }
\title{Optional Type Classes}
\date{\today}

\begin{document}

\maketitle

\begin{abstract}

This paper explores an approach for allowing type classes to be
optionally declared by programmers, i.e. for allowing programmers to
overload symbols without having to declare the types of these symbols
in type classes.

The idea is baseds on defining the type of un-anottated overloaded
symbol as the anti-unification of instance types defined for the
symbol in a module, by automatically creating a type class with a
single overloaded name. This depends on a modularization of instance
visibility (as well as on a redefintion of Haskell's ambiguity rule).
The paper presents the modifications to Haskell's module system that
are necessary for allowing instances to have a modular scope (based on
previous work published by one of the authors). The definition of the
type of overloaded symbols as the anti-unification of available
instance types and the redefined ambiguity rule is also based on
previous works by the authors. 

The added flexibility to Haskell-style of overloading is illustrated
by defining a type system and a type inference algorithm that allows
overloaded record fields. 

\end{abstract}

\section{Introduction}
\label{sec:intro}

The {\em version\/} of Haskell supported by GHC --- the prevailing
Haskell compiler --- is becoming complex, to the point of affecting
the view of Haskell as being the best choice for general-purpose
software development. A critical issue in this regard is the need of
extending the language to allow multiple parameter type classes
(MPTCs). Unfortunately, this extension is thought to require
additional mechanisms, such as functional dependencies or type
families. In another paper
\cite{JBCS-Ambiguity-and-constrained-polymorphism}, we have shown how
MPTCs can be introduced in the language without the need of additional
mechanisms. A simplifying change is sufficient, in Haskell's ambiguity
rule. Interested readers are referred to
\cite{JBCS-Ambiguity-and-constrained-polymorphism}, but a summary is
included below.

....


In conformance with this change, ... aqui vem texto relativo a
controle modular da visibilidade de instâncias...

Also in conformance with this change is the possibility, explored in
this paper, of allowing type classes to be optionally declared by
programmers, i.e. for allowing programmers to overload symbols without
having to declare the types of these symbols in type classes.

The idea is baseds on defining the type of un-anottated overloaded
symbol as the anti-unification of instance types defined for the
symbol in a module, by automatically creating a type class with a
single overloaded name. This depends on a modularization of instance
visibility (as well as on a redefintion of Haskell's ambiguity rule).
The paper presents the modifications to Haskell's module system that
are necessary for allowing instances to have a modular scope (based on
previous work published by one of the authors). The definition of the
type of overloaded symbols as the anti-unification of available
instance types and the redefined ambiguity rule is also based on
previous works by the authors. 

\section{Modularization of Instances}
\label{sec:modular-instances}

Let the instance type of a non-overloaded variable be the variable's
type. Unless the type of a variable is explicitly annotated, either in
a type class or in a type annotation, a variable's type is the
anti-unification of instance types defined for the variable in a
module.

Consider, for example, ...

\section{Ambiguity Rule}
\label{sec:ambig}

\section{Anti-unification of instance types}
\label{sec:anti-unif}

\section{Records with overloaded record fields}
\label{sec:overloaded-record-fields}

\section{Conclusion}
\label{sec:conclusion}

\bibliographystyle{plain}
\bibliography{main}

\end{document}
