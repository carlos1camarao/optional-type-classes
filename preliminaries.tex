\section{Preliminaries}\label{prelimirares}

This section introduces basic definitions and notations. Meta-variable
usage and the syntax of types are given in Figure~\ref{fig:meta}.  
%Meta-variables can be possibly primed or subscripted).

\begin{figure} 
\[ \begin{array}[c]{llll}
{\rm Class\ Name}         &\hspace{.1cm} & A         & \\
{\rm Type\ variable}      &         & a,b & \\
{\rm Type\ constructor}   &         & T              & \\
{\rm Simple\ Constraint}  &         & \pi            & ::= A\,\overline{\tau}\\
{\rm Set\ of\ Simple\ Constraints}  &  & C           & \\
%{\rm Unquantified Constraint} &    & \psi           & ::= C\Rightarrow \pi\\
{\rm Constraint}          &         & \theta         & ::= \forall\,\overline{a}.\,C\Rightarrow\pi\\
{\rm Simple\ Type}        &         & \tau,\rho      & ::= a \mid T \mid \tau\:\tau' \\
{\rm Constrained\ Type}   &         & \delta         & ::= C\Rightarrow \tau\\
{\rm Type}                &         & \sigma         & ::= \forall\,\overline{a}.\,\delta \\
{\rm Substitution}        &         & \phi           & \\
\end{array} \]
\caption{Syntax of Types}
\label{fig:meta}
\end{figure}

As usual, we assume the existence of type constructor $\to$, that is
written as an infix operator ($\tau \to \tau'$). A type
$\forall\,\overline{a}.\,C\Rightarrow \tau$ is equivalent to
$C\Rightarrow \tau$ if $\overline{a}$ is empty and, similarly,
$C\Rightarrow \tau$ is equivalent to $\tau$ if $C$ is empty.

Notation $\overline{x}^{\,n}$, or simply $\overline{x}$, is used
throughout this paper to denote the sequence $x_1 \cdots x_n$, or
$x_1, \ldots, x_n$, or $x_1;\ldots;x_n$, depending on the context
where it is used, where $n\geq 0$, and $x$'s can be either type
variables, or mappings, or bindings etc.  When used in a context of a
set, it denotes $\{x_1,\ldots,x_n\}$.

The set of type variables occurring in $X$ is denoted by $\tv(X)$,
where $X$ can be a type, a constraint, sets of types or constraints,
or a typing context.

A substitution $\phi$ is a function from type variables to simple type
expressions. A type variable can be a constructor variable and the
image of a substitution can be a type expression that is not a type
(for example a type constructor with an arity that is not zero). Kinds
are not considered in type expressions, and type expressions which are
not simple types are not explicitly distinguished from simple types,
but whenever necessary we distinguish the arity of type
expressions. 

The identity substitution is denoted by \id. $\phi(\sigma)$
(or simply $\phi\,\sigma$) represents the capture-free operation of
substituting $\phi(a)$ for each free occurrence of $a$ in $\sigma$.

We overload the substitution application on constraints, constraint
sets and sets of types. Definition of application on these elements is
straightforward. The symbol $\circ$ denotes function composition and
$\dom{\phi}=\{\alpha \mid\ \phi(\alpha) \neq \alpha\}$.

The notation $\phi[\overline{a\mapsto\tau}^{\,n}]$ denotes the
substitution $\phi'$ such that $\phi'(b) = \tau_i$ if $b = a_i$, for
$i = 1,\ldots,n$, otherwise $\phi(b)$. Also,
$[\overline{a\mapsto\tau}] = \id[\overline{a\mapsto\tau}^{\,n}]$.
