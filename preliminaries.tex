\section{Preliminaries}\label{prelimirares}

This section introduces basic definitions and notations. Meta-variable
usage and the syntax of types are given in Figure~\ref{fig:meta}.  
%Meta-variables can be possibly primed or subscripted).

\begin{figure} 
\begin{mdframed}
\[ \begin{array}[c]{llll}
\textrm{Class Name}         &\hspace{.1cm} & A         & \\
\textrm{Type variable}      &         & a,b & \\
\textrm{Type constructor}   &         & T              & \\
\textrm{Simple Constraint}  &         & \pi            & ::= A\,\overline{\tau}\\
\textrm{Set of Simple Constraints} &  & C            & \\
%\textrm{Unquantified Constraint} &    & \psi           & ::= C\Rightarrow \pi\\
\textrm{Constraint}         &         & \theta         & ::= \forall\,\overline{a}.\,C\Rightarrow\pi\\
\textrm{Simple Type}        &         & \tau,\rho     & ::= a \mid T \mid \tau\:\tau' \\
\textrm{Constrained Type}   &         & \delta & ::= C\Rightarrow \tau\\

\textrm{Type}               &         & \sigma         & ::= \forall\,\overline{a}.\,\delta \\
\textrm{Substitution}    &         & \phi         & \\
\end{array} \]
\end{mdframed} \vspace{-.2cm}
\caption{Syntax of Types}
\label{fig:meta}
\end{figure}

For simplicity and following common practice, kinds are not considered
in type expressions and type expressions which are not simple types
are not explicitly distinguished from simple types. 

As usual, we assume the existence of type constructor $\to$, that is
written as an infix operator ($\tau \to \tau'$). A type
$\forall\,\overline{a}.\,C\Rightarrow \tau$ is equivalent to
$C\Rightarrow \tau$ if $\overline{a}$ is empty and, similarly,
$C\Rightarrow \tau$ is equivalent to $\tau$ if $C$ is empty.

The set of type variables occurring in $X$ is denoted by $\tv(X)$,
where $X$ can be a type, a constraint, sets of types or constraints,
or a typing context.

Notation $\overline{x}^{\,n}$, or simply $\overline{x}$, is used
throughout this paper to denote the sequence $x_1 \cdots x_n$, or
$x_1, \ldots, x_n$, or $x_1;\ldots;x_n$, depending on the context
where it is used, where $n\geq 0$, and $x$'s can be either type
variables, or mappings, or bindings etc.  When used in a context of a
set, it denotes $\{x_1,\ldots,x_n\}$.

A substitution $\phi$ is a function from type variables to simple type
expressions. The identity substitution is denoted by
\id. $\phi(\sigma)$ (or simply $\phi\,\sigma$) represents the
capture-free operation of substituting $\phi(a)$ for each free
occurrence of $a$ in $\sigma$.

We overload the substitution application on constraints, constraint
sets and sets of types. Definition of application on these elements is
straightforward. The symbol $\circ$ denotes function composition and
$\dom{\phi}=\{\alpha \mid\ \phi(\alpha) \neq \alpha\}$.

The notation $\phi[\overline{a\mapsto\tau}^{\,n}]$ denotes the
substitution $\phi'$ such that $\phi'(b) = \tau_i$ if $b = a_i$, for
$i = 1,...,n$, otherwise $\phi(b)$. Also, $[\overline{a\mapsto\tau}] =
\id[\overline{a\mapsto\tau}^{\,n}]$.
