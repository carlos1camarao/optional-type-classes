\section{Preliminaries}\label{prelimirares}

In this section we introduce some basic definitions and notations. We
consider that meta-variables defined can appear primed or subscripted.

%Meta-variable usage is defined in the paper as follows: $x,y$ denote
%term variables, $\alpha, \beta$ ($a, b,...$
%in examples) type variables, $e$ a term,
%$\tau,\rho$ simple types, $\sigma$ a type, $\Gamma$ a typing context, 
%that is, a set of pairs written as $x:\sigma$, and $S$ a
%substitution. A constraint is formed by a pair of a class name $C$ and
%a sequence of types $\overline{\tau}$. We slighly abuse notation and 
%use $\kappa$ to denote both a single constraint and a constraint set.

The notation $\overline{a}^{\,n}$, or simply $\overline{a}$, denotes
the sequence $a_1 \cdots a_n$, or $a_1, \ldots, a_n$, or
$a_1;\ldots;a_n$, depending on the context where it is used, where
$n\geq 0$. When used in a context of a set, it denotes
$\{a_1,\ldots,a_n\}$. It can be used with more than one variable; for
example, in $\overline{x = e}^{\,n}$, it denotes the sequence
$x_1=e_1, \ldots, x_n = e_n$.

A substitution is a function from type variables to simple type
expressions. The identity substitution is denoted by
\id. $\phi(\sigma)$ (or simply $\phi\,\sigma$) represents the
capture-free operation of substituting $\phi(\alpha)$ for each free
occurrence of $\alpha$ in $\sigma$.

We overload the substitution application on constraints, constraint
sets and sets of types. Definition of application on these elements is
straightforward. The symbol $\circ$ denotes function composition and
$\dom{\phi}=\{\alpha \mid\ \phi(\alpha) \neq \alpha\}$.

The notation $\phi[\overline{\alpha}\mapsto\overline{\tau}]$ denotes
the updating of $\phi$ such that $\overline{\alpha}$ maps to
$\overline{\tau}$, that is, the substitution $\phi'$ such that
$\phi'(\beta) = \tau_i$ if $\beta = \alpha_i$, for $i = 1,...,n$,
otherwise $\phi(\beta)$. Also, $[\overline{\alpha}\mapsto\overline{\tau}]
= \id[\overline{\alpha}\mapsto\overline{\tau}]$.

