\section{Modularization of Instances}
\label{sec:modular-instances}

This paper does not attempt to discuss any major revision to Haskell's
module system. Our aim here is to briefly summarize a modification,
presented in \cite{Controlling-scope-instances}, that allows a modular
control of the visibility of instance definitions. This has the
additional benefit of enabling type classes to be optionally declared
by programmers. We essentially follow the work of Marco Gontijo and
Carlos Camarão \cite{Controlling-scope-instances}, summarized
below. The main idea is to introduce an additional rule, to account
for the possibility of type classes to be declared or not:

\begin{definition}[Type of overloaded variable]

If the type of an overloaded variable (i.e.~a variable that is
introduced in an instance definition) is not explicitly annotated in a
type class declaration, then the variable's type is the
anti-unification of instance types defined for the variable in the
current module; otherwise, it is the annotated type.

\label{overloaded-variable-type}
\end{definition}

Consider, for example:

\proga{xx\=\instance\ \=\member\ $a$ \=($b$:$x$) \=\kill\\
\module\ $L$ \where\+\\
  \ldots\ \ldots\ \\
  \instance\ \eempty\  = []\\
  \instance\ \iinsert\ = (:)\\
  \instance\ \>\member\ $a$ \>[]        \>= \False\\
             \>\member\ $a$ \>($b$:$x$) \>= ($a$==$b$) || \member\ $a$ $x$ 
}


\proga{xx\=\instance\ \=\member\ $a$ \=($b$:$x$) \=\kill\\
\module\ $T$ \where ...
}
            
Modularization of instance definitions is allowed in a simple way by
means of the importation and exportion of instances as shown in
\cite{Controlling-scope-instances}. Essentially, import and export
clauses can specify, instead of just names, occurrences of {\tt
  instance $T$ $D$}, where $T$ is a type and $D$ is a class name.
We have: 

  \[ \text{\module\ $M$ (\instance\ $T$ $D$ \where\ \ldots)} \]
specifies that the instance of type $T$ for class $D$ is exported in
module $M$. 

  \[ \text{\import\ $M$ (\instance\ $T$ $D$)} \]
specifies that the instance of type $T$ for class $D$ is imported from
$M$, in the module where the import clause occurs.

Alternatively could simply give a name to an instance, in an instance
declaration, and use that name in import and export clauses (see
\cite{Controlling-scope-instances}).

Among the advantages of this simple change, we cite (following
\cite{Controlling-scope-instances}):

\begin{itemize}

  \item programmers have better control of which entities are
    necessary and should be in the scope of each module in a program;

  \item it is possible to define and use more than one instance for
    the same type in a program;

  \item problems with orphan instances (i.e.~instances defined in a
    module where neither the definition of the data type nor the
    definition of the type class) do not occur (for example, distinct
    instances of \Either\ for class \Monad, say one from package
    \mtl\ and another from \transformers, can be used in a program);

  \item the introduction of newtypes, as well as the use of functions
    that include additional (-by) parameters (such as e.g.~\sortBy\ in
    module \Data.\List) can be avoided.

\end{itemize}

With instance modularization, programmers need to be aware of which
entities are exported and imported (i.e.~which entities are visible in
the scope of a module) and their types, i.e.~in particular if it is an
overloaded name or not. 

A simple change, like a type annotation for a variable ($x$) in a
module ($A$), can lead to a change in the scope of another module
($B$), which can thus lead to a change of the semantics of using this
variable in this module ($B$). Consider the following example:

.... 

% The change is very significant if the type annotation instantiates
% the type of $x$ so that overloading of $x$ is resolved in $A$, since
% this leads to a change in the type (and meaning) of $x$ in module
% $B$.





