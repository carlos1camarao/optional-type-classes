\section{Modularization of Instances}
\label{sec:modular-instances}

This section presents simple modifications to Haskell's module system
that are necessary to allow instances to have a modular scope
\cite{Controlling-scope-instances} (we do not attempt to discuss any
major revision to Haskell's module system). Essentially, import and
export clauses can specify, instead of just names, also {\tt instance
  $A$ $\overline{\tau}$}, where $\overline{\tau}$ is a (non-empty)
sequence of types and $A$ is a class name:
  \[ \parbox{\textwidth}{\rm{ \module\ $M$ (\instance\ $A$ $\overline{\tau}$, \ldots) \where\ \ldots}} \]
specifies that the instance of $\overline{\tau}$ for class $A$ is
exported in module $M$.
  \[ \parbox{\textwidth}{\rm{ \import\ $M$ (\instance\ $A$ $\overline{\tau}$, \ldots)}} \]
specifies that the instance of $\overline{\tau}$ for class $A$ is
imported from $M$, in the module where the import clause occurs.

In this paper we consider that, in the case of instances without a
type class, keyword \instance\ must be specified, in an import clause,
if there exists more than one visible instance of the overloaded name
in the imported module (if there is just one visible instance in the
imported module, \instance\ may be specified or not); similarly, when
exporting, \instance\ must be specified if there exists more than one
visible instance of the overloaded name in the module.

\subsection{Pros and Cons of Instance Modularization}

Among the advantages of having instances with modular scope, we cite:

\begin{itemize}

  \item The types of all names, overloaded or not, and of all
    expressions in a module depend on the scope of variables imported
    by the module, and can be inferred if not explicitly annotated.

  \item It is possible to define and use more than one instance for
    the same type or for the same type constructor in a program. For
    example, distinct instances of \Either\ for class \Monad\ (say one
    from package \mtl\ and another from \transformers), can be used in
    a program.

  \item Use of newtypes to wrap distinct instance definitions can be
    avoided, if the distinct instances are not used in the same
    module. For example, newtypes to wrap distinct instances of
    Monoids for integer types, one for handling addition and one for
    handling multiplication.
    
  \item Modularized instance scopes and optional type classes with a
    revised ambiguity rule can avoid the use of functions that include
    additional (-by) parameters, such as e.g.~the (first) parameter of
    function \sortBy\ in module \Data.\List.

  \item Modularized instance scopes with a revised ambiguity rule and
    optional type classes can be used to support overloaded record
    fields in a simple way, based on the automatic creation of a type
    class for each overloaded record field (Section
    \ref{sec:overloaded-record-fields}).

  \item The need for type-directed name resolution fades with modular
    instance scopes, a revised ambiguity rule and optional type
    classes (Section \ref{sec:type-directed-name-resolution}).
    
  \item Modularized instance scopes with a revised ambiguity rule and
    optional type classes may avoid the use of qualified imports (as
    used e.g.~in the {\em classy-prelude}, used in e.g.~Yesod
    \cite{Yesod}).

  \item Compile time consumed because the use of orphan instances can
    be avoided.  Orphan instances are instances defined in a module
    where neither the definition of the data type nor the definition
    of the type class occur. Any instance declaration in any module
    imported directly or indirectly by a module $M$ is visible in
    Haskell; in principle, a Haskell compiler must then read the
    interface files of each module imported by $M$, directly or
    indirectly, to check if it contains an instance declaration used
    in $M$, which can consume significant compile time.
    
\end{itemize}

However, instance modularization requires programmers to be aware of
which entities are exported and imported --- i.e.~which entities are
visible in the scope of a module --- and their types.  A simple
change, like a type annotation for a variable exported from a module,
can lead to a change in the semantics of using this variable in
another module.

Consider the canonical ambiguity example in Haskell, of {\tt
  (\sshow\ $\!$.$\!$ \rread)}, but considering instances without type
classes and with a modular scope:

\proga{xxxxxxxxxxxxxxxxx\=xx\= \+\kill
\module\ $M$ (\myshow, \myread, $f$) where\+\\
  \myshow$\,$::$\,$\Int\ $\rightarrow$ \String\\
  \myshow\ = \ldots \\ 
  \myread$\,$::$\,$\String\ $\rightarrow$ \Int\\
  \myread\ = \ldots\\
  $f$ = \myshow $\:$.$\:$\myread\-\\ \\

\module\ $N$ \where\+\\
  \import\ $M$ (\myshow, \myread, $f$)\\
  \myshow$\,$::$\,$\Bool\ $\rightarrow$ \String\\
  \instance\ \myshow\ = \ldots\\
  \myread$\,$::$\,$\String\ $\rightarrow$ \Bool\\
  \instance\ \myread\ = \ldots\\
  $g$ = $f$ "123"
}

In Haskell, any use of {\tt (\sshow\ $\!$.$\!$ \rread)} leads to a
type error, since the type {\tt (\SShow\ $a$, \RRead\ $a$)
  $\Rightarrow$ \String $\rightarrow$ \String} is ambiguous. In our
approach, the definition of $f$ in module $M$ is well-typed: there
exists in module $M$ a single instance of \myshow\ and \myread. As a
result, $f$ has type \String\ $\rightarrow$ \String. Its use in module
$N$ is (then) also well-typed ($f$ is a function that receives a value
of type \String\ and returns a value of type \String, according to the
definition of $f$ given in module $M$), even though {\tt (\myshow
  $\:$.$\:$\myread)\ "123"} is not well-typed in module $N$, since
{\tt (\myshow $\:$.$\:$\myread)} is ambiguous in $N$. Explicit
annotation of types of imported names would make module interfaces
clearer.


