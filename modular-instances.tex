\section{Modularization of Instances}
\label{sec:modular-instances}

This paper does not attempt to discuss any major revision to Haskell's
module system. Our aim here is to briefly summarize a modification,
presented in \cite{...}, that allows a modular control of the
visibility of instance definitions. This has the additional benefit of
enabling type classes to be optionally declared by programmers.

We essentially follow the work of Marco Gontijo and Carlos Camarão
\cite{Controlling-scope-instances}, summarized below. The main idea is
to have the following additional rule for allowing type classes to be
optionally specified.

\begin{definition}[Type of overloaded variables]

If the type of an overloaded variable (i.e.~a variable that is defined
in an instance definition) is not explicitly annotated in a type class
declaration, then the variable's type is the anti-unification of
instance types defined for the variable in the current module;
otherwise, it is the annotated type.

\label{overloaded-variable-type}
\end{definition}

Consider, for example, ...



