\section{Related Work}
\label{sec:related-work}

Haskell type system has been extended with several advanced typing features
such as functional dependencies~\cite{Jones2008}, type families~\cite{Chakravarty2005} and
GADTs~\cite{Chen2016}, just to name a few. To the best of our knowledge, there's no
previous work on optional declaration of type classes. In this section, we
summarize some recent Haskell type system extensions.

Functional dependencies (FDs) were introduced by Mark Jones as a way to specify
type class parameter dependencies to improve infered typesin the
context of MPTCs. FDs where also used to support some form of type level
programming~\cite{Hallgren2000} and to define heterogeneous lists and
extensible records~\cite{KiselyovLS04}.

Type families~\cite{Chakravarty2005} (TFs) where introduced as a ``more functional'' alternative
to FDs, that is relational in nature. However, there are some issues with
type families injectivity~\cite{Eisenberg2014} that motivated the so-called closed type
families and type families dependencies~\cite{Eisenberg2014a}. Closed type families define all
possible instances of a type family a priori and type families dependencies allows
the specification of parameter dependencies, in a similar fashion of FDs.
All type families related extensions cater to better type improvement.

Datatype promotion~\cite{Yorgey2012,Eisenberg2014} is useful Haskell extension that lifts user
defined algebraic datatypes to kinds and data constructors to types. Such
extension allows for defining some types dependently typed programs.
Singleton types and promoted functions~\cite{Eisenberg2012} have been used to automate
(through Template Haskell) some constructions commonly needed in Haskell-style
dependent types. Lindley and McBride~\cite{Lindley2013} describes some dependently
typed programs in Haskell and how to use GHC's constraint solver as theorem
prover to discharge proof obligations in a implementation of a merge-sort
algorithm.

Type level literals~\cite{type-lits} is an extension that complements datatype
promotion to numeric and string types. The Haskell prime proposal for
overloaded records relies on this extension to overload field
access and update functions. Our approach, based on optional declaration of
type classes, is simpler since it doesn't demand type promotion features and
it doesn't need to create an instance for each record field (overloaded or not)
as GHC's proposal.
