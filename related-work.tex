\section{Related Work}
\label{sec:related-work}

Haskell type system has been extended with several advanced typing
features such as functional dependencies~\cite{Jones2008}, type
families~\cite{Chakravarty2005} and GADTs~\cite{Chen2016}, just to
name a few. To the best of our knowledge, there's no previous work on
optional declaration of type classes. In this section, we summarize
some recent Haskell type system extensions.

Functional dependencies (FDs) were introduced by Mark Jones as a way
to specify type class parameter dependencies in order to avoid
ambiguity and to improve inferred types in the context of MPTCs. FDs
where also used to support some form of type level
programming~\cite{Hallgren2000} and to define heterogeneous lists and
extensible records~\cite{KiselyovLS04}.

Type families~\cite{Chakravarty2005} (TFs) where introduced as a
``more functional'' alternative to FDs (which is relational in
nature). However, there are some issues with type family
injectivity~\cite{Eisenberg2014} that motivated so-called closed type
families and type family dependencies~\cite{Eisenberg2014a}. Closed
type families define all possible instances of a type family a priori
and type family dependencies allows the specification of parameter
dependencies, in a similar way of FDs.  All type family related
extensions cater to better type improvement.

Datatype promotion~\cite{Yorgey2012,Eisenberg2014} lifts user defined
algebraic datatypes to kinds and data constructors to types. It allows
the definition of some dependently typed programs.  Singleton types
and promoted functions~\cite{Eisenberg2012} have been used to automate
(through Template Haskell) some constructions commonly needed in
Haskell-style dependent types. Lindley and McBride~\cite{Lindley2013}
describe some dependently typed programs in Haskell and how to use
GHC's constraint solver as a theorem prover to discharge proof
obligations in an implementation of a merge-sort algorithm.

Type level literals~\cite{type-lits} is an extension that complements
datatype promotion to numeric and string types. The Haskell prime
proposal for overloaded record fields relies on this extension to
overload field access and update functions. Our approach, based on
optional declaration of type classes, does not demand type promotion
features and does not need to create an instance for each record field
(overloaded or not). 
