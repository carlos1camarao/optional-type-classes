\subsection{Entailment}
\label{sec:entailment}

We define in this appendix constraint set provability, called
entailment in Haskell terminology. Entailment of a set of constraints
is defined with respect to the set of class and instance declarations
that occur in a program, a so-called program theory
(cf.~\cite{Understanding-FDs-via-CHRs}).

\begin{Definition}

A program theory $P$ is a set of axioms of first-order logic generated
from class and instance declarations occurring in the program, as
follows (where $C \Rightarrow \pi$ is considered syntactically
equivalent to $\pi$ if $C$ is empty):

\begin{itemize}

\item For each class declaration
  \vspace*{-.5\baselineskip}
  \prog{class $C$ $\Rightarrow$ \TCC\ $\overline{a}$ where \ldots}
  \vspace*{-.5\baselineskip}
the program theory contains the following formula if $C$ is not empty:
  \vspace*{-.5\baselineskip}
    \[ \forall\,\overline{a}.\,C \Rightarrow \TCC\ \overline{a}\]

\item For each instance declaration
  \vspace*{-.5\baselineskip}
  \prog{instance $C$ $\Rightarrow$ \TCC\ $\overline{t}$ where \ldots}
  \vspace*{-.5\baselineskip}
the program theory contains the following formula:
  \vspace*{-.5\baselineskip}
               \[ \forall\,\overline{a}.\,C \Rightarrow \TCC\ \overline{t} \]
where $\overline{a} = \tv(\overline{t}) \cup \tv(C)$;
if $C$ is empty, then the instance declaration is of the form 
  \vspace*{-.5\baselineskip}
  \prog{instance \TCC\ $\overline{t}$ where \ldots}
  \vspace*{-.5\baselineskip}
and the program theory contains the formula:
  \vspace*{-.5\baselineskip}
     \[ \forall\,\overline{a}.\,\TCC\:\: \overline{t} \]
  \vspace*{-.5\baselineskip}
\end{itemize}
\label{program-theory-def}
\end{Definition}

\vspace*{-.5\baselineskip}
The property that a set of constraints $C$ is entailed by a program
theory $P$, written as $P \entail C$, is defined in Figure
\ref{Entailment-fig}.  Following
\cite{Associated-types-with-class,Associated-type-synonyms},
entailment is obtained from closed constraints contained in a program
theory $P$.

%  (unlike in \cite{MarkJones94a}, entailment and satisfiability rules
% do not move constraints from types to non-closed constraints in the
% program theory, nor vice-versa).

\begin{figure}
   \[ \begin{array}{r}
         \begin{array}{cr}
   		\displaystyle\frac{}{P \entail \emptyset} (\ento)\hspace*{.3cm} &
		\displaystyle\frac
                        {(\forall\,\overline{a}.\,C\Rightarrow \pi) \in P}
			{P \entail \{ (C\Rightarrow\pi)[\overline{a}\,\mapsto\,\overline{\tau}] \} }
			(\entinst)
         \end{array}\\ \\
         \begin{array}{cr}
		\displaystyle\frac
			{P \entail C \:\:\: P \entail \{C\Rightarrow\pi\} }
			{P \entail \{ \pi \}} (\entmp)
			&
		\displaystyle\frac
			{P \entail C \:\:\: P \entail D}
			{P \entail C \cup D} (\entn)
	 \end{array}
       \end{array}
   \]
\caption{Constraint Set Entailment}
\label{Entailment-fig}
\end{figure}

\begin{Definition}[Entailed instances and Entailing Substitutions]

  \normalfont
  
$\lfloor C \rfloor_P$ is the set of {\em entailed instances\/} of constraint set
$C$ with respect to program theory $P$:
\[ \lfloor C \rfloor_P = \{\,\phi(C) \,\mid\, P\, \entail \phi(C)\, \} \]
and the corresponding substitutions as {\em entailing substitutions\/}:
  \[ \entailingSubs(C,P) = \{\,\phi \,\mid\, P\, \entail \phi(C)\, \} \]

%If $\phi(C) \in \lfloor C \rfloor_P$ then $\phi$, denoted by  is called a
%entailing substitution for $C$ in $P$.

\end{Definition}

\begin{Example} {\rm
As an example, consider:
  \[ P = \{ \forall a,b.\,D\, a\,
             b\Rightarrow C\, \text{\tt{[$a$]}}\, b, D\, \text{\it Bool\/}\,
             \text{\tt {[{\it Bool\/}]}}\}\]
We have that
  $\lfloor C\:\:a\:\:a\rfloor_P$ =
  $\lfloor C\:\text{\tt{[\it Bool\/}]}\: \text{\tt{[\it Bool\/}]}\rfloor_P$.
Both constraints
  $D\:\text{\Bool\ [\Bool]} \Rightarrow C\:\text{\tt{[{\it Bool\/}]}}\: \text{\tt [{\it Bool\/}]}$
and
  $C\:\text{\tt{[{\it Bool\/}]}}\: \text{\tt [{\it Bool\/}]}$
are members of
  $\lfloor C\:\:a\:\:a\rfloor_P$ and also members of
  $\lfloor C\:\text{\tt{[\it Bool\/}]}\: \text{\tt{[\it Bool\/}]}\rfloor_P$.

A proof that $P \entail \{ C\:\text{\tt{[\it Bool\/}]}\: \text{\tt{[\it Bool\/}]} \}$
holds can be given from the entailment rules given in Figure \ref{Entailment-fig},
since this is the conclusion of rule (\entmp) with premises
  $P \entail \{ D\:\text{\it Bool\/}\: \text{\tt{[\it Bool\/}]} \}$ and
  $P \entail \{ D\:\text{\Bool\ [\Bool]} \Rightarrow C\:\text{\tt{[{\it Bool\/}]}}
                                                                               \:\text{\tt{[{\it Bool\/}]}}\}$,
and these two premises can be derived by using rule (\entinst).}

\end{Example}

Equality of constraint sets is considered modulo type variable
renaming. That is, constraint sets $C,D$ are also equal if there
exists a renaming substitution $\phi$ that can be applied to $C$ to
make $\phi\,C$ and $D$ equal.

$\phi$ is a renaming substitution if for all $a\in\dom(S)$ we have
that $\phi(a)=b$, for some type variable $b\not\in\dom{\phi}$.


