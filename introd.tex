\section{Introduction}
\label{sec:intro}

The Haskell approach to overloading
\cite{ghc-users-guide,MarkJones94a} is a distinctive feature of the
language that has been widely used. In Haskell, the types of
overloaded names (or symbols) are provided in {\em type class\/}
declarations and their definitions are introduced in {\em class
  instance\/} declarations. The type of an expression that uses an
overloaded symbol is a constrained polymorphic type, that contains a
type class constraint if the overloading is not resolved. For example,
consider (where the body of definitions are replaced by {\tt
  \ldots}):

\progb{xxxxxxxxxxxxxxxxx\=xx\=  \+ \kill
   \class\ \SShow\ $a$ \where\+ \\ 
      \sshow\ :: $a \rightarrow$ \String \-\\ 
   \instance\ \SShow\ \Int\ \where\+ \\ 
     \sshow\ = \ldots\-\\
   \putStrLn\ :: \String\ $\rightarrow$ \IO\ ()\\
   \putStrLn\ = \ldots
}
The type of \print, defined as: \texttt{\print\ $x$ =
  \putStrLn\ (\sshow\ $x$)} is: \[ \texttt{\SShow\ $a$ $\Rightarrow$
  $a$ $\rightarrow$ \IO\ ()} \] where constraint
(\texttt{\SShow\ $a$}) is due to the use of the overloaded function
\sshow, of type \SShow\ $a$ $\Rightarrow$ $a$ $\rightarrow$
\String\ (declared in class \SShow).

A type in Haskell is of the form $\forall\,\overline{a}.\,C
\Rightarrow \tau$, where $\overline{a}$ is a possibly empty set of
type variables, $C$ is a possibly empty set of constraints and $\tau$
is a simple (unconstrained, unquantified) type. A constraint is a
class name followed by a sequence of simple types.

Overloading resolution in Haskell is {\em context-dependent}, that is,
it depends on the context where an expression occurs. For function
calls, this means that it depends also on the type of the function
result (i.e.~overloading resolution in an expression $f\:x$ can be
based also on the type required by the context in which $f\:x$
occurs). Also, constants can be overloaded --- for example, literals
like {\tt 1}, {\tt 2} etc.~can be used to represent fixed and
arbitrary precision integers as well as fractional numbers.

The possibility of using overloading without requiring overloading
resolution in each function call makes overloading more
expressive. Functions with types that differ only on the type of the
result can be overloaded, for example {\em read\/} functions with
types {\tt \String\ $\rightarrow$ \Bool}, {\tt \String\ $\rightarrow$
  \Int}, each taking a string and generating the denoted value in the
corresponding type.  More importantly, functions can use overloaded
functions in the definition of other functions, as well as define
instances for distinct type constructors. For example, by using
mapping and folding functions, e.g.~with types {\tt \Functor\ $f$
  $\Rightarrow$ $(a\rightarrow b)\rightarrow f\:a\rightarrow f\:b$}
and {\tt \Foldable\ $f$ $\Rightarrow$ $(a\rightarrow b\rightarrow
  b)\rightarrow b\rightarrow f\:a\rightarrow b$}, respectively,
mapping and folding functions for instances of $f$ as well as other
higher-order polymorphic functions that use mapping and folding
functions can be defined.

%Context-dependency allows in this way overloading to have a more
%prominent role in the presence of parametric polymorphism, as explored
%mainly in the programming language Haskell.

Another fundamental characteristic of overloading in Haskell, that
supports context-dependent overloading resolution, is the reliance on
an open-world approach.
%In general what is an open-world approach is not very well explained.
This implies that overloading resolution needs to be checked only when
there exist variables in constraints that occur only in the
constraints, i.e.~do not occur in the simple (unconstrained part of
the) type. The fact that overloading resolution does not need to be
checked in every function application is important not only to defer
overloading resolution but also for compile time efficiency.  In the
presence of multi-parameter type classes (MPTCs), functional
dependencies (FDs) \cite{Type-classes-with-FDs-MarkJones00,Jones:2008}
or type families (TFs) \cite{Chakravarty2005,Schrijvers:2008} are used
to avoid ``ambiguity''.  With MPTCs, ``do not occur in the simple
type'' is modified in GHC \cite{GHC}, the prevailing Haskell compiler,
to ``is not uniquely determined from the set of type variables that
occur in the simple type''. This unique determination is such that,
for each type variable $a$ that occurs in the constraints $C$ but not
in the simple type $\tau$ of a constrained type $C\Rightarrow \tau$,
there must exist a FD $b \mapsto a$ for some $b$ in $\tau$ (in the
case of TFs, a similar unique determination is specified). We use $b
\mapsto a$ for a FD, instead of $b \rightarrow a$, to avoid confusion
with the notation used to denote functional types.

However, in Haskell the presence of type variables in constraints that
are not uniquely determined from type variables in the simple type
characterizes that ``the type is ambiguous'', not that overloading
resolution needs to be checked.  This guarantees that new instances
can be introduced in well-typed Haskell programs without ever
introducing a type error. On the other hand, i) it does not consider
that the presence of unreachable type variables in constraints from
the set of type variables in the simple type (Section
\ref{sec:ambiguity}) may be used to check overloading resolution,
avoiding the need of using an extra mechanism for dealing with
ambiguity problems in the presence of MPTCs; and ii) it does not
distinguish overloading resolution from ambiguity, and thus cannot
recover an intuitive common sense meaning of ambiguity as existence of
two or more instances that can be used for an overloaded name in an
expression, when overloading is resolved; iii) the expressivity of
overloading obtained together with optional type classes a simple
support of type directed name resolution (Subsection
\ref{sec:type-directed-name-resolution}) and overloaded record fields
(Subsection \ref{sec:overloaded-record-fields}).
%Further examples are envisaged but are left for further work.

The paper presents minimalist modifications to Haskell's module system
that are necessary for overloaded names to have a modular scope, with
types annotated or not in type classes. Benefits and drawbacks of
instance modularization are discussed in Section
\ref{sec:modular-instances}.

The mechanism of optional type classes presented allows instance
declarations of the form $\instance\ x=e$, without explicitly
declaring $x$'s type in a type class.  Its type is determined from the
anti-unification of visible instance types for $x$. The algorithm
used for computing the type of $x$ is presented in
Section~\ref{sec:anti-unif}.

The automatic creation of a type class for $x$ simply uses $x$ also as
the name of the class. The initial constraint introduced when $x$ is
used is $x\:\tau$, where $\tau$ is the anti-unification of visible
instance types of $x$. This was preferred to the use of a single
parameter type class $x\: a$, or a possibly MPTC $x\:\overline{a}$,
where $\overline{a}$ is a sequence of parameters that corresponds to
the sequence of type variables that occur in the the anti-unification
of visible instance types of $x$, ordered (say) lexicographically.

The proposed approach is formalized in
Section~\ref{Optional-type-classes}, where a type system is presented
for a mini-Haskell language with type classes optionally declared.

%The added flexibility of the overloading mechanism, with respect to
%Haskell, is illustrated by the enabled simple support for overloaded
%record fields (Section \ref{sec:overloaded-record-fields}) and type
%directed name resolution (Section
%\ref{sec:type-directed-name-resolution}).

A type inference algorithm is presented in Section
\ref{sec:type-inference}. It is proved to be sound and to infer, for a
given expression in a given typing context, a type that is a minimal
generalization of types derivable in the type system.

% A semantics is presented in Section \ref{sec:semantics}. 

Related work is outlined in Section \ref{sec:related-work} and Section
\ref{sec:conclusion} concludes.

%Appendices include the definition of the entailment relation
%(\ref{sec:entailment}), used in the type system, the improvement and
%context-reduction relations
%(\ref{sec:improvement},\ref{sec:context-reduction}), whose composition
%yield the constraint-set simplification relation, also used in the
%type system, and the constraint-set satisfiability function
%(\ref{sec:satisfiability}), used in the improvement relation and in
%the type inference algorithm.

A prototype implementation of a type inference algorithm for Haskell
supporting overloading without the need of defining a type class is
available \cite{opt-rep}.
