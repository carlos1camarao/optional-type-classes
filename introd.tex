\section{Introduction}
\label{sec:intro}

The versions of Haskell supported by GHC \cite{GHC} --- the prevailing
Haskell compiler --- are becoming complex, to the point of affecting
the view of Haskell as the best choice for general-purpose software
development. A basic issue in this regard is the need of extending the
language to allow multiple parameter type classes (MPTCs). This
extension is thought to require additional mechanisms, such as
functional dependencies or type families. In another paper
\cite{JBCS-Ambiguity-and-constrained-polymorphism}, we have shown that
the introduction of MPTCs in the language can be done without the need
of additional mechanisms: a simplifying change is sufficient, to
Haskell's ambiguity rule. Interested readers are referred to
\cite{JBCS-Ambiguity-and-constrained-polymorphism}. The main ideas are
summarized below.

Haskell with MPTCs uses constrained types of the form $\forall
\overline{a}.\,C \Rightarrow \tau$, where $C$ is a set of constraints
and $\tau$ is a simple (unconstrained, unquantified) type; a
constraint is a class name followed by a sequence of type variables.

In (GHC) Haskell, ambiguity is a property of a type: a type $\forall
\overline{a}.\,C \Rightarrow \tau$ is ambiguous if there exists a type
variable that occurs in the constraints ($C$) that is not uniquely
determined from the set of type variables that occur in the simple
type ($\tau$). This unique determination specifies that, for each type
variable $a$ that occurs in $C$ but not in $\tau$ there must exist a
functional dependency $b \mapsto c$ for some $b$ in $\tau$ (or a
similar unique determination specified via type families). Notation $b
\mapsto c$ is used, instead of $b \rightarrow c$, to avoid confusion
with the notation used to denote functional types.

A new definition, which we prefer to call here {\em expression
  ambiguity\/} (in \cite{JBCS-Ambiguity-and-constrained-polymorphism}
it is called {\em delayed closure ambiguity\/}), uses a similar
property, of variable reachability, that is independent of functional
dependencies and type families: a type variable $a$ that occurs in a
set of constraints is reachable from the set of type variables in
$\tau$ if it occurs in $\tau$ or there exists a type variable $b$ in
$C$ that is reachable. For example, in $C \Rightarrow b$, where
$C=(D\: a\: b, E\: a)$, type variable $a$ is reachable from the set of
type variables in $b$.

The presence of unreachable variables in a constraint $\pi\in C$
characterizes overloading resolution, or, in other words, that
overloading is resolved for $\pi$: it characterizes that there is no
context in which an expression with such a type could be placed that
could instantiate such unreachable variables. The presence of
unreachable variables does not necessarily imply ambiguity. Ambiguity
is a property of an expression, and it depends on the context in which
the expression occurs, and on entailment of the constraints on the
expression's type.

Entailment of constraints and its algorithmic (functional) counterpart
are well-known in the Haskell world (see
e.g.~\cite{MarkJones94a,TheoryOfOverloading,JBCS-Ambiguity-and-constrained-polymorphism}).

Informally, a set of constraints $C$ is entailed (or satisfied) in a
program $P$ if there exists a substitution $\phi$ such that $\phi(C)$
is contained in the set of instance declarations of $P$, or is implied
by the transitivity implied by the set of class and instance
declarations occuring in $P$. For a formal definition, see
e.g.~\cite{MarkJones94a,JBCS-Ambiguity-and-constrained-polymorphism}. In
this case we say that $C$ is entailed by $\phi(C)$. 

For example, {\tt \Eq\ [[\Integer]]} is entailed if we have instances
{\tt \Eq\ \Integer} and {\tt \Eq\ $a$ => \Eq\ [$a$]}, visible in the
context where an expression whose type has a constraint {\tt
  \Eq\ [[\Integer]]} occurs.

If overloading is resolved for a constraint $C$ occurring in a type
$\sigma = C,D \Rightarrow \tau$ then exactly one of the following
holds:
\begin{itemize}

\item $C$ is entailed by a single instance; in this case a type
  simplification (also called ``improvement'') occurs: $\sigma$ can be
  simplified to $D \Rightarrow \tau$;

\item $C$ is entailed by more than instance; in this case we have a
  type error: ambiguity; 

\item $C$ is not entailed (by any instance); in this case we have also
  a type error: unsatisfiability.

\end{itemize}

Note that variables in a single constraint are either all reachable or
all unreachable. If they are unreachable, either the constraint can be
removed, in the case of single entailment, or there is a type error
(either ambiguity, in the case of two or more entailments, or
unsatisfiability, in the case of no entaiment).

Instead of being dependent on the specification or not of functional
dependencies or type families, ambiguity depends on the existence of
(two or more) instances in a program context when overloading is
resolved for a constraint on the type of an expression.

The possibility of a modular control of the visibility of instance
definitions conforms to this simplifying change. This is the subject
of Section \ref{sec:modular-instances}.

Also in conformance with this change is the possibility, explored in
this paper, of allowing type classes to be optionally declared by
programmers, i.e. for allowing programmers to overload symbols without
having to declare the types of these symbols in type classes. 

A type system and a type inference algorithm for a core-Haskell
language where type classes can be optionally declared is presented in
Section \ref{Optional-type-classes}.  The idea is based on defining
the type of unanottated overloaded symbol as the anti-unification of
instance types defined for the symbol in a module, by automatically
creating a type class with a single overloaded name. This depends on a
modularization of instance visibility (as well as on a redefintion of
Haskell's ambiguity rule). 

The paper presents the modifications to Haskell's module system that
are necessary to allow instances to have a modular scope, based on
previous work published by one of the authors. The definition of the
type of overloaded symbols as the anti-unification of available
instance types and the redefined ambiguity rule is also based on
previous works by the authors.

The added flexibility to Haskell-style of overloading is illustrated
by defining a type system and a type inference algorithm that allows
overloaded record fields (Section \ref{sec:overloaded-record-fields}).

% The redefinition of Haskell's ambiguity rule can also be used to
% address some of the issues related to type directed name resolution
% (Section \ref{sec:type-directed-name-resolution}).

