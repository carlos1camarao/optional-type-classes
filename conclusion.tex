\section{Conclusion}
\label{sec:conclusion}

This paper has presented an approach for allowing type classes to be
optionally declared by programmers, so that programmers can overload
symbols without declaring their types in type classes. 

An overloaded symbol is defined by means of an instance declaration
that is a normal declaration with keyword \instance. The type of an
overloaded symbol is automatically determined from the
anti-unification of instance types defined for the symbol in the
relevant module.

The approach depends on a modularization of instance visibility, as
well as on a redefinition of Haskell's ambiguity rule. The paper
presents the simple modifications to Haskell's module system that are
necessary for allowing instances to have a modular scope.

We have provided an illustration of the added flexibility by showing
how overloaded record fields can be allowed in the presence of a
presented type system that supports instance modularization and
instance definitions of undeclared type classes that have a single
member.
