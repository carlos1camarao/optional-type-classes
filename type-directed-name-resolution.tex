\subsection{Type directed name resolution}
\label{sec:type-directed-name-resolution}

The proposal for type directed name resolution (TDNR)
\cite{TDNR-proposal} is similar to overloading, but uses the so-called
dot notation, as in object-oriented languages, to provide an
alternative way to specify which function is intended, based on the
type of the value that occurs before the dot. For example, instead of:

\progb{xx\=\kill
  \module\ $U$ \where\+\\
    \import\ \Button\ (\Button, \reset) \as\ $B$\\
    \import\ \Canvas\ (\Canvas, \reset) \as\ $C$\\ 
    $f\,$::$\,$\Button\ $\rightarrow$ \Canvas\ $\rightarrow$ \IO()\\
    $f$ $b$ $c$ = $B$.\reset\ $b$ >> $C$.\reset\ $c$
}

TDNR proposes the use of dot notation to enable the definition of $f$
to be: % (avoiding module qualification)

\progb{
    $f\,$::$\,$\Button\ $\rightarrow$ \Canvas\ $\rightarrow$ \IO()\\
    $f$ $b$ $c$ = $b$.\reset\ >> $c$.\reset
}

Optional type classes can avoid module qualification in the definition
of $f$ and allows it to be named \reset:
    
  \progb{xx\=\kill
      \module\ $U$ \where\+\\
      \import\ \Button\ (\Button,\reset)\\
      \import\ \Canvas\ (\Canvas,\reset)\\
      \reset$\,$::$\,$(\Button, \Canvas) $\rightarrow$ \IO()\\
      \instance\ \reset\ ($b$,$c$) = \reset\ $b$ >> \reset\ $c$
  }

The type of \reset\ in module $U$ is:
\[ \texttt{ \reset\ ($a$ $\rightarrow$ \IO()) => $a$ $\rightarrow$ \IO()} \]
  
An overloaded curried instance definition:

  \progb{xx\=\kill
      \reset$\,$::$\,$\Button\ $\rightarrow$ \Canvas\ $\rightarrow$ \IO()\\
      \instance\ \reset\ $b$ $c$ = \reset\ $b$ >> \reset\ $c$
  }
leads to \reset\ having type:
{\tt \reset\ ($a$ $\rightarrow$ $b$) $\Rightarrow$ $a \rightarrow b$}, where $a\rightarrow b$ can be
instantiated to one of the following types:
\[ \begin{array}{l}
  \Button\ \rightarrow \,\IO\texttt{()} \\
  \Canvas\ \rightarrow \,\IO\texttt{()}\\
  \Button\ \rightarrow \,\Canvas\ \rightarrow \IO\texttt{()}
\end{array}
\]
  
%The simplification of type \mbox{$\reset\ (\Button\ \rightarrow \IO())
%  \Rightarrow \IO()$} to {\tt \IO()} uses syntactically equivalence
%between constrained types, when {\tt
%  \reset\ (\Button$\,\rightarrow$\,\IO())} is in the program theory
%(when a resolved constraint is in scope, it can be eliminated).

