\section{Type inference}
\label{sec:type-inference}

In this section we present a type inference algorithm for
mini-Haskell, and discuss soundness and completeness of type inference
with respect to the type system.

For this, we need to consider functional counterparts of relations
used in the type system (Section \ref{Optional-type-classes}). The
main one is satisfiability, the counterpart of entailment (Subsection
\ref{sec:satisfiability}).  We do not consider in this paper
algorithms for constraint set simplification; interested readers may
consult~\cite{OutsideIn2011}. In an abuse of notation the same symbols
are used in this section for functions corresponding to the relations
of constraint set simplification ($\simplifies{P}$, where $P$ is a
program theory) and type generalization (\gen).

We use partial orders on types, constraints, substitutions, and typing
contexts with program theories, defined in Figure \ref{Order}.

Type ordering disregards constraint set entailment, which is important
only for considering whether a constraint $\pi$ can be removed from a
constraint $C$ occurring in a constrained type $C \Rightarrow \tau$;
$\pi$ can be removed if and only if overloading for $\pi$ has been
resolved and there exists a single entailing substitution for $C$, as
defined in Figure \ref{fig:constraint-set-improvement}, page
\pageref{fig:constraint-set-improvement}.

\begin{figure}
   \[ \begin{array}{cc}
   	\displaystyle\frac
          {}
          {\sigma \leq \phi\,\sigma}
   	  & 
   	\displaystyle\frac
   	  {}
   	  {\pi \leq \phi\,\pi} \\[.4cm]
        \multicolumn{2}{c}{
          \displaystyle\frac
	    {\Gamma(x) \leq \Gamma'(x) \text{ for all } x\in \dom{\Gamma}}
	    {P;\Gamma \leq P;\Gamma'}}\\[.4cm]
        \multicolumn{2}{c}{
          \displaystyle\frac
          {\text{there exists $\phi_1$ such that $\phi = \phi_1 \circ \phi'$}}
          {\phi\leq \phi'}}
  \end{array} \]
\caption{Partial orders}
\label{Order}
\end{figure}

A type inference algorithm for core-Haskell is presented in Figure
\ref{Type-inference-fig}, using rules of the form $P;\Gamma \ti
e:(\delta,\phi)$, which means that $\delta$ is the least (principal)
type derivable for $e$ in typing context $\phi\Gamma$ and program
theory $P$, where $\phi\Gamma \leq \Gamma$ and, whenever $\Gamma' \leq
\Gamma$ is such that $P;\Gamma' \ti e: (\delta',\phi')$, we have that
$\phi\Gamma \leq \Gamma'$ and $\delta' \leq \phi\delta$. Furthermore,
we have that $P;\phi\,\Gamma \ti e:(\delta,\phi')$ holds whenever
$P;\Gamma \ti e:(\delta,\phi)$ holds, where $\phi'\leq \phi$
(cf.~theorem \ref{Minimal-type-minimal-typing-context} below).

\begin{Example} {\rm
Consider expression $x$ and ty\-ping context $\Gamma = \{ f: \Int
\rightarrow {\it Int}, x:\alpha \}$; we can derive $\Gamma \ti f\: x:
(\Int, \phi)$, where $\phi = [\alpha \mapsto \Int]$. From
$\phi\,\Gamma = \{ f: \Int \rightarrow \Int, x:\Int \}$, we can derive
$\phi\,\Gamma \ti f\: x:(\Int,\id)$.}
\end{Example}

\begin{Theorem}

If $P;\Gamma \ti e:(\delta,\phi)$ holds then $P;\phi\,\Gamma \ti
e:(\delta,\phi')$ holds, where $\phi'\leq \phi$.

\label{Minimal-type-minimal-typing-context}
\end{Theorem}

\begin{figure}
\[ \begin{array}{cc}
      \displaystyle\frac
        {\begin{array}[t]{ll}
            (\Gamma(\self)(x) = \forall\,\overline{a}.\,\delta) \in \Gamma &
            \overline{b} \mbox{{\rm fresh}}
         \end{array}}
        {P;\Gamma \ti x: (\delta[\overline{a} \mapsto \overline{b}], \id)} (\VVARi) \\ \\

	\displaystyle\frac
          {\begin{array}[t]{lll}
            P;(\Gamma,x:a) \ti e: (C \Rightarrow \tau,\phi) & a\: \mbox{{\rm fresh}} & \tau' = \phi\,a
           \end{array}}
	  {P;\Gamma \ti \lambda x.\:e: (C\Rightarrow \tau' \rightarrow \tau, \phi)} (\ABSi) \\ \\

	\displaystyle\frac
	 {\begin{array}[t]{ll}
             P;\Gamma \ti e: (C \Rightarrow \tau_1, \phi_1) & P;\phi_1\Gamma \ti e': (C' \Rightarrow \tau_2, \phi_2)\\
             \phi' = \mguI(\tau_1 = \tau_2\rightarrow a) & a\: \mbox{{\rm fresh}}, \:
                \phi = \phi' \circ \phi_2 \circ \phi_1\:\: \  \\
             \tau = \phi\:a, \: V = \tv(\tau) \cup \tv(\phi C) & (\phi C \oplus_V\,\phi C') \simplifies{P_\Gamma} D
           \end{array}}
	  {\Gamma \ti e\:e': (D\Rightarrow \tau,\phi)} (\APPi) \\ \\

	\displaystyle\frac
	  {\begin{array}{ll}
             \Gamma \ti e\!:(C\Rightarrow \tau,\phi_1) & C \simplifies {P_\Gamma} C' \\
             \gen(\sigma,C'\Rightarrow \tau,\tv(\phi_1\Gamma)) & \phi_1\Gamma,\,x\!:\!\sigma \ti e_2\!:(\delta,\phi)
          \end{array}}
	 {\Gamma \ti \mlet\ x=e\ \iin\ e':(\delta,\phi)} (\LETi)
\end{array} \]
\caption{Type Inference}
\label{Type-inference-fig}
\end{figure}

Relation $\mgu$ is the most general (least) unifier
relation~\cite{Robinson65}: $\mgu(\Tau,\phi)$ is defined to hold
between a set of pairs of simple types $\Tau$ and a substitution
$\phi$ if i) $\phi$ is a unifier of every pair in $\Tau$ (i.e.~$\phi
\tau = \phi\tau'$ for every $(\tau,\tau')\in \Tau$), and ii) it is the
least such unifier (i.e.~if $\phi'$ is a unifier of all pairs in
$\Tau$, then $\phi\leq \phi'$). The relation holds similarly for
constraints instead of types.

$\mguI$ is a function that gives a most general unifier of a set of
pairs of simple types (or simple constraints). We define also that
$\phi = \mguI(\tau = \tau')$ is an alternative notation for
$\phi = \mguI(\{(\tau, \tau')\})$. We have:

\begin{Theorem}[Soundness]

If $P;\Gamma \ti e: (\delta,\phi)$ holds then $P;\phi\Gamma \vdash e: \delta$ holds.

\label{thm:type-inference-sound}
\end{Theorem}

\begin{Theorem}[Principal type]

If $P;\Gamma \ti e: (\delta,\phi)$ holds then, for all $\delta'$ such
that $P;\phi\Gamma \vdash e: \delta'$ holds, we have that $\delta \leq
\delta'$.

\label{thm:principal-type}
\end{Theorem}

A completeness theorem does not hold. For example, the canonical
Haskell ambiguity example of expression $e_0 = (\sshow\ $\!\!$.$\!\!$
\rread))$ --- where \sshow\ has type \SShow a $\Rightarrow$ $a
\rightarrow$ \String, and \rread\ has type \RRead a $\Rightarrow$
\String $\rightarrow a$ ---, we have that there exists $P$ and
$\Gamma$ such that $P;\Gamma \vdash e_0: \String \rightarrow \String$
holds, but there is no $\delta,\phi$ such that $P;\Gamma \vdash e_0:
(\delta,\phi)$ holds.

The greater simplicity obtained by allowing type instantiation to
occurr in a context-independent way, in a type system for a language
with support for context-dependent overloading, has significant
counterparts. The disadvantages are: ambiguous expressions are allowed
to be well-typed, and there exist several translations for
expressions, one of them a principal translation, for a semantics
defined inductively on the type system rules.

A declarative specification of type inference, with a unique type
derivable for each expression, where type instantiation is restricted
to be done only in a context-dependent way, defined by considering
functions used in the type inference algorithm as relations, is a
possible alternative. In this case, the type inference algorithm is
obtained directly from a declarative specification of the type system
by transforming relations used into functions. The fact that every
element has a unique type is consonant with everyday spoken
language. It is straightforward to define, a posteriori, the set of
types that are valid instances of the type of an expression.

The fact that only a single type can be derived for each expression
rules out the possibility of having distinct derivations for an
expression's type. Thus, an error message for an expression such as
{\tt (\sshow\ $\!\!$.$\!\!$ \rread)}, in a context with more than one
instance for \SShow\ and \RRead, should be that the expression can not
be given a well-defined semantics (there is no type that would allow
it to have a well-defined semantics). Distinct meanings of {\tt
  (\sshow\ $\!\!$.$\!\!$ \rread)} would be obtained from distinct
instance types of \sshow\ and \rread.

Type inference for mini-Haskell is obtained by extending type
inference for core-Haskell in straightforward way, namely by directly
using the rules of Subsection \ref{sec:mini-Haskell} (Figures
\ref{fig:mini-haskell-module-rule}, \ref{fig:import-relation} and
\ref{fig:mini-haskell-rules-for-declarations}), by replacing relations
with functions.

The overloading resolution theorem below considers a type inference
algorithm that differs from mini-Haskell's by only i) disregarding
improvement, and thus not removing any {\em resolved\/} constraint,
and ii) not allowing constraints in an argument to be removed, using
$C \cup C'$ instead of $C \oplus_V C'$ (where $V = \tv(\tau)\cup
\tv(C)$). We use $\tiUm$ instead of $\ti$ in typing formulas of this
(mini-Haskell without improvement and constraint selection by
$\oplus$) type inference algorithm. We also define a program context
$\CO[e]$ as any expression that has $e$ as a subexpression.

\begin{Theorem}[Overloading Resolution]
  For all $P, \Gamma, e, C, \tau$ such that $P;\Gamma \tiUm e:
  C\Rightarrow \tau$ holds, $\pi\in C$ and
  $a\in\unreachableVars(\{\pi\},\tv(\phi(\tau)))$, then, for all
  $P;\phi\Gamma\tiUm\CO[e]: C'\Rightarrow \tau'$ that holds, we have
  that $\pi \in C'$.
\label{thm:overloading-resolution}
\end{Theorem}

{\em Proof\/}: By induction on the structure of $\CO[e]$. $\qed$

Informally speaking, theorem \ref{thm:overloading-resolution} shows
that there is no program context where an expression can be used that
will cause a constraint with an unreachable type variable to be
instantiated.

\subsection{Satisfiability}
\label{sec:satisfiability}

This subsection contains a description of constraint set
satisfiability, including a discussion of decidability (taken from
\cite{Ambiguity-and-context-dependent-overloading}). Constraint set
satisfiability is in general an undecidable problem
\cite{Smith-PhD-Thesis91}. It is restricted here so that it becomes
decidable, as described below. The restriction is based on a measure
of constraints, a measure of the sizes of types in a constraint head,
given by a so-called constraint-head-value function. Essentially, the
sequence of constraints that unify with a constraint axiom in
recursive calls of the function that checks satisfiability of a type
constraint is such that either the sizes of types of each constraint
in this sequence is decreasing or there exists at least one type
parameter position with decreasing size.

The definition of the constraint-head-value function is based on the
use of a constraint value $\nu(\pi)$ that gives the number of
occurrences of type variables and type constructors in $\pi$:
  \[ \begin{array}{ll}
        \nu(C\: \overline{\tau}) & = \sum_{i=1}^n \nu(\tau_i)\\
        \nu(T)                   & = 1\\
        \nu(\alpha)              & = 1\\
        \nu(\tau\: \tau')        & = \nu(\tau) + \nu(\tau')
     \end{array}
  \]

Consider computation of satisfiability of a given constraint set $C$
with respect to program theory $P$ and consider that, during the
process of checking satisfiability of a constraint $\pi\in C$, a
constraint $\pi'$ unifies with the head of constraint $\forall\,
\overline{a}.C_0 \Rightarrow \pi_0$ in $P$, with unifying
substitution $\phi$. Then, for any constraint $\pi_1$ that, in this
process of checking satisfiability of $\pi$, also unifies with
$\pi_0$, where the corresponding unifying substitution is $\phi_1$,
the following is required, for satisfiability of $\pi$ to hold:

\begin{enumerate}
\item $\nu(\phi\,\pi')$ is less than $\nu(\phi_1\,\pi_1)$ or, if
  $\nu(\phi\, \pi')=\nu(\phi_1 \pi_1)$, then $\phi\,\pi' \not= \pi''$,
  for all $\pi''$ that has the same constraint value as $\pi'$ and has
  unified with $\pi_0$ in process of checking for satisfiability of
  $\pi$, or

\item $\nu(\phi\,\pi')$ is greater than $\nu(\phi_1\,\pi_1)$ but then
  there is a type argument position such that the number of type
  variables and constructors of constraints that unify with $\pi_0$ in
  this argument position decreases.

\end{enumerate}

\label{Phi0}
More precisely, constraint-head-value-function $\Phi$ associates a
pair $(I,\Pi)$ to each constraint in P, where $I$ is a tuple of
constraint values and $\Pi$ is a set of constraints. Let
$\Phi_0(\pi_0) = (I_0,\emptyset)$ for each constraint axiom
\mbox{$\forall\,\overline{a}.\,P_0 \Rightarrow\pi_0\in P$}, where
$I_0$ is a tuple of values filled with any value greater than
$\nu(\pi)$ for every constraint $\pi$ in the program theory;
decidability is guaranteed by defining the operation $\Phi[\pi_0,\pi]$
of updating $\Phi(\pi_0) = (I,\Pi)$ as follows, where $I = (v_0,
v_1,\ldots, v_n)$ and $\pi = C\,\overline{\tau}$:

\[ \Phi[\pi_0,\pi] = \left\{ \begin{array}{lll}
                                   \textit{Fail}  & & \text{if } v'_i = -1 \text{ for } i=0,\ldots,n \\
                                   \Phi'          & & \text{otherwise}
                             \end{array} \right.
\]
where $\begin{array}[t]{ll}
              \Phi' (\pi_0) &=  ((v'_0,v'_1,\ldots,v'_n), \Pi \cup \{ \pi \} ) \\
              \Phi' (x)     &= \Phi(x) \text{ for } x  \not= \pi_0
              \end{array}$

\[ \begin{array}{l}
   v'_0 = \left\{ \begin{array}{lll}
                \nu(\pi) & & \text{if } \nu(\pi) < v_0 \text{ or} \\
                          & & \nu(\pi) = v_0 \text{ and } \pi \not\in \Pi \\
                -1        & & \text{otherwise}
              \end{array} \right. \\ \\
   \text{for $i=1,\ldots,n$ }
   \hspace{.5cm} v'_i = \left\{ \begin{array}{lll}
                                           \nu(\tau_i) & & \text{if } \nu(\tau_i) < v_i \\
                                           -1           & & \text{otherwise}
                                        \end{array} \right. \\
   \end{array}
\]
$\satsUm\bigl(\pi,P,\Delta)$ is defined to hold if
\[ \Delta = \left\{ (\phi|_{\tv(\pi)},\phi C_0,\pi_0)\,\,
					\begin{array}{|l}
	                  \,\,(\forall\,\overline{a}.\,C_0 \Rightarrow \pi_0) \in P,\\
                  		\,\,\phi = \mguI(\pi = \pi_0) 
                  	\end{array} \right\}
  \]

The set of satisfying substitutions for $C$ with respect to the
program theory $P$ is given by $\mathbb{S}$, such that $C \sats
{P,\Phi_0} \mathbb{S}$ holds, as defined in Figure \ref{fig-tsat}.
The restriction $\phi|_V$ of $\phi$ to $V$ denotes the substitution
$\phi'$ such that $\phi'(a) = \phi(a)$ if $a\in V$, otherwise $a$.


\begin{figure}
  \[ \begin{array}{cc}
  	\displaystyle\frac{}{C \sats {P,\Fail} \emptyset} (\SatFailUm)
  		{} &
  	\displaystyle\frac{}{\emptyset \sats {P,\Phi} \{ id\}} (\SatEmptyUm)\\ \\ \\
	\multicolumn{2}{c}{
    \displaystyle\frac
    	{\begin{array}{l}
	     \{ \pi \} \sats {P,\Phi} \mathbb{S}_0 \\[.1cm]
	     \mathbb{S} = \{ \phi' \circ \phi \,\mid\, \phi \in \mathbb{S}_0,\, \phi' \in \mathbb{S}_1,\:
                             \phi(C) \sats {P,\Phi} \mathbb{S}_1 \}
	\end{array}}
      	{\{\pi\} \cup C \sats {P,\Phi} \mathbb{S}} (\SatConjUm)
	}\\ \\ \\
	\multicolumn{2}{c}{
	\displaystyle\frac
	{\begin{array}{l}
	     \satsUm(\pi, P,\Delta) \\[.1cm]
	     \mathbb{S} = \left\{  \phi'\circ \phi \,\,
	     				\begin{array}{|c}
	     					\,\,(\phi,D,\pi') \in \Delta,\, \phi' \in \mathbb{S}_0,\\
	     					\,\,D  \sats {P,\Phi[\pi',\phi\pi]} \mathbb{S}_0
	     				\end{array}\right\}
	 \end{array}}
	{\{\pi\} \sats {P,\Phi} \mathbb{S}} (\SatInstUm) }
      \end{array} \]
\caption{Decidable Constraint Set Satisfiability}
\label{fig-tsat}
\end{figure}

The following examples illustrate the definition of constraint set
satisfiability as defined in Figure~\ref{fig-tsat}.  Let $\Phi(\pi).I$
and $\Phi(\pi).\Pi$ denote the first and second components of
$\Phi(\pi)$, respectively, and $v_i$ the $i$-th component of a tuple
of constraint values $I$: 

\begin{Example}
\label{EqL}
{\rm Consider satisfiability of $\pi = \text{{\tt {\it
        Eq\/}[[\I]]}}$ in $P = \{ \text{\it Eq \I\/},\: \forall\,
  a.\,\text{\it Eq\/}\: a \Rightarrow \text{{\tt {\it Eq\/}[$a$]}}
  \}$, letting $\pi_0 = \text{{\tt {\it Eq\/}[$a$]}}$; we have:

  \[ \displaystyle\frac{
         \begin{array}{l}
            \satsUm (\pi,P,
               \{ \bigl( \phi |_\emptyset,
                         \{ \text{\tt{\Eq[\I]}} \}, \pi_0\bigr) \}), \:\: \phi = [a_1\mapsto \text{\tt [\I]}]\\
               \mathbb{S}_0 = \{ \phi_1\circ \id \mid \: \phi_1 \in \mathbb{S}_1,\:\:\:
                \text{\tt{\Eq[\I]}} \sats {P,\Phi_1} \mathbb{S}_1\}
         \end{array}}
      {\pi \sats {P,\Phi_0} \mathbb{S}_0} (\SatInstUm)
  \]
where $\Phi_1 = \Phi_0[\pi_0,\pi]$, which implies that $\Phi_1(\pi_0) = ((3, 3), \{ \pi \} )$,  since
      $\nu(\pi) = 3$, and
      $a_1$ is a fresh type variable; then:
  \[ \displaystyle\frac{
         \begin{array}{l}
            \satsUm(\text{\tt {\it Eq\/}[\I]},P,
              \{\bigl(\phi'|_\emptyset,
                     \{ \text{{\tt {\it Eq\/}}}\,\I \}, \pi_0\bigr)\}), \:\: \phi' = [a_2\mapsto \I]\\
            \mathbb{S}_1 = \{ \phi_2\circ \id \mid \: \phi_2 \in \mathbb{S}_2,\:\:\:
             \text{\it Eq\/}\,\I \sats {P,\Phi_2} \mathbb{S}_2\}
         \end{array}}
      {\text{{\tt {\it Eq\/}[\I]}} \sats {P,\Phi_1} \mathbb{S}_1} (\SatInstUm)
  \]
where $\Phi_2 = \Phi_1[\pi_0,\text{\tt {\it Eq\/}[\I]}]$, which implies that
      $\Phi_2(\pi_0) = ((2,2), \Pi_2)$, with $\Pi_2 = \{ \pi, \text{\tt{\it Eq\/}[\I]}  \}  )$, since
      $\nu(\text{\tt{\it Eq\/}[\I]}) = 2$ is less than
       $\Phi_1(\pi_0).I.v_0 = 3$; then:
  \[ \displaystyle\frac{
         \begin{array}{l}
            \satsUm\bigl(\text{\it Eq\/}\,\I,P, \{ (\id, \emptyset, \text{\it Eq\/}\,\I ) \}\bigr)\\
            \mathbb{S}_2 = \{ \phi_3\circ \id \mid \: \phi_3 \in \mathbb{S}_3,\:\:\:
            \emptyset \sats {P,\Phi_3} \mathbb{S}_3\}
         \end{array}}
      {\text{\it Eq\/}\,\I \sats {P,\Phi_2} \mathbb{S}_2} (\SatInstUm)
  \]
where $\Phi_3 = \Phi_2[\text{\it Eq\/}\,\I,\text{\it Eq\/}\,\I]$
      and $\mathbb{S}_3 = \{ \id\}$ by ($\text{\tt SEmpty}_1$).}
\end{Example}

The following illustrates a case of satisfiability involving a
constraint $\pi'$ that unifies with a constraint head $\pi_0$ such
that $\nu(\pi')$ is greater than the upper bound associated to
$\pi_0$, which is the first component of $\Phi(\pi_0).I$.

\begin{Example}
\label{sat-eta-greater} {\rm

Consider satisfiability of $\pi=A\,\I\,(T^3\,\I)$ in program theory $P
= \{ A\,(T\,a)\,\I, \forall\, a,b.\,A\,(T^2\, a)\,b \Rightarrow
A\,a\,(T\,b)\}$. We have, where $\pi_0 = A\,a\,(T\,b)$:

\[
	\displaystyle\frac
		{\begin{array}{c}
            \satsUm\bigl(\pi,P,\{ ( \phi\,|_\emptyset, \{ \pi_1 \}, \pi_0 ) \}\bigr) \\
            \phi = [a_1\mapsto \I, b_1\mapsto T^2\:\I] \\
            \pi_1 = A\:(T^2\,\I)\:(T^2\,\I)\\
            \mathbb{S}_0 = \{ \phi_1\circ \id \mid \phi_1 \in \mathbb{S}_1,\:\:\:
            \pi_1 \sats {P,\Phi_1} \mathbb{S}_1\}
         \end{array}}
		{\pi \sats {P,\Phi_0} \mathbb{S}_0} (\SatInstUm)
\]
where $\Phi_1 = \Phi_0 [\pi_0, \pi]$, which implies that $\Phi_1(\pi_0).I = (5,1,4)$; then:

\[
	\displaystyle\frac
		{\begin{array}{c}
            \satsUm\bigl(\pi_1,P, \{ ( \phi'\,|_\emptyset, \{\pi_2\}, \pi_0 ) \}\bigr)\\
            \phi' = [a_2\mapsto T^2\,\I, b_2\mapsto T\,\I] \\
	    \pi_2 = A\:(T^4\,\I)\:(T\,\I)\\
            \mathbb{S}_1 = \{ \phi_2\circ [a_1\mapsto T^2\,a_2] \mid \phi_2 \in \mathbb{S}_2,\:\:
            \pi_2 \sats {P,\Phi_2} \mathbb{S}_2\}
         \end{array}}
		{\pi_1 \sats {P,\Phi_1} \mathbb{S}_1} (\SatInstUm)
\]
where  $\Phi_2 = \Phi_1 [\pi_0, \pi_1]$.
Since $\nu(\pi_1) = 6 > 5 = \Phi_1(\pi_0).I.v_0$,
we have that $\Phi_2(\pi_0).I = (-1,-1,3)$.

Again, $\pi_2$ unifies with $\pi_0$, with unifying substitution
$\phi' =  [a_3\mapsto T^4\,\I, b_2\mapsto \I] $, and
updating $\Phi_3 = \Phi_2[\pi_0,\pi_2]$ gives $\Phi_3(\pi_0).I = (-1,-1,2)$.
Satisfiability is then finally tested for $\pi_3 = A\,(T^6\,\I) \I$, that unifies with
$A\,(T\,a)\,\I$, returning $\mathbb{S}_3 = \{ [a_3\mapsto
  T^5\,\I]|_\emptyset\} = \{ \id\}$.  Constraint $\pi$ is thus
satisfiable, with $\mathbb{S}_0 = \{\id\}$.}
\end{Example}

The following example illustrates a case where the information kept in
the second component of $\Phi(\pi_0)$ is relevant.

\begin{Example}
\label{Paterson-condition-failure-example}
{\rm Consider the satisfiability of $\pi = A\,(T^2\,\I)\,\F$ in
  program theory $P = \{ A\,\I\,(T^2\,\F), \forall\,a,b.\,A\,a\,(T\,b)
  \Rightarrow A\,(T\,a)\,b\}$ and let $\pi_0 = A\,(T\,a)\,b$. Then:}

\[
	\displaystyle\frac
		{\begin{array}{c}
            \satsUm(\pi,P,\{ \bigl( \phi\,|_\emptyset, \{ \pi_1 \}, \pi_0 \bigr) \}) \\
            \phi = [a_1\mapsto (T\,\I), b_1 \mapsto \F] \\ \pi_1 = A\,(T\:\I)\,(T\:\F)\\
            \mathbb{S}_0 = \{ \phi_1\circ \id \mid \: \phi_1 \in \mathbb{S}_1,\:\:\:
                                                \pi_1 \sats {P,\Phi_1} \mathbb{S}_1\}
         \end{array}}
		{\pi \sats {P,\Phi_0} \mathbb{S}_0} (\SatInstUm)
\]
{\rm where $\Phi_1 = \Phi_0[\pi_0,\pi]$, giving $\Phi_1(\pi_0) = ((4,3,1), \{ \pi \})$; then:
\[
	\displaystyle\frac
		{\begin{array}{c}
            \satsUm(\pi_1,P,\{ \bigl( \phi'\,|_\emptyset, \{ \pi_2 \}, \pi_0 \bigr) \})\\
            \phi' = [a_2\mapsto \text{\tt \I}, b_2 \mapsto T\,\F],\,\,\,\,\,\,\, \pi_2 = A\,\I\, (T^2\,\F)\\
            \mathbb{S}_1 = \{ \phi_2\circ \id \mid \: \phi_2 \in \mathbb{S}_2,\:\:\:
            \pi_2 \sats {P,\Phi_2} \mathbb{S}_2\}
         \end{array}}
		{\pi_1 \sats {P,\Phi_1} \mathbb{S}_1} (\SatInstUm)
\]
where $\Phi_2 = \Phi_1[\pi_0,\pi_1]$. Since
      $\nu(\pi_1) = 4$, which is equal to the first component of $\Phi_1(\pi_0).I$,
      and
      $\pi_1$ is not in $\Phi_1(\pi_0).\Pi$, we obtain that
 $\mathbb{S}_2 = \{ \id \}$ and $\pi$ is thus satisfiable
 (since $\satsUm(A\,\I\,(T^2\,\F),P) =
   \{ (\id, \emptyset, A\,\I\,(T^2\,\F)\}$). }
\end{Example}

Since satisfiability of type class constraints is in general
undecidable \cite{Smith-PhD-Thesis91}, there exist satisfiable
instances which are considered to be unsatisfiable according to the
definition of Figure \ref{fig-tsat}. Examples can be constructed by
encoding instances of solvable Post Correspondence problems by means
of constraint set satisfiability, using G.~Smith's scheme
\cite{Smith-PhD-Thesis91}.

%\begin{example}\label{PCP-Example}
%{\rm This example uses a PCP instance taken from
%\cite{Ling-Zhao-Master-Thesis}. A PCP instance can be defined as
%composed of pairs of strings, each pair having a top and a bottom
%string, where the goal is to select a sequence of pairs such that the
%two strings obtained by concatenating top and bottom strings in such
%pairs are identical. The example uses three pairs of strings: $p_1 =
%(\text{\tt{100}}, \text{\tt{1}})$ (that is, pair 1 has string {\tt
%  100} as the top string and {\tt 1} as the bottom string), $p_2 =
%(\text{{\tt 0}}, \text{\tt{100}})$ and $p_3 =
%(\text{\tt{1}},\text{\tt{00}})$.}

%{\rm This instance has a solution: using numbers to represent corresponding
%pairs (i.e. {\tt 1} represents pair 1 and analogously for {\tt 2} and
%{\tt 3}), the sequence of pairs {\tt 1311322} is a solution.}

%{\rm A satisfiability problem that has a solution if and only if the
%  PCP instance has a solution can be constructed by adapting
%  G.~Smith's scheme to Haskell's notation. We consider for this a
%  two-parameter class $C$, and a constraint context such that $\Theta
%  = \Theta_1 \cup \Theta_2 \cup \Theta_3$, where $\Theta_i$ is
%  constructed from pair $i$, for $i=1,2,3$:}
%  \[ \begin{array}{l}
%     \Theta_1 = \{ \begin{array}[t]{l}
%                     C\, (1 \rightarrow 0 \rightarrow 0) \, 1, \\
%                     \forall\, a,b.\, C\, a\, b \Rightarrow
%                                      C\, (1 \rightarrow 0 \rightarrow 0 \rightarrow a) \, (1 \rightarrow b)\: \}
%                   \end{array} \\
%     \Theta_2 = \{ \begin{array}[t]{l}
%                    C\, 0\, (1 \rightarrow 0 \rightarrow 0), \\
%                    \forall\, a,b.\, C\, a\, b \Rightarrow
%                    C\, (0 \rightarrow a) \,  (1 \rightarrow 0 \rightarrow 0 \rightarrow b)\: \}
%                    \end{array} \\
%     \Theta_3 = \{ \begin{array}[t]{l}
%                     C\, 1\: (0 \rightarrow 0), \\
%                    \forall\, a,b.\, C\, a\, b \Rightarrow
%                    C\, (1 \rightarrow a)\, (0 \rightarrow 0 \rightarrow b)\: \}
%                    \end{array}
%     \end{array}
%\]
%{\rm We have that constraint $C\:a\:a$ is satisfiable, with a solution
%constructed from solution {\tt 1311322} of the PCP
%instance. Computation by our algorithm terminates, erroneously
%reporting unsatisfiability. The steps of the computation are
%omitted. The computation proceeds until {\tt 131}, when updating parameter
%$\Phi$ results in {\it Fail}.}
%\end{example}

To prove that satisfiability as defined in Figure~\ref{fig-tsat} is
decidable, consider that there exist finitely many constraints in
program theory $P$, and that, for any constraint $\pi$ that unifies
with $\pi_0$, we have, by the definition of $\Phi[\pi_0,\pi]$, that
$\Phi(\pi_0)$ is updated so as to include a new value in its second
component (otherwise $\Phi[\pi_0,\pi] = \text{\it Fail\/}$ and
satisfiability yields $\emptyset$ as the set of satisfying
substitutions for the original constraint). The conclusion follows
from the fact that $\Phi(\pi_0)$ can have only finitely many distinct
values, for any $\pi_0$.


