\subsubsection{Context Reduction}
\label{sec:context-reduction}

Context reduction is a process that reduces a constraint $\pi$ into
constraint set $D$ according to a {\it matching instance\/} for $\pi$
in a program theory $P$: if there exists
$(\forall\,\overline{\alpha}.\,C\Rightarrow \pi')\in P$ such that
$\phi(\pi') = \pi$, for some $\phi$ such that $\phi(C)$ reduces to
$D$; if there is no matching instance for $\pi$ or no reduction of
$\phi(C)$ is possible, then $\pi$ reduces to a constraint set
containing only itself.

%(in Haskell-terminology, $P$ is called the context and $\pi$ the head
%of constrained type $P\Rightarrow \pi$).

As an example of a context reduction, consider an instance declaration
that introduces $\forall a.\,{\it Eq\/}\, a \Rightarrow \text{\tt {\it
    Eq\/}[$a$]}$ in program theory $P$; then {\tt {\it Eq\/}[$a$]} is
reduced to {\it Eq\/} $a$.

Context reduction can also occur due to the presence of superclass
class declarations, but we only consider the case of instance
declarations in this paper, which is the more complex process. The
treatment of reducing constraints due to the existence of superclasses
is standard; see e.g.~\cite{MarkJones94a,Hall96,Faxen2002}.

Context reduction uses $\matches$, defined as follows:
  \[ \begin{array}{l}
        \matches\bigl(\pi,(P,\Phi'), \Delta) \text{ holds if }\\
          \:\:\:\: \Delta = \left\{ (\phi(C_0), \pi_0, \Phi')\,\,
          						\begin{array}{|l}
                             		\,\,(\forall\,\overline{\alpha}.\,C_0\Rightarrow \pi_0) \in P,\\
                             		\,\,\mgm(\pi_0 = \pi,\phi),\, \Phi' = \Phi[\pi_0,\pi]
                             	\end{array}\right\}
     \end{array}
  \]
where $\mgm$ is analogous to $\mgu$ but denotes the most general
matching substitution, instead of the most general unifier.

The third parameter of $\matches$ is either empty or a singleton set,
since overlapping instances \cite{ghc-users-guide} are not considered.

Context reduction, defined in Figure~\ref{Context-reduction-fig}, uses
rules of the form $C \contextreduces {P,\Phi} D;\Phi'$, meaning that
either $C$ reduces to $D$ under program theory $P$ and least
constraint value function $\Phi$, causing $\Phi$ to be updated to
$\Phi'$, or $C \contextreduces {P,\Fail} C;\Fail$. Failure is used to
define a reduction of a constraint set to itself.

The least constraint value function is used as in the definition of
{\it sats\/} to guarantee that context reduction is a decidable
relation.

\begin{figure}

  \[ \begin{array}{c}
       \begin{array}{cc}
         \displaystyle\frac{}
                           {\emptyset \contextreduces {P,\Phi} \emptyset;\Phi} (\redo) &
         \displaystyle\frac{\{ \pi \} \contextreduces {P,\Phi} C;\Phi_1  \:
                            D \contextreduces {P,\Phi_1} D';\Phi'}
      	                   {\{ \pi \} \cup D \contextreduces {P,\Phi} C \cup D';\Phi'} (\conj)
      \end{array}\\[.8cm]

       \displaystyle\frac{\matches\bigl(\pi,(P,\Phi),\{(C,\pi',\Phi')\}\bigr)
                           \:\:\: C \contextreduces {P,\Phi'} D;\Phi''}
       	                 {\{ \pi \} \contextreduces {P,\Phi} D;\Phi''}\: (\instum)\\[.8cm]
       \displaystyle\frac{\matches\bigl(\pi,(P,\Phi),\{(C,\pi',\Phi')\}\bigr) \:\:\:
                            C \contextreduces {P,\Phi'} D;\Fail}
       	                 {\{ \pi \} \contextreduces {P,\Phi} \{ \pi \}; \Fail} \: (\stopFail) \\[.8cm]
       \begin{array}{c}
       \displaystyle\frac{\matches\bigl(\pi,(P,\Phi),\{(C,\pi',\Fail)\}\bigr)}
       	                 {\{ \pi \} \cup C \contextreduces {P,\Phi} \{ \pi \}\cup C;\Fail} \: (\stopo)
       \end{array}
    \end{array}
  \]
\caption{Context Reduction}
\label{Context-reduction-fig}
\end{figure}

An empty constraint set reduces to itself ($\redo$).  Rule ($\conj$)
specifies that constraint set simplification works, unlike constraint
set satisfiability, by performing a union of the result of
simplifying separately each constraint in the constraint set.
To see that a rule similar to ($\conj$) cannot be used in the case of
constraint set satisfiability, consider a simple example, of
satisfiability of $C = \{A\:a, B\: a\}$ in $P = \{A\:\Int,A\:
\Bool,B\: \Int,B\: \Char\}$. Satisfiability of $C$ yields a single
substitution where $a$ maps to $\Int$, not the union of computing
satisfiability for $A\:a$ and $B\:a$ separately.

Rule ($\instum$) specifies that if there exists a constraint axiom
$\forall\,\overline{\alpha}.\,C \Rightarrow A\,\overline{\tau}$, such
that $A\,\overline{\tau}$ matches with an input constraint $\pi$, then
$\pi$ reduces to any constraint set $D$ that $C$ reduces to.

Rules ($\stopFail$) and ($\stopo$) deal with failure due to updating
of the constraint-head-value function.
