\subsection{Satisfiability}
\label{sec:satisfiability}

This subsection contains a description of constraint set
satisfiability, including a discussion of decidability (taken from
\cite{Ambiguity-and-context-dependent-overloading}). Constraint set
satisfiability is in general an undecidable problem
\cite{Smith-PhD-Thesis91}. It is restricted here so that it becomes
decidable, as described below. The restriction is based on a measure
of constraints, a measure of the sizes of types in a constraint head,
given by a so-called constraint-head-value function. Essentially, the
sequence of constraints that unify with a constraint axiom in
recursive calls of the function that checks satisfiability of a type
constraint is such that either the sizes of types of each constraint
in this sequence is decreasing or there exists at least one type
parameter position with decreasing size.

The definition of the constraint-head-value function is based on the
use of a constraint value $\nu(\pi)$ that gives the number of
occurrences of type variables and type constructors in $\pi$:
  \[ \begin{array}{ll}
        \nu(C\: \overline{\tau}) & = \sum_{i=1}^n \nu(\tau_i)\\
        \nu(T)                   & = 1\\
        \nu(\alpha)              & = 1\\
        \nu(\tau\: \tau')        & = \nu(\tau) + \nu(\tau')
     \end{array}
  \]

Consider computation of satisfiability of a given constraint set $C$
with respect to program theory $P$ and consider that, during the
process of checking satisfiability of a constraint $\pi\in C$, a
constraint $\pi'$ unifies with the head of constraint $\forall\,
\overline{a}.C_0 \Rightarrow \pi_0$ in $P$, with unifying
substitution $\phi$. Then, for any constraint $\pi_1$ that, in this
process of checking satisfiability of $\pi$, also unifies with
$\pi_0$, where the corresponding unifying substitution is $\phi_1$,
the following is required, for satisfiability of $\pi$ to hold:

\begin{enumerate}
\item $\nu(\phi\,\pi')$ is less than $\nu(\phi_1\,\pi_1)$ or, if
  $\nu(\phi\, \pi')=\nu(\phi_1 \pi_1)$, then $\phi\,\pi' \not= \pi''$,
  for all $\pi''$ that has the same constraint value as $\pi'$ and has
  unified with $\pi_0$ in process of checking for satisfiability of
  $\pi$, or

\item $\nu(\phi\,\pi')$ is greater than $\nu(\phi_1\,\pi_1)$ but then
  there is a type argument position such that the number of type
  variables and constructors of constraints that unify with $\pi_0$ in
  this argument position decreases.

\end{enumerate}

\label{Phi0}
More precisely, constraint-head-value-function $\Phi$ associates a
pair $(I,\Pi)$ to each constraint in P, where $I$ is a tuple of
constraint values and $\Pi$ is a set of constraints. Let
$\Phi_0(\pi_0) = (I_0,\emptyset)$ for each constraint axiom
\mbox{$\forall\,\overline{a}.\,P_0 \Rightarrow\pi_0\in P$}, where
$I_0$ is a tuple of values filled with any value greater than
$\nu(\pi)$ for every constraint $\pi$ in the program theory;
decidability is guaranteed by defining the operation $\Phi[\pi_0,\pi]$
of updating $\Phi(\pi_0) = (I,\Pi)$ as follows, where $I = (v_0,
v_1,\ldots, v_n)$ and $\pi = C\,\overline{\tau}$:

\[ \Phi[\pi_0,\pi] = \left\{ \begin{array}{lll}
                                   \textit{Fail}  & & \text{if } v'_i = -1 \text{ for } i=0,\ldots,n \\
                                   \Phi'          & & \text{otherwise}
                             \end{array} \right.
\]
where $\begin{array}[t]{ll}
              \Phi' (\pi_0) &=  ((v'_0,v'_1,\ldots,v'_n), \Pi \cup \{ \pi \} ) \\
              \Phi' (x)     &= \Phi(x) \text{ for } x  \not= \pi_0
              \end{array}$

\[ \begin{array}{l}
   v'_0 = \left\{ \begin{array}{lll}
                \nu(\pi) & & \text{if } \nu(\pi) < v_0 \text{ or} \\
                          & & \nu(\pi) = v_0 \text{ and } \pi \not\in \Pi \\
                -1        & & \text{otherwise}
              \end{array} \right. \\ \\
   \text{for $i=1,\ldots,n$ }
   \hspace{.5cm} v'_i = \left\{ \begin{array}{lll}
                                           \nu(\tau_i) & & \text{if } \nu(\tau_i) < v_i \\
                                           -1           & & \text{otherwise}
                                        \end{array} \right. \\
   \end{array}
\]
$\satsUm\bigl(\pi,P,\Delta)$ is defined to hold if
\[ \Delta = \left\{ (\phi|_{\tv(\pi)},\phi C_0,\pi_0)\,\,
					\begin{array}{|l}
	                  \,\,(\forall\,\overline{a}.\,C_0 \Rightarrow \pi_0) \in P,\\
                  		\,\,\phi = \mguI(\pi = \pi_0) 
                  	\end{array} \right\}
  \]

The set of satisfying substitutions for $C$ with respect to the
program theory $P$ is given by $\mathbb{S}$, such that $C \sats
{P,\Phi_0} \mathbb{S}$ holds, as defined in Figure \ref{fig-tsat}.
The restriction $\phi|_V$ of $\phi$ to $V$ denotes the substitution
$\phi'$ such that $\phi'(a) = \phi(a)$ if $a\in V$, otherwise $a$.


\begin{figure}
  \[ \begin{array}{cc}
  	\displaystyle\frac{}{C \sats {P,\Fail} \emptyset} (\SatFailUm)
  		{} &
  	\displaystyle\frac{}{\emptyset \sats {P,\Phi} \{ id\}} (\SatEmptyUm)\\ \\ \\
	\multicolumn{2}{c}{
    \displaystyle\frac
    	{\begin{array}{l}
	     \{ \pi \} \sats {P,\Phi} \mathbb{S}_0 \\[.1cm]
	     \mathbb{S} = \{ \phi' \circ \phi \,\mid\, \phi \in \mathbb{S}_0,\, \phi' \in \mathbb{S}_1,\:
                             \phi(C) \sats {P,\Phi} \mathbb{S}_1 \}
	\end{array}}
      	{\{\pi\} \cup C \sats {P,\Phi} \mathbb{S}} (\SatConjUm)
	}\\ \\ \\
	\multicolumn{2}{c}{
	\displaystyle\frac
	{\begin{array}{l}
	     \satsUm(\pi, P,\Delta) \\[.1cm]
	     \mathbb{S} = \left\{  \phi'\circ \phi \,\,
	     				\begin{array}{|c}
	     					\,\,(\phi,D,\pi') \in \Delta,\, \phi' \in \mathbb{S}_0,\\
	     					\,\,D  \sats {P,\Phi[\pi',\phi\pi]} \mathbb{S}_0
	     				\end{array}\right\}
	 \end{array}}
	{\{\pi\} \sats {P,\Phi} \mathbb{S}} (\SatInstUm) }
      \end{array} \]
\caption{Decidable Constraint Set Satisfiability}
\label{fig-tsat}
\end{figure}

The following examples illustrate the definition of constraint set
satisfiability as defined in Figure~\ref{fig-tsat}.  Let $\Phi(\pi).I$
and $\Phi(\pi).\Pi$ denote the first and second components of
$\Phi(\pi)$, respectively, and $v_i$ the $i$-th component of a tuple
of constraint values $I$: 

\begin{Example}
\label{EqL}
{\rm Consider satisfiability of $\pi = \text{{\tt {\it
        Eq\/}[[\I]]}}$ in $P = \{ \text{\it Eq \I\/},\: \forall\,
  a.\,\text{\it Eq\/}\: a \Rightarrow \text{{\tt {\it Eq\/}[$a$]}}
  \}$, letting $\pi_0 = \text{{\tt {\it Eq\/}[$a$]}}$; we have:

  \[ \displaystyle\frac{
         \begin{array}{l}
            \satsUm (\pi,P,
               \{ \bigl( \phi |_\emptyset,
                         \{ \text{\tt{\Eq[\I]}} \}, \pi_0\bigr) \}), \:\: \phi = [a_1\mapsto \text{\tt [\I]}]\\
               \mathbb{S}_0 = \{ \phi_1\circ \id \mid \: \phi_1 \in \mathbb{S}_1,\:\:\:
                \text{\tt{\Eq[\I]}} \sats {P,\Phi_1} \mathbb{S}_1\}
         \end{array}}
      {\pi \sats {P,\Phi_0} \mathbb{S}_0} (\SatInstUm)
  \]
where $\Phi_1 = \Phi_0[\pi_0,\pi]$, which implies that $\Phi_1(\pi_0) = ((3, 3), \{ \pi \} )$,  since
      $\nu(\pi) = 3$, and
      $a_1$ is a fresh type variable; then:
  \[ \displaystyle\frac{
         \begin{array}{l}
            \satsUm(\text{\tt {\it Eq\/}[\I]},P,
              \{\bigl(\phi'|_\emptyset,
                     \{ \text{{\tt {\it Eq\/}}}\,\I \}, \pi_0\bigr)\}), \:\: \phi' = [a_2\mapsto \I]\\
            \mathbb{S}_1 = \{ \phi_2\circ \id \mid \: \phi_2 \in \mathbb{S}_2,\:\:\:
             \text{\it Eq\/}\,\I \sats {P,\Phi_2} \mathbb{S}_2\}
         \end{array}}
      {\text{{\tt {\it Eq\/}[\I]}} \sats {P,\Phi_1} \mathbb{S}_1} (\SatInstUm)
  \]
where $\Phi_2 = \Phi_1[\pi_0,\text{\tt {\it Eq\/}[\I]}]$, which implies that
      $\Phi_2(\pi_0) = ((2,2), \Pi_2)$, with $\Pi_2 = \{ \pi, \text{\tt{\it Eq\/}[\I]}  \}  )$, since
      $\nu(\text{\tt{\it Eq\/}[\I]}) = 2$ is less than
       $\Phi_1(\pi_0).I.v_0 = 3$; then:
  \[ \displaystyle\frac{
         \begin{array}{l}
            \satsUm\bigl(\text{\it Eq\/}\,\I,P, \{ (\id, \emptyset, \text{\it Eq\/}\,\I ) \}\bigr)\\
            \mathbb{S}_2 = \{ \phi_3\circ \id \mid \: \phi_3 \in \mathbb{S}_3,\:\:\:
            \emptyset \sats {P,\Phi_3} \mathbb{S}_3\}
         \end{array}}
      {\text{\it Eq\/}\,\I \sats {P,\Phi_2} \mathbb{S}_2} (\SatInstUm)
  \]
where $\Phi_3 = \Phi_2[\text{\it Eq\/}\,\I,\text{\it Eq\/}\,\I]$
      and $\mathbb{S}_3 = \{ \id\}$ by ($\text{\tt SEmpty}_1$).}
\end{Example}

The following illustrates a case of satisfiability involving a
constraint $\pi'$ that unifies with a constraint head $\pi_0$ such
that $\nu(\pi')$ is greater than the upper bound associated to
$\pi_0$, which is the first component of $\Phi(\pi_0).I$.

\begin{Example}
\label{sat-eta-greater} {\rm

Consider satisfiability of $\pi=A\,\I\,(T^3\,\I)$ in program theory $P
= \{ A\,(T\,a)\,\I, \forall\, a,b.\,A\,(T^2\, a)\,b \Rightarrow
A\,a\,(T\,b)\}$. We have, where $\pi_0 = A\,a\,(T\,b)$:

\[
	\displaystyle\frac
		{\begin{array}{c}
            \satsUm\bigl(\pi,P,\{ ( \phi\,|_\emptyset, \{ \pi_1 \}, \pi_0 ) \}\bigr) \\
            \phi = [a_1\mapsto \I, b_1\mapsto T^2\:\I] \\
            \pi_1 = A\:(T^2\,\I)\:(T^2\,\I)\\
            \mathbb{S}_0 = \{ \phi_1\circ \id \mid \phi_1 \in \mathbb{S}_1,\:\:\:
            \pi_1 \sats {P,\Phi_1} \mathbb{S}_1\}
         \end{array}}
		{\pi \sats {P,\Phi_0} \mathbb{S}_0} (\SatInstUm)
\]
where $\Phi_1 = \Phi_0 [\pi_0, \pi]$, which implies that $\Phi_1(\pi_0).I = (5,1,4)$; then:

\[
	\displaystyle\frac
		{\begin{array}{c}
            \satsUm\bigl(\pi_1,P, \{ ( \phi'\,|_\emptyset, \{\pi_2\}, \pi_0 ) \}\bigr)\\
            \phi' = [a_2\mapsto T^2\,\I, b_2\mapsto T\,\I] \\
	    \pi_2 = A\:(T^4\,\I)\:(T\,\I)\\
            \mathbb{S}_1 = \{ \phi_2\circ [a_1\mapsto T^2\,a_2] \mid \phi_2 \in \mathbb{S}_2,\:\:
            \pi_2 \sats {P,\Phi_2} \mathbb{S}_2\}
         \end{array}}
		{\pi_1 \sats {P,\Phi_1} \mathbb{S}_1} (\SatInstUm)
\]
where  $\Phi_2 = \Phi_1 [\pi_0, \pi_1]$.
Since $\nu(\pi_1) = 6 > 5 = \Phi_1(\pi_0).I.v_0$,
we have that $\Phi_2(\pi_0).I = (-1,-1,3)$.

Again, $\pi_2$ unifies with $\pi_0$, with unifying substitution
$\phi' =  [a_3\mapsto T^4\,\I, b_2\mapsto \I] $, and
updating $\Phi_3 = \Phi_2[\pi_0,\pi_2]$ gives $\Phi_3(\pi_0).I = (-1,-1,2)$.
Satisfiability is then finally tested for $\pi_3 = A\,(T^6\,\I) \I$, that unifies with
$A\,(T\,a)\,\I$, returning $\mathbb{S}_3 = \{ [a_3\mapsto
  T^5\,\I]|_\emptyset\} = \{ \id\}$.  Constraint $\pi$ is thus
satisfiable, with $\mathbb{S}_0 = \{\id\}$.}
\end{Example}

The following example illustrates a case where the information kept in
the second component of $\Phi(\pi_0)$ is relevant.

\begin{Example}
\label{Paterson-condition-failure-example}
{\rm Consider the satisfiability of $\pi = A\,(T^2\,\I)\,\F$ in
  program theory $P = \{ A\,\I\,(T^2\,\F), \forall\,a,b.\,A\,a\,(T\,b)
  \Rightarrow A\,(T\,a)\,b\}$ and let $\pi_0 = A\,(T\,a)\,b$. Then:}

\[
	\displaystyle\frac
		{\begin{array}{c}
            \satsUm(\pi,P,\{ \bigl( \phi\,|_\emptyset, \{ \pi_1 \}, \pi_0 \bigr) \}) \\
            \phi = [a_1\mapsto (T\,\I), b_1 \mapsto \F] \\ \pi_1 = A\,(T\:\I)\,(T\:\F)\\
            \mathbb{S}_0 = \{ \phi_1\circ \id \mid \: \phi_1 \in \mathbb{S}_1,\:\:\:
                                                \pi_1 \sats {P,\Phi_1} \mathbb{S}_1\}
         \end{array}}
		{\pi \sats {P,\Phi_0} \mathbb{S}_0} (\SatInstUm)
\]
{\rm where $\Phi_1 = \Phi_0[\pi_0,\pi]$, giving $\Phi_1(\pi_0) = ((4,3,1), \{ \pi \})$; then:
\[
	\displaystyle\frac
		{\begin{array}{c}
            \satsUm(\pi_1,P,\{ \bigl( \phi'\,|_\emptyset, \{ \pi_2 \}, \pi_0 \bigr) \})\\
            \phi' = [a_2\mapsto \text{\tt \I}, b_2 \mapsto T\,\F],\,\,\,\,\,\,\, \pi_2 = A\,\I\, (T^2\,\F)\\
            \mathbb{S}_1 = \{ \phi_2\circ \id \mid \: \phi_2 \in \mathbb{S}_2,\:\:\:
            \pi_2 \sats {P,\Phi_2} \mathbb{S}_2\}
         \end{array}}
		{\pi_1 \sats {P,\Phi_1} \mathbb{S}_1} (\SatInstUm)
\]
where $\Phi_2 = \Phi_1[\pi_0,\pi_1]$. Since
      $\nu(\pi_1) = 4$, which is equal to the first component of $\Phi_1(\pi_0).I$,
      and
      $\pi_1$ is not in $\Phi_1(\pi_0).\Pi$, we obtain that
 $\mathbb{S}_2 = \{ \id \}$ and $\pi$ is thus satisfiable
 (since $\satsUm(A\,\I\,(T^2\,\F),P) =
   \{ (\id, \emptyset, A\,\I\,(T^2\,\F)\}$). }
\end{Example}

Since satisfiability of type class constraints is in general
undecidable \cite{Smith-PhD-Thesis91}, there exist satisfiable
instances which are considered to be unsatisfiable according to the
definition of Figure \ref{fig-tsat}. Examples can be constructed by
encoding instances of solvable Post Correspondence problems by means
of constraint set satisfiability, using G.~Smith's scheme
\cite{Smith-PhD-Thesis91}.

%\begin{example}\label{PCP-Example}
%{\rm This example uses a PCP instance taken from
%\cite{Ling-Zhao-Master-Thesis}. A PCP instance can be defined as
%composed of pairs of strings, each pair having a top and a bottom
%string, where the goal is to select a sequence of pairs such that the
%two strings obtained by concatenating top and bottom strings in such
%pairs are identical. The example uses three pairs of strings: $p_1 =
%(\text{\tt{100}}, \text{\tt{1}})$ (that is, pair 1 has string {\tt
%  100} as the top string and {\tt 1} as the bottom string), $p_2 =
%(\text{{\tt 0}}, \text{\tt{100}})$ and $p_3 =
%(\text{\tt{1}},\text{\tt{00}})$.}

%{\rm This instance has a solution: using numbers to represent corresponding
%pairs (i.e. {\tt 1} represents pair 1 and analogously for {\tt 2} and
%{\tt 3}), the sequence of pairs {\tt 1311322} is a solution.}

%{\rm A satisfiability problem that has a solution if and only if the
%  PCP instance has a solution can be constructed by adapting
%  G.~Smith's scheme to Haskell's notation. We consider for this a
%  two-parameter class $C$, and a constraint context such that $\Theta
%  = \Theta_1 \cup \Theta_2 \cup \Theta_3$, where $\Theta_i$ is
%  constructed from pair $i$, for $i=1,2,3$:}
%  \[ \begin{array}{l}
%     \Theta_1 = \{ \begin{array}[t]{l}
%                     C\, (1 \rightarrow 0 \rightarrow 0) \, 1, \\
%                     \forall\, a,b.\, C\, a\, b \Rightarrow
%                                      C\, (1 \rightarrow 0 \rightarrow 0 \rightarrow a) \, (1 \rightarrow b)\: \}
%                   \end{array} \\
%     \Theta_2 = \{ \begin{array}[t]{l}
%                    C\, 0\, (1 \rightarrow 0 \rightarrow 0), \\
%                    \forall\, a,b.\, C\, a\, b \Rightarrow
%                    C\, (0 \rightarrow a) \,  (1 \rightarrow 0 \rightarrow 0 \rightarrow b)\: \}
%                    \end{array} \\
%     \Theta_3 = \{ \begin{array}[t]{l}
%                     C\, 1\: (0 \rightarrow 0), \\
%                    \forall\, a,b.\, C\, a\, b \Rightarrow
%                    C\, (1 \rightarrow a)\, (0 \rightarrow 0 \rightarrow b)\: \}
%                    \end{array}
%     \end{array}
%\]
%{\rm We have that constraint $C\:a\:a$ is satisfiable, with a solution
%constructed from solution {\tt 1311322} of the PCP
%instance. Computation by our algorithm terminates, erroneously
%reporting unsatisfiability. The steps of the computation are
%omitted. The computation proceeds until {\tt 131}, when updating parameter
%$\Phi$ results in {\it Fail}.}
%\end{example}

To prove that satisfiability as defined in Figure~\ref{fig-tsat} is
decidable, consider that there exist finitely many constraints in
program theory $P$, and that, for any constraint $\pi$ that unifies
with $\pi_0$, we have, by the definition of $\Phi[\pi_0,\pi]$, that
$\Phi(\pi_0)$ is updated so as to include a new value in its second
component (otherwise $\Phi[\pi_0,\pi] = \text{\it Fail\/}$ and
satisfiability yields $\emptyset$ as the set of satisfying
substitutions for the original constraint). The conclusion follows
from the fact that $\Phi(\pi_0)$ can have only finitely many distinct
values, for any $\pi_0$.

