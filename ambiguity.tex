\section{Ambiguity Rule}
\label{sec:ambiguity}

As mentioned in the introduction, Haskell considers ambiguity as a
property of a type, and does not distinguish it from a trigger
condition to check overloading resolution: a type
$\forall\,\overline{a}.\,C \Rightarrow \tau$ is ambiguous in Haskell
if there exists a type variable that occurs in the set of constraints
($C$) but does not occur in $\tau$.  In the presence of FDs or TFs,
``does not occur in $\tau$'' is modified in GHC to ``is not uniquely
determined from the set of type variables that occur in $\tau$''. This
unique determination is such that, for each type variable $a$ that
occurs in $C$ but not in $\tau$ there must exist a FD $b \mapsto a$
for some $b$ in $\tau$ (in the case of type familes, a similar unique
determination is specified). As mentioned the introduction, we use a
different trigger condition to check overloading, based on variable
reachability:

\begin{Definition}[Reachable Variable]

A variable $a\in \tv(C)$ is {\em reachable} from a set of type
variables $V$ if $a\in V$ or if $a\in \pi$ for some $\pi\in C$ such
that there exists $b\in \tv(\pi)$ such that $b$ is reachable. Type
variable $a\in \tv(C)$ is unreachable if it is not reachable.

The set of reachable type variables of constraint set $C$ {\em from
  V\/} is denoted by $\reachableVars(C,V)$.

The set of unreachable type variables of constraint set $C$ {\em from
  V\/} is denoted by $\unreachableVars(C,V)$.

\label{def:reachable}
\end{Definition}

For example, in $(A_1\: a\: b, A_2\: a) \Rightarrow b$, type variable
$a$ is reachable $\{b\}$, because $a$ occurs in constraint $A_1\: a\:
b$, and $b$ is reachable. Similarly, if $C = (A_1\: a\: b, A_2\: b\:
c, A_3\: c)$, then $c$ is reachable from $\{a\}$.

The presence of unreachable variables in a constraint $\pi\in C$, on a
type $C \Rightarrow \tau$, is a trigger condition for overloading
resolution (cf.~Theorem \ref{thm:overloading-resolution}, Section
\ref{Optional-type-classes}). This is motivated by the fact that there
is no context in which an expression with such a type could be placed
that could instantiate any of the unreachable variables occurring in
$\pi$.

The presence of unreachable variables does not necessarily imply
ambiguity. Ambiguity is a property of an expression, not of its
type. It depends on the context where the expression occurs, and on
\textit{entailment\/} of the constraints on the expression's type.
Also, by virtue of Haskell's {\em open-world\/} style of overloading,
ambiguity can be checked only when overloading is resolved, i.e. only
when there exist unreachable variables. When there are no unreachable
variables, overloading is yet unresolved.

%In Section \ref{sec:modular-instances} we consider a canonical example
%of ``ambiguous type'' in Haskell, namely {\tt (\SShow\ $a$,
%  \RRead\ $a$) => \String}; an expression $e$ with this type is not
%ambiguous if there exists a single constraint for \SShow, and a single
%constraint for \RRead, in the context where $e$ is used.

Entailment of constraints and its algorithmic (functional)
counterpart, satisfiability, are well-known in the Haskell world (see
e.g.~\cite{MarkJones94a,TheoryOfOverloading,JBCS-Ambiguity-and-constrained-polymorphism}).
For completeness, the definition of the entailment relation, used in
the type system (Section \ref{Optional-type-classes}), is defined in
Subsection \ref{sec:entailment}, and satisfability in 
\ref{sec:satisfiability}.

Informally, a set of constraints $C$ is entailed (or satisfied) in a
program $P$ if there exists a substitution $\phi$ such that $\phi(C)$
is contained in the set of instance declarations of $P$, or is
transitively implied by the set of class and instance declarations
occurring in
$P$~\cite{MarkJones94a,JBCS-Ambiguity-and-constrained-polymorphism}. In
this case we say that $C$ is entailed by $\phi$.

For example, {\tt \Eq\ [[\Integer]]} is entailed if we have instances
{\tt \Eq\ \Integer} and {\tt \Eq\ $a\,$=>$\,$\Eq\ [$a$]}, in the
context where an expression of a type with constraint {\tt
  \Eq\ [[\Integer]]} occurs.

When overloading resolution is checked for a constraint $\pi$
occurring in a constrained type $\delta = C \cup \{ \pi \} \Rightarrow
\tau$ --- in Haskell notation $(C, \pi) \Rightarrow \tau$ ---, one of
the following holds:
\begin{enumerate}

\item $\pi$ is entailed by a single instance. In this case a type
  simplification (also called ``improvement'') occurs: $\delta$ can be
  simplified to $C \Rightarrow \tau$.

\item $\pi$ is entailed by two or more instances. In this case we have
  a type error: ambiguity.

\item $\pi$ is not entailed (by any instance). In this case we have also
  a type error: unsatisfiability.

\end{enumerate}

Variables in a single constraint are either all reachable or all
unreachable:

\begin{Lemma}
  \label{lemma:unreachable-means-all-unreachable}
  For all $\pi, V$, we have that $\unreachableVars(\{\pi\},V)\not=\emptyset$
  if and only if $\unreachableVars(\{\pi\},V)=\tv(\pi)$.
\end{Lemma}

{\em Proof\/}: Directly from Definition \ref{def:reachable}. $\qed$

The possibility of a modular control of the visibility of instance
definitions conforms to this simplifying change. This is the subject
of Section \ref{sec:modular-instances}. 
